\par Mióta elkezdtem a pszichológia alapképzést 2016-ban az ELTE Pedagógiai és Pszichológiai Karán, hatalmas fejlődésen mentem keresztül nem csak szakmailag, de emberileg is. Egyetemi pályafutásom alatt számos olyan kurzuson vettem részt, melyek biztos elméleti és gyakorlati tudással ajándékoztak meg, s melyeket a későbbi tanulmányaim során is lehetőségem lesz hasznosítani. Többek közt megtanultam szakirodalmi összefoglalót írni, s ezáltal bepillantást nyerhettem a pszichológiai kutatások szerteágazó s mégis oly összetett világába. A teljes műhelymunkám elvégzése közben pedig alkalmam nyílt elsőkézből is megtapasztalni, hogy milyen komplex és izgalmas folyamat egy kutatás lebonyolítása, a tervezéstől az adatfelvételen át, az elemzésig. Az alapképzés alatt négy műhelymunkát készítettem, melyből három rész műhelymunka, s egy teljes műhelymunka volt. 

\subsubsection*{Személyiségpszichológia műhelymunka}

\par A legelső rész műhelymunkámat a boldogság-averzió, boldogságtól való félelem témában írtam, Bányai-Nagy Henriett segítségével. A kurzus során elsajátítottam a szakirodalmi összefoglaló írásának alapjait. Megtanultam a megfelelő empirikus források felkeresésének és használatának fortélyait, valamint elsajátítottam az azonos témájú, szerteágazó kutatások összefoglalásának képességét. A kurzusnak köszönhetően célzottan tudok szakirodalmat keresni, s mondandómat képes vagyok megfelelő nyelvezettel megfogalmazni, empirikus adatokkal alátámasztani.

\subsubsection*{Fejlődés-lélektani műhelymunka}

\par A hatodik félévemben három műhelymunka megírására is vállalkoztam. Ezek közül az egyik, Oláh Katalin vezetésével készült el, s témája a halláskárosult gyermekek tudatelmélete volt. A témában számos cikket, kutatást, illetve vizsgálatot kellett áttekintenem, hogy megfelelő képet tudjak adni a jelenlegi, tudományos álláspontról. A kurzus során tovább fejleszthettem az igényes szakirodalmi összefoglaló megírására vonatkozó képességeimet, s rengeteg új információt sajátítottam el a témában. 

\subsubsection*{Általános pszichológia műhelymunka}

\par A harmadik rész műhelymunkámat Honbolygó Ferenc segítségével írtam meg, mely szintén szakirodalmi összefoglaló volt, s témája a zeneterápia hatása volt, autista gyermekek szociális képességeire. A témában folytatott alapos tájékozódás során nem csak rengeteg új információt tanulhattam meg a zeneterápiáról s annak hatásairól, de hobbi zenészként különösen érdekesnek találtam ezt a fajta intervenciót. A rész műhelymunka megírása során megtanultam hogyan is kell esszészerűen összefoglalni az adott témán belüli kutatásokat, egy vázlatos, darabos leírás helyett, melynek hatalmas hasznát vettem a nagy műhelymunka megírása során.

\subsubsection*{Szociálpszichológia műhelymunka}

\par A teljes műhelymunkámat Szekeres Hanna Flóra PhD hallgató segítségével írtam meg, mely a heteroszexuális személyek homoszexuális férfiakkal szembeni attitűdjét és viselkedését vizsgálta. Egészen pontosan arra voltam kíváncsi, hogy azok a heteroszexuális személyek, akiknek több meleg barátja van, vajon pozitívabb attitűddel is rendelkeznek-e a csoport felé, illetve valószínűbben konfrontálják-e a csoporttal szembeni diszkriminációt. A kurzus során megtapasztaltam milyen lebonyolítani egy kutatást a kutatási kérdés megfogalmazásától, az adatfelvételen, kódoláson és elemzésen át, az eredmények publikálásáig. A vizsgálat során szembekerültem az adatfelvétel nehézségeivel, illetve rutint szerezhettem a különböző mérőeszközök kialakításában is, s mindeközben felelevenítettem az SPSS tudásomat. A műhelymunka megírása során megtanultam hogyan illesszem be az általam végzett vizsgálatot egy általános elméleti keretbe, s hasonlítsam össze saját vizsgálatom eredményeit másokéval. A vizsgálat végeredményére különösen büszke vagyok, hiszen egyrészt, a magyar mintán végzett hasonló kutatások száma igen limitált, másrészt olyan eredményeket kaptunk, melyek egy teljesen új ágát térképezik fel a témának, s egy kevésbé ismert terület szakirodalmát bővítik. A vizsgálatunk során mi ugyanis valós viselkedéses helyzeteket alkalmaztunk, vignettek, illetve önbeszámolók helyett, s így az eredményeink is teljesen mások lettek, mint amire a szakirodalmi összegzés alapján számítottunk. Úgy gondolom kifejezetten hasznos volt számomra a kurzus, hiszen olyan új készségeket sajátíthattam el általa, melyek nem csak a jövőbeli tanulmányaim során, de a karrieremben is hasznomra válhatnak.
\\
\par Összességében úgy gondolom, hogy a négy műhelymunkám által elsajátítottam az alapvető készségeket egy önálló kutatás megtervezéséhez s kivitelezéséhez. Többek közt megtanultam hogyan írjak tudományosan igényes szakirodalmi összefoglalót, hogyan vegyek fel, illetve elemezzek adatokat, állítsak fel hipotéziseket, illetve hogyan alkalmazzam az American Psychological Association (APA) által előírt formát. Tisztában vagyok a kutatási eredményeket torzító tényezőkkel, csapdákkal, veszélyforrásokkal és tudom hogyan kerüljem el őket. Megfelelő tudással rendelkezem a statisztikai elemzőprogramok használatát illetően, s nem jelent gondot számomra az angol nyelvű szakirodalom megértése sem. Mindazonáltal megtanultam, hogyan nézzem kritikus szemmel a saját kutatási tervemet, ötleteimet, s ezáltal kiküszöböltem számos módszertani hibát. 

\subsubsection*{További megszerzett kompetenciák}

\par Az ELTE Pedagógiai és Pszichológiai Karán eltöltött három évem alatt számos olyan kurzuson vettem részt, melyek megerősítettek abban, hogy jól döntöttem, amikor ezt a szakot és egyetemet választottam. Magyaródi Tímea Autogén Tréning gyakorlata például nem csak feledhetetlen saját élményt nyújtott számomra, de emellett olyan önszuggesztiós technikával ruházott fel, melyet hatékonyan tudok alkalmazni a mindennapjaimban. A Konfliktuskezelés és asszertív kommunikáció című kurzus által pedig lehetőségem nyílt a kommunikációs szokásaim, illetve problémamegoldó stratégiáim fejlesztésére. A Klinikai pszichológia gyakorlat keretein belül pedig kipróbálhattam magam első interjús helyzetben egy pszichiátriai osztályon, ami az egyik legemlékezetesebb momentuma volt az egyetemi éveimnek. 2017-ben segédkeztem a Pszinapszis promóciós videójának elkészítésében, illetve tanulmányaim kezdete óta minden évben részt is vettem azon. A rendezvény minden évben rendkívül inspirált engem, illetve sokat formált a szakmai hozzáállásomon is. Lehetőségem nyílt többet között kipróbálni egy kutyaterápiás foglalkozást a Pszinapszis keretein belül, ami megerősített abban, hogy a jövőben ezzel szeretnék foglalkozni. Azonban tanulmányaim során igyekeztem minden lehetőséget megragadni, hogy ne csak szakmailag, de emberileg is fejlődhessek. Éppen ezért már elsőévesként jelentkeztem az Erasmus Student Network (ESN) csapatba önkéntesnek. A csoportban eltöltött három évem alatt olyan feladataim voltak, mint: események, csapatépítő találkozók megszervezése, promóciója, lebonyolítása, nemzetközi tanulók orientációja, tanulmányi és magánéleti asszisztálása. Ezen feladataimnak köszönhetően rengeteget fejlődtem olyan aspektusokban mint az asszertivitás, csapatmunka, időmenedzsment és problémamegoldás. Ráadásul a külföldi hallgatókkal való szoros kapcsolatom által lehetőségem nyílt az angol nyelv gyakorlására, illetve nemzetközi baráti kapcsolatok kialakítására. Mindezek mellett a Pedagógiai és Pszichológiai Karon eltöltött idő rengeteg önismeretet adott számomra, illetve egy teljesen új perspektívával szélesítette a látókörömet. 
\\
\par Összességében úgy érzem, hogy a Pedagógiai és Pszichológiai Karon töltött éveim alatt maximálisan kihasználtam az egyetem által nyújtott lehetőségeket és olyan élményeket, illetve tudást szereztem, melyek meghatározó részeivé váltak az identitásomnak. Bár a tanulmányaimat külföldön tervezem folytatni, úgy gondolom, hogy az ELTE által nyújtott stabil alapok a továbbiakban is elkísérnek majd engem s segítenek az esetleges akadályok legyőzésében.