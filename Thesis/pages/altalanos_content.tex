\section*{Bevezetés}

Az Egészségügyi Világszervezet szerint világ\-szerte 160 gyermekből 1 megtalálható az autizmus spektrumon, s bár sokukból független, önellátó felnőtt válik, mások egy életen át tartó támogatásra szorulnak (\cite{WHO}). Az autizmus spektrum zavar vagy ASD olyan fejlődési zavar, melyben a személy szociális, kommunikációs és beszédkészsége sérült, érdeklődési köre szűk, viselkedése ritualizált, erősen ragaszkodik az állandósághoz, rutinjaihoz. Kétségtelen, hogy az autizmus spektrum zavar szignifikánsan korlátozhatja az autista személyt mindennapi feladatainak elvégzésé\-ben, azonban megfelelő pszichoszociális intervencióval fejleszthetőek az ASD-s gyermekek kommunikációs, illetve szociális viselkedéses képességei (\textcite{boso_emanuele_minazzi_abbamonte_politi_2007}; \textcite{finnigan_starr_2010}; \textcite{kern_wolery_aldridge_2006}; \textcite{kim_wigram_gold_2009}). Ezen intervenciók egyik, egyre elterjedtebb formája a \textit{zeneterápia.} Ennek a tanulmánynak a célja, hogy feltérképezze a zeneterápia autista gyerekekre gyakorolt hatásaival foglalkozó, szerteágazó vizsgá\-latokat és átfogó képet adjon a témával kapcsolatos jelenlegi, tudományos álláspontról.
\pagebreak
\section*{Diszkusszió}

Bár a zenét, illetve hangszereket egyre elterjedtebben alkalmazzák ASD-s személyek terápiájában, a zeneterápiának számos különböző fajtája van. Kern, Wolery és Aldrigde (2007) például egyénre szabott, zeneterapeuták által költött, zenés mondóká\-kat tanítottak a vizsgálatukban résztvevő gyermekek tanítóinak. A mondókák mind a reggeli rutinhoz kapcsolódtak, s céljuk az otthon-óvoda közötti átmenet könnyebbé tétele, valamint a független funkcionálás és a csoporttársakkal való interakció elősegí\-tése volt. A mondó\-káknak mindegyik esetben \textit{5 lépése} volt:

\begin{enumerate}
	\item A gyermek önállóan belép a terembe
	
	\item A gyermek üdvözöl valakit a teremben (tanítót vagy társat) verbálisan vagy non-verbálisan
	
	\item A gyermek üdvözöl valaki mást a teremben (tanítót vagy társat) verbálisan vagy non-verbálisan
	
	\item A gyermek elköszön a gondozójától (szóban vagy integet), aki ezután elhagyja a termet
	
	\item A gyermek játszani kezd egy általa választott játékkal
\end{enumerate}

A mondókák mindkét esetben hatékonynak bizonyultak, azaz a személyre szabott ének hatására mindkét kisfiú könnyebben tudott alkalmazkodni az új környezethez és több lépést végeztek el tanítói közbeavatkozás nélkül, mint korábban (\cite{kern_wolery_aldridge_2006}). Érdekes eredmény az is, hogy nem csak a kisfiúk alkalmazkodását segítette elő a zeneterápia, hanem a csoporttársaik is szívesebben üdvözölték őket, miután közösen elénekelték a rutint (\cite{kern_wolery_aldridge_2006}) Ezen eredmények nem csak bizonyítják, hogy a zeneterápia egy rendkívül hasznos intervenciós stratégia, de gyakorlati megoldást is kínálnak a spektrumon lévő gyermekek szociális és kommunikációs képességeinek fejlesztésére. Ennél azonban létezik egy sokkal interaktívabb megközelítés is. Kim, Wigram és Gold (2009) vizsgálatukban egyéni improvizációs zeneterápiát tartott gyermekeknek, ahol többek között énekelhettek, dobolhattak, zongorázhattak. 
\\ \\ \\ 
A vizsgálatban \textit{4 feltételt} különböztettek meg:
\begin{enumerate}
	\item A terapeuta által irányított közös zenélés
	\item A gyermek által irányított közös zenélés
	\item A terapeuta által irányított játék (nem zenés)
	\item A gyermek által irányított játék (nem zenés)
\end{enumerate}    
A szerzők a gyermekek \textit{érzelmi, motivációs és interperszonális válaszkészségét } mérték fel, a különböző helyzetekben:

\begin{enumerate}
	\item Mérték a gyermekek örömét, a különböző helyzetekben, vagyis a mosolygásuk, nevetésük gyakoriságát és időtartamát.
	\item Mérték a gyermek-terapeuta közti érzelmi szinkronizációt (amikor a közös tevékeny\-ség során mindketten szomorúságot vagy örömöt élnek át).
	\item Ezek mellett mérték, hogy a gyermekek kezdeményeznek, engedelmeskednek, vagy figyelmen kívül hagyják a terapeuta interakciós kezdeményezését a különböző felté\-telekben.
\end{enumerate} 

Az eredmények szerint a gyermekek gyakrabban éltek át örömöt, s ez az öröm tovább is tartott a zeneterápiában, mint a sima játékban (\cite{kim_wigram_gold_2009}). Továbbá a gyermekek több örömöt éltek át az általuk irányított zeneterápiás foglalkozások keretében, mint a terapeuta által irányítottakban (\cite{kim_wigram_gold_2009}). Az érzelmi szinkronizáció, illetve a kezdeményezés is szignifikánsan többször fordult elő a zeneterápiában - azon belül is a gyermek által irányított feltételben - mint a játékban (\cite{kim_wigram_gold_2009}). Különösen izgalmas eredmény, hogy a gyermekek kétszer olyan valószínűséggel hagyták figyelmen kívül a terapeuta interakciós kezdeményezéseit a sima játék során, mint a zeneterápiában (\cite{kim_wigram_gold_2009}). Ezeknek az eredmények fontos klinikai implikációi vannak. Először is, megállapíthatjuk, hogy a gyermekek \textit{akkor élték át a legtöbb örömöt, amikor ők irányíthatták a foglalkozást}, mindkét feltételben. Másodszor, mivel a közös zenélés kétségtelenül érzelmi bevonódást igényel, a zeneterápia ily módon segíthet fejleszteni az autizmus spektrum zavaros gyermekek adekvát érzelmi reakcióit.

Ezeket az eredményeket támasztja alá egy, az előbbihez hasonló kutatás is. Boso és munkatársai (2007) 52 héten át tartó vizsgálatukban heti rendszerességgel tartottak csoportos zeneterápiát 8 spektrumon lévő fiatal felnőttnek. A résztvevőket 3 alkalommal értékelte saját pszichiáterük a vizsgálat során a CGI-I és BPRS skálán. A majdnem egy éven tartó vizsgálat végére a résztvevők CGI-I és BPRS pontszámai nagyban fejlődtek (tehát kevesebb és kevésbé extrém tüneteket mutattak), illetve különösen izgalmas eredmény, hogy a résztvevők zenés képességei szignifikánsan javultak a kezdeti kompetenciájukhoz képest (\cite{boso_emanuele_minazzi_abbamonte_politi_2007}). Finnigan és Starr (2010) vizsgálatukban a zeneterápia egy újabb formáját alkalmazták. Tanulmányukban \textit{2 feltételt} különböztettek meg:

\begin{enumerate}
	\item Játék zenés aláfestéssel a terapeutától
	\item Játék zene nélkül
\end{enumerate}  

Finnigan és Starr arra volt kíváncsi, hogy vajon a zeneterápia növeli-e az ASD-s gyermek szociális válaszkészségét (a szemkontaktus gyakoriságát, a terapeuta imitálá\-sát és a sorban  következés kivárását, megértését). A gyermek tehát az előző vizsgála\-toktól eltérően, itt mindkét esetben játékokkal játszott (pl.: labda, autó, dob), s a szerzők azt figyelték meg, hogy e játék során hogyan viselkedik a különböző feltételektől függően. Eredményeik szinte teljes összhangot mutatnak az ezen tanulmányban korábban bemutatott vizsgálatokéval. A zeneterápiás feltételben a gyermek szignifikánsan többször létesí\-tett szemkontaktust, imitálta a terapeutát, és követte a sorban következés szabályait, mint a terápia kezdete előtt, illetve mint a zene nélküli feltételben (\cite{finnigan_starr_2010}). A vizsgálatból levonható legfontosabb következtetés, hogy a zeneterápia növelheti a szociális válaszkészséget az autizmus spektrum zavaros személyekben, illetve kiváló motivációs eszköznek is bizonyult (\cite{finnigan_starr_2010}).
\pagebreak

\section*{Összegzés}

Összességében elmondható, hogy a tanulmány keretein belül feldolgozott vizsgálatok szignifikánsan egybevágnak. Mindegyik cikkben, ahol a zeneterápiát hasonlították össze a szerzők valamilyen zene nélküli játékos terápiával, a zeneterápia szignifikánsan jobb eredményeket hozott (\cite{finnigan_starr_2010}; \cite{kim_wigram_gold_2009}). Ez azt jelenti, hogy a zeneterápia nem csak az olyan szociális képességeket  tudja fejleszteni, mint a szemkontaktus fenntartása (\cite{finnigan_starr_2010}), vagy a kezdeményezés (\cite{kim_wigram_gold_2009}), de a független funkcionálást is elősegíti (\cite{kern_wolery_aldridge_2006}). Ezek mellett Kim, Wigram és Gold (2009) cikkéből tisztán látszik, hogy az autizmus spektrum zavaros gyermekek egyszerűen \textit{jobban élvezték} a zeneterápiát, mint a sima játékos foglalkozásokat, ami egy kiemelkedően fontos szempont a megfelelő intervenció kiválasztásánál. Boso és munkatársai (2007) vizsgálata pedig arra is felhívja az olvasó figyelmét, hogy a zeneterápia nemcsak az ASD-s gyermekek szociális képességeit fejleszti, hanem a \textit{zenei kompetenciájukat} is. Mindezek ellenére fontos megjegyezni, hogy az ezen tanulmány keretein belül felsorolt vizsgálatok mindegyike aggasztóan kevés vizsgálati személlyel dolgozott, s emiatt nem lehetünk biztosak benne, hogy eredményeik általánosíthatóak nagyobb populációkra. A jövőben a zeneterápiát s annak hatásait vizsgáló tanulmányoknak egyik központi célja lehet ezen módszertani hiba kiküszöbölése.