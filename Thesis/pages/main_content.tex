\section{Előszó}
A leszbikus, meleg, biszexuális, transznemű és queer (LMBTQ) csoporttal szembeni általános attitűd vizsgálata fontos szerepet tölt be a modern szociálpszichológiában, hiszen a témában folytatott kutatásoknak általában jelentős társadalmi implikációi vannak. Több kutatás is kimutatta például, hogy azok a személyek, akik pozitív attitűddel rendelkeznek a csoport felé, hajlamosabbak voltak támogatni a melegek egyenjogúságára törekvő intézkedéseket \parencite{dasgupta_rivera_2008}.  Poteat és Vecho (2016) pedig kimutatta, hogy azok a gyerekek, akiknek több homoszexuális barátja van, aktívabb védelmező viselkedést mutatnak az áldozat felé homofób bullying helyzetekben. Ezek az adatok arra utalnak, hogy a csoportok közti barátság nem csak attitűdbeli változáshoz vezethet hanem a viselkedésben is kifejtheti hatását. Ez azt jelenti, hogy azok a személyek, akiknek több meleg barátja van, feltehetően nem csak pozitívabban vélekednek az LMBTQ csoport tagjairól, de aktívabban is avatkoznak közbe a csoportot érő kirekesztés ellen.  
A jelen vizsgálatnak alapvetően két célja van. Egyrészt, hogy feltérképezze, befolyásolja-e a heteroszexuális egyének homoszexualitással kapcsolatos attitűdjét az, hogy hány meleg barátjuk van. Másrészt, hogy felmérje ezen baráti kapcsolatok hatását a heteroszexuális személyek viselkedésére, az LMBTQ csoport tagjait diszkrimináló kísérleti helyzetben. A kutatás mögötti motivációt egyrészt a magyar mintán végzett hasonló témájú kutatások limitált száma, másrészt a téma társadalmi és aktuálpolitikai relevanciája adta. 
\pagebreak
\section{Bevezetés}
Bár az Egészségügyi Világszervezet revíziója során törölte a homoszexualitáshoz kapcsolódó tételeket (F66) a Betegségek Nemzetközi Osztályozásából (BNO), a heteroszexuálistól eltérő szexuális orientációk patologizálását igazoló empirikus bizonyítékok hi\-ányában, a magyar populáció számottevő része devianciaként gondol arra \parencite{judit_2011}, és többségük elutasítja az azonos nemű párok örökbefogadáshoz való jogát \parencite{judit_2011}. Ezeknek az adatoknak a tükrében különösen fontos és érdekes  feladat azoknak a tényezők\-nek az azonosítása, melyek egy elfogadóbb attitűd kialakításához vezetnek az LMBTQ+ csoporttal szemben. Allport kontaktushipotézise óta tudjuk, hogy a megfelelő csoportközi kontaktus csökkenti az előítéletet \parencite{christie_allport_1954}. A minőségi csoportközi kontaktus (barátság) pedig számos kutatás szerint növeli a csoporttal szembeni empátiát  \parencite{abbott_cameron_2014,pettigrew_1997}, sőt \textcite{antonio_guerra_moleiro_2017} kutatásában még a kiterjesztett kontaktus is (olyan barátok, akiknek vannak homoszexuális barátaik) növelte azt. 
Anderssen (2002) norvég mintán folytatott longitudinális vizsgálatában azt tanulmányozta, hogy a meleg, illetve leszbikus személyekkel való kontaktus miként hat a heteroszexuális személyek LMTBQ csoporttal szembeni attitűdjére egy két éves periódusban. Eredményei szerint, minél gyakoribb, illetve minőségibb volt a kontaktus az outgrouppal, annál pozitívabb volt feléjük az attitűd is \parencite{anderssen_2002}. Kölcsönösen pedig, a pozitív attitűdbeli változás a csoport felé, több csoportközi kontaktushoz vezetett \parencite{anderssen_2002}. A saját csoporton kívüli, azaz outgroup tagokkal való kontaktusnak számos más előnye is van. \parencite{capozza_falvo_trifiletti_pagani_2014} vizsgálatukban azt találták, hogy a kiterjesztett kontaktus összefügg a csökkent outgroup infrahumanizációval. Ez azért fontos eredmény, mert az infrahumanizáció során az egyén az ingroup-ot felruházza olyan humán-specifikus jellemzőkkel, mint a másodlagos érzelmek (pl.: bűntudat) és a tudatosság, míg az outgroup tagjait kevésbé.
\\ \par
Azonban a csoportközi barátságok nem csak attitűdbeli változást vonhatnak maguk után. Egy másik, a homoszexualitással kapcsolatos implicit attitűdöket és explicit viselkedéses szándékokat vizsgáló kísérletben a kutatók a hosszútávú közvetlen kontaktus hatásait hasonlították össze az outgroup személynek való rövidtávú kitettséggel. Az eredmények azt mutatták, hogy azok a személyek, akiknek limitált kontaktusa volt melegekkel és leszbikusokkal nem csak szignifikánsan magasabb melegellenes implicit attitűddel rendelkeztek \parencite{dasgupta_rivera_2008}, de hajlamosabbak voltak diszkrimináló szavazói szándékot mutatni a viselkedéses helyzetben \parencite{dasgupta_rivera_2008} . Ezzel szemben azok a személyek, akiknek már volt előzetes hosszútávú kapcsolata a csoporttal, alacsonyabb melegellenes implicit attitűddel rendelkeztek, és hajlamosabbak voltak megszavazni az egyenlő jogokat (pl.: házasság) a csoportnak \parencite{dasgupta_rivera_2008} . Ezek az eredmények azért különösen lényegesek, mert egyrészt bizonyítják a csoportközi barátságok pozitív attitűddel való együttjárását, másrészt arra utalnak, hogy a minőségi csoportközi kontaktus az explicit viselkedéses helyzetekben is hatást gyakorol. Ezeket az eredményeket támasztja alá \textcite{antonio_guerra_moleiro_2017} korábban már említett vizsgálata, hiszen a kiterjesztett kontaktus nem csak a személyek megnövekedett empátiáját jósolta be a csoport iránt, de aktívabb intervenciókhoz is vezetett a vinyetták által leírt homofób bullying helyzetekben. \textcite{pettigrew_1997} szerint a csoportközi barátság azért is nagyon lényeges, mert az általa létrejött kapcsolat megfelel a kontaktushipotézis \parencite{christie_allport_1954} feltételeinek, ami pedig az előítéletesség csökkentésének alapja \parencite{christie_allport_1954}.  Ezek a feltételek a következők:
1. Egyenlő státusz a csoportok közt, 2. Közös célok, 3. Együttműködés, 4. Támogató társadalmi normák és intézmények.
Az LMBTQ csoport tagjaival való kontaktus még arra is hatással van, hogy az iskolai alkalmazottak (pl.:tanárok) mennyire legitimálják, vagy ellenzik a homofób bullying-ot az iskolai környezetben \parencite{zotti_carnaghi_piccoli_bianchi_2018}. \textcite{zotti_carnaghi_piccoli_bianchi_2018} friss tanulmányukban azt találták, hogy a leszbikus, illetve meleg személyekkel való kontaktus hiánya meggátolta az intervenciót a csoport elleni diszkriminációs helyzetekében \parencite{zotti_carnaghi_piccoli_bianchi_2018} , illetve elősegítette a  homofób bullying személyes legitimációját az iskola dolgozói körében \parencite{zotti_carnaghi_piccoli_bianchi_2018}. \\
\\
Az LMBTQ személyekkel kötött barátságok azért is nagyon fontosak, mert \textcite{poteat_vecho_2016} adatai szerint a 722 fő középiskolás mintájuk\-nak 66.8\%-a tapasztalt homofób viselkedést legalább egyszer az elmúlt egy hónapban. Ezek az adatok különösen riasztóak, ha figyelembe vesszük, hogy azoknak a gyermekeknek, akik voltak már bullying áldozatai, többször vannak öngyilkos gondolataik \parencite{hinduja_patchin_2010}, illetve nagyobb valószínűséggel követnek el öngyilkosságot \parencite{hinduja_patchin_2010} . \textcite{poteat_vecho_2016} kutatásában mindazonáltal kiválasztották, azokat a gyermekeket, akik tapasztaltak homofób viselkedést az elmúlt egy hónapban, majd megkérdezték őket, hogy az elmúlt 30 napban milyen gyakran csinálták a következőket: 1. Szóltak egy felnőttnek az incidensről, 2. Megpróbálták megállítani az elkövetőt, 3. Támogatták az áldozatot, 4. Rávettek másokat, hogy támogassák az áldozatot, 5. Felszólaltak az áldozat érdekeiért, 6. Segítették, bátorították az áldozatot, hogy szóljon az incidensről egy felnőttnek, 7. Kifejezték a nemtetszésüket a történtekkel kapcsolatban, 8.  Nem járultak hozzá a szituációhoz (pl.: cyberbullying kapcsán nem osztották meg képet, posztot).
Az eredmények szerint, azok a gyermekek, akiknek volt LMBTQ barátjuk, elmondásuk szerint gyakrabban vettek részt a felsorolt intervenciókban \parencite{poteat_vecho_2016} illetve többször választottak aktívabb intervenciókat (pl.: konfrontáció), mint azok a társaik, akiknek nem volt ilyen kapcsolata \parencite{poteat_vecho_2016}.
Ezek az adatok pedig szintén támogatják azt az elképzelést, hogy a csoportközi barátságok a viselkedésre is hatást gyakorolnak. 
\\
\par
Mindezek alapján megállapíthatjuk, hogy az LMBTQ csoporttal való kontaktus növeli az empátiát \parencite{abbott_cameron_2014}, növeli az egyenjogúságra való törekvő intézkedé\-sek támogatását \parencite{dasgupta_rivera_2008}, pozitívabb implicit attitűdöt eredményez \parencite{dasgupta_rivera_2008} , aktívabb intervenciós szándékhoz vezet \parencite{poteat_vecho_2016}, és csökkenti a homofób bullying legitimációját \parencite{zotti_carnaghi_piccoli_bianchi_2018}, valamint az outgroup  tagok infrahumanizációját \parencite{capozza_falvo_trifiletti_pagani_2014}. Azonban kulcsfontosságú meghatározni ezen kontaktus jellegét. \textcite{baunach_burgess_muse_2009} amerikai egyetemistákon folytatott kísérlete kimutatta, hogy a különböző típusú érintkezések a csoporttal, más-más hatással voltak a homoszexuálisok felé mutatott előítéletességre. Először is azt találták, hogy a kapcsolatok minősége nagyobb szerepet játszott az előítélet csökkentésében, mint azok mennyisége \parencite{baunach_burgess_muse_2009}. Másodszor pedig megállapították, hogy bizonyos fajta kapcsolatok fontosabbak voltak, mint mások \parencite{baunach_burgess_muse_2009}. Nevezetesen, míg tulajdonképpen az összes fajta kontaktus csökkentette az előítéletességet a heteroszexuálisokban \parencite{baunach_burgess_muse_2009}, az előítéletesség\-re gyakorolt legnagyobb hatást a baráti kapcsolatok száma okozta \parencite{baunach_burgess_muse_2009}. A minőségi baráti kapcsolatok még a meleg családtagokkal való kapcsolatnál is szignifikánsabb hatást fejtett ki az egyének előítéletességére \parencite{baunach_burgess_muse_2009}. Ennek oka lehet például az, hogy a sztereotípiák fenntartása érdekében a személy a meleg családtagokra alkategóriát képez a csoporton belül, s így nem csökkenti az outgroup iránti előítéletességét.
\\
\par
Összességében elmondhatjuk tehát, hogy az LMBTQ csoporttal létesített kontaktus, de főleg a barátság \parencite{baunach_burgess_muse_2009} számos kutatás szerint nem csak pozitív attitűdbeli változásokkal jár \parencite{dasgupta_rivera_2008}, de emellett a viselkedéses helyzetekben is aktívabb illetve asszertívabb intervencióhoz vezet \parencite{zotti_carnaghi_piccoli_bianchi_2018,}. 

\section{Hipotézisek}
Ennek a tanulmánynak alapvetően két fő hipotézise van:
\begin{enumerate}
	\item Azok a személyek, akiknek több meleg barátja van, pozitívabb implicit attitűddel fognak rendelkezni a csoport felé, mint akiknek nincs ilyen kapcsolata.
	\item Azok a személyek, akiknek több meleg barátja van, nagyobb valószínűséggel fogják konfrontálni a homofób megszólalást az explicit viselkedéses helyzetben.
\end{enumerate}


\section{Módszer}
\subsection{Résztvevők}
106 fő magyar nemzetiségű személy (75 nő és 28 férfi) vett részt a kutatásban. A résztvevők életkora 18 évtől, 51 évig terjedt (M=24.67, SD=7.34). A résztvevők többsége felsőfokú tanulmányokat folytató diák volt (57\%), 28\% felsőfokú végzettséggel rendelkezett, 9\% - nak középiskolai végzettsége és 3\% -nak általános iskolai végzettsége volt. A kísérleti személyek 83\% - a vallotta heteroszexuálisnak magát, s mivel a tanulmányban vizsgált célcsoport a homoszexuálisok voltak, azokat a személyeket, akik homoszexuálisnak, biszexuálisnak, vagy egyébnek vallották magukat, rájöttek a megtévesztésre, illetve nem kívánták meghatározni a nemüket vagy szexuális orientációjukat, elimináltuk az elemzésből. Így a végső minta 96 személyből állt (72 nő és 26 férfi). 
\subsection{Eljárás}
\subsubsection{Adatfelvétel}
A kutatás az Eötvös Loránd Tudományegyetem Pedagógiai és Pszichológia Karának és Szociálpszichológia Tanszékének keretein belül jött létre. Az adatok online kerültek begyűjtésre. A résztvevőket (\textgreater 18) a közösségi médiumon keresztül toboroztuk, majd tájékoztattuk őket, hogy a részvétel önkéntes, anonim, illetve bármikor  indoklás nélkül megszakítható. A kérdőív kitöltése kizárólag asztali gépről, vagy laptopról volt lehetséges, mivel a Bizalom Játékot \parencite{szekeres_halperin_kende_saguy} más eszköz nem támogatta. 
\subsubsection{Kérdőív}
A résztvevők először a Bizalom Játékkal játszottak (kísérleti manipuláció), majd ezután egy homoszexualitással kapcsolatos kérdőívcsomagot töltöttek ki, ami egy Homofóbia skálából, illetve Rokonszenv skálából állt. Ezután megkérdeztük a résztvevőket, hogy hány meleg barátjuk van, illetve, hogy véleményük szerint barátaiknak hány százaléka meleg. A játék, illetve a kérdőív kitöltése összesen nagyjából 30-40 percet vett igénybe. A kérdőív kitöltése után a résztvevők utólagos tájékoztatásban részesültek, majd megköszön\-tük a részvételüket. 

\subsubsection{Kísérleti Manipuláció}
A résztvevők azt a tájékoztatást kapták, hogy először egy online játékot fognak játszani, majd egy kérdőív kitöltésére kértük meg őket. A kísérleti személyek úgy tudták, hogy a ¨Bizalom Játék¨ - kal \parencite{szekeres_halperin_kende_saguy} fognak játszani, ahol azt szeretnénk megfigyelni, hogy változik-e az emberek bizalma, ha valaki megfigyeli őket, illetve befolyásolja-e a bizalmat a megfigyelő neme. A résztvevők ezután ezt az instrukciót kapták:
\begin{quoting}[\itshape]
	¨Hogyan játsszák a Bizalom Játékot? \\
	- Két ember játszik egymással. Az első játékos kap egy kezdő összeget (például 200 Ft-ot) \\
	- Az első játékos három opció közül választhat: a második játékosnak odaadja az egészet
	(200Ft), a felét (100Ft) vagy semmit (0Ft, tehát megtartja magának a 200Ft-ot). Ezeket az
	opciókat a játékban az „MINDEN”, „FELE” és „SEMMI” gombok jelölik majd. Tegyük fel, hogy az első játékos úgy dönt, hogy a pénze felét (100Ft) odaadja. \\
	- Ez az összeg automatikusan megháromszorozódik, és ezt a tripla összeget kapja meg a második játékos (300Ft). \\
	- A második játékos két opció közül választhat: vagy megosztja ezt a tripla összeget (tehát mindketten 150Ft-ot kapnak) vagy semmit nem ad vissza (tehát megtartja magának a 300Ft-ot). Tegyük fel, hogy a második játékos úgy dönt, hogy megosztja (150Ft-ot ad vissza az első játékosnak). \\
	- Ezzel vége a játéknak. \\
	- A példa alapján, az első játékos összesen 250 Ft-ot nyert (100 Ft, amit az elején megtartott, 
	plusz 150 Ft, amit a második játékos visszaküldött), míg a második játékos 150 Ft-ot nyert.
	Figyeljük meg, hogy a játékosok akkor nyernek a legtöbb pénzt, ha az első játékos megbízik a második játékosban, és odaadja az összes pénzt, a második játékos pedig megbízható és fair módon megosztja azt, amit kapott. Ebben az optimális bizalmi helyzetben, ha 100 Ft a kezdő összeg, akkor mindkét játékos a végén 300 Ft-ot keresne.¨
\end{quoting}

A résztvevők tehát úgy tudták, hogy valódi személyekkel fognak játszani a Bizalom Játékban, azonban a többi résztvevő valójában nem volt valós. A játékban a kísérleti személy, először mindig megfigyelő szerepet töltött be, azaz láthatta ahogy másik két ¨játékos¨ játssza a Bizalom Játékot. A játék során a résztvevők által megfigyelt személy (Márk) készségesen megosztotta vagyonát partnereivel, mindaddig, amíg a harmadik kör\-ben össze nem került egy Dani nevű játékossal, akinek neve mellett megjelent a szivárvány\-zászló, a meleg közösség egyik legfontosabb szimbóluma. A kísérleti személy ekkor azt láthatta, hogy Márk nem osztja meg vagyonát Danival, majd homofób üzenetet írt a résztvevőnek. Ez az üzenet tartalmazhatott enyhe (¨na erre a langyira én pénzt nem bízok¨), illetve súlyos (¨inkább pofont adnék ennek a buzinak, mint pénzt¨) homofób megszólalást. Ezt a két feltételt azonban az analízis során egybevettük, főként az alacsony elemszám miatt. A résztvevőknek ekkor lehetőségük nyílt válaszolni az üzenetre (konfrontálódni) vagy továbblépni válaszolás nélkül. A következő körben a kísérleti személyek is játszhattak Danival, ahol kiválaszthatták, hogy vagyonuk egészét (MINDEN), felét (FELE), vagy semmit (SEMMI) adnak neki. Ezután egy előre kódolt hiba véget vetett a játéknak. Erre a manipulációra azért volt szükség, mert anélkül a társas kívánatosság nagyban befolyásolta volna a viselkedést, s  ez megakadályozta volna, hogy a valóságot tükröző eredményeket kapjunk.
\pagebreak

\subsection{Mérőeszközök}

\subsubsection{Konfrontáció}
A Bizalom Játék során a résztvevőknek lehetősége volt válaszolni az általuk megfigyelt személy (Márk) homofób üzenetére, vagy továbblépni válaszolás nélkül. Ezeket a lehetőségeket a \textit{Válaszol} illetve \textit{Folytat}  gombok jelölték. A beérkező válaszokat egyesével értékeltük, átkódoltuk majd három kategóriába soroltuk őket: nem konfrontáló, konfrontáló, egyetértő.
\subsubsection{Homofóbia skála}
A résztvevők egy általunk összerakott 10 elemből álló attitűdskálát töltöttek ki, ami tulajdonképpen egy módosított \textit{Attitudes Toward Homosexuality Scale} \parencite{anderson_koc_falomir-pichastor_2017} és módosított \textit{Attitudes Toward Gays and Policy Support Index} \parencite{jang_lee_2014} volt. A skála egy ötfokú Liker-skála volt, ahol az 1 = egyáltalán nem értek egyet, 2 = inkább nem értek egyet 3 = nem tudom eldönteni 4 = inkább egyetértek 5 = teljesen egyetértek választ jelölte. A skála tételei a következőek voltak:
\begin{enumerate}
	\item A homoszexualitás természetellenes
	\item Szolidaritást vállalok a melegekkel és leszbikusokkal
	\item Támogatom, hogy a meleg párok is fogadhassanak örökbe gyerekeket
	\item Egyáltalán nem zavarna, ha kiderülne, hogy a gyermekem meleg, vagy leszbikus
	\item Támogatom a meleg házasságot
	\item Zavarna, ha kiderülne, hogy a gyermekem egyik tanára meleg vagy leszbikus
	\item Empátiát érzek a homoszexuális emberek iránt
	\item Zavarba jövök, ha egy meleg párt látok az utcán egymás kezét fogva sétálni
	\item Egy meleg pár (gyermekkel vagy anélkül) családnak számít
	\item Nem szavaznék olyan politikusra, aki nyíltan meleg-ellenes
	
\end{enumerate}
A skála megbízhatósága jó ($\alpha=0.88 \ ).

\subsubsection{Rokonszenv skála}
Ezután a rokonszenv skálán (Feelings Thermometer) a résztvevőknek egy 100 fokos skálán kellett megjelölniük, hogy az egyes csoportok átlagos tagjait mennyire találják ellenszenvesnek, vagy rokonszenvesnek, ahol a 0 fok = rendkívül ellenszenves, 100 fok = rendkívül rokonszenves választ jelölte. A rokonszenv skálán megítélt csoportok: muszlimok, zsidók, romák, bevándorlók, melegek.

\subsubsection{Barátok száma}
A résztvevők először egy 4 fokos skálán ítélték meg, hogy hány meleg barátjuk van: 1 = egyáltalán nincs, 2 = inkább nincs, 3 = inkább van, 4 = sok van. Ezután azt kérdeztük tőlük, hogy megítélésük szerint barátaiknak hány százaléka meleg (0\% - 100\%).

\pagebreak
\section{Eredmények}
A Kísérleti manipuláció során tett homofób megjegyzésre beérkező válaszok a kutatás szerzői  által egyesével értékelésre kerültek. A homofób megjegyzést egyértelműen elítélő válaszok (pl.:"szégyeld magad!"), illetve inkább elítélő válaszok (pl.:"ez nem volt szép tőled") a \textit{Konfrontáló} csoportba kerültek. A \textit{Nem konfrontáló} csoportba azok a személyek kerültek, akik vagy egyáltalán nem válaszoltak az üzenetre, vagy irreleváns, nem egyetértő, de nem is konfrontáló választ adtak. A homofób megjegyzést buzdító válaszok az \textit{Egyetértő} csoportba kerültek. Azoknál a megjegyzéseknél, amiknek jellegét nem tudtuk egyértelműen eldönteni (pl.: "LOL") a vizsgált személy önbeszámolója döntött, vagyis, hogy ő úgy érezte-e, hogy konfrontálódott vagy sem.
Az adatok elemzése az IBM SPSS szoftver 25.0.0.-ás verziójával történt. A hipotézis \textsubscript{1} és a a hipotézis \textsubscript{2} is Pearson -féle korrelációval került ellenőrzésre.
A kutatási kérdések megválaszolása előtt leíró statisztikát hajtottam végre a mérőeszközökre (Homofóbia skála, Rokonszenvskála, Barátok száma). A játékban résztvevő 96 főből sajnos csak 50 fő töltötte ki a kérdőívet is, a többiek a játék után megszakították a részvételt. A Homofóbia skála megbízható ($\alpha=0.88,\  N=50, \  M=2.42, \ SD=0.96$). A rokonszenv skálának ($N=50,\  M=63,\  SD=29$) és barátok számának ($N=50, \ M=2.38, \ SD=0.87$) leíró statisztikája, az 1. számú "Leíró statisztika" című táblázatban látható.
\begin{table}[h]
	\centering
	\begin{tabular}{|l|l|l|l|}
		\hline
		& \textit{N}  & \textit{M}    & \textit{SD}   \\ \hline
		Homofóbia skála  & 50 & 2.42 & 0.96 \\ \hline
		Rokonszenv skála & 50 & 63   & 29   \\ \hline
		Barátok száma    & 50 & 2.38 & 0.87 \\ \hline
	\end{tabular}
	\caption{Leíró statisztika}
	\label{table:1}
\end{table}

Az adatainkon végzett Pearson-korrelációs tesztek alapján a Homofóbia skála és a meleg barátok száma között negatív korreláció figyelhető meg ($r= -0.492, \  p<0.05$), illetve ugyanez a negatív korreláció mutatkozik a Homofóbia skála és a meleg barátok százalékos aránya közt ($r= -0.335, \  p=0.017$). Mindezekhez hasonlóan a Rokonszenv skála és a meleg barátok száma közt pozitív korreláció figyelhető meg ($r= 0.413, \  p=0.003$), illetve a Rokonszenv skála és a meleg barátok százalékos aránya közt is pozitív a korreláció ($r=0.318, \  p=0.024$). 


\begin{table}[h]
	\small
	\begin{tabular}{@{}lp{2cm}p{2cm}lp{2cm}l@{}}
		\toprule
		& Konfrontáció & Meleg barátok száma & Bizalom & Homofóbia & Rokonszenv \\ \midrule
		Konfrontáció            & -            & 0.279               & 0.319   & -0.236    & 0.240      \\
		Meleg barátok száma     & 0.279        & -                   & 0.114   & -.492**   & .413**     \\
		Bizalom                 & 0.319        & 0.114               & -       & -0.118    & 0.040      \\
		Homofóbia               & -0.236       & -.492**             & -0.118  & -         & -.531**    \\
		Rokonszenv              & 0.240        & .413**              & 0.040   & -.531**   & -          \\
		Iskolai végzettség      & 0.159        & .356*               & 0.027   & -0.099    & 0.184      \\
		SES & 0.326        & .353*               & 0.062   & -0.218    & 0.230      \\
		Nem                     & -0.060       & .375**              & -0.114  & -0.275    & .408**     \\
		Meleg barátok (\%-ban)  & -0.205       & .430**              & -0.117  & -.335*    & .318*      \\
		Életkor                 & -0.154       & 0.119               & 0.153   & .300*     & -0.147    
	\end{tabular}
	\caption{Korreláció  1}
	\label{table:2}
\end{table}


% Please add the following required packages to your document preamble:
% \usepackage{booktabs}
\begin{table}[h]
	\begin{tabular}{@{}lp{2cm}p{2cm}lp{2cm}l@{}}
		\toprule
		& Iskolai végzettség & SES & Nem    & Meleg barátok (\%-ban) & Életkor \\ \midrule
		Konfrontáció            & 0.159              & 0.326                   & -0.060 & -0.205                 & -0.154  \\
		Meleg barátok száma     & .356*              & .353*                   & .375** & .430**                 & 0.119   \\
		Bizalom                 & 0.027              & 0.062                   & -0.114 & -0.117                 & 0.153   \\
		Homofóbia               & -0.099             & -0.218                  & -0.275 & -.335*                 & .300*   \\
		Rokonszenv              & 0.184              & 0.230                   & .408** & .318*                  & -0.147  \\
		Iskolai végzettség      & -                  & .300**                  & .248*  & -0.092                 & .338**  \\
		SES & .300**             & -                       & 0.021  & 0.164                  & 0.074   \\
		Nem                     & .248*              & 0.021                   & -      & .364**                 & -0.120  \\
		Meleg barátok (\%-ban)  & -0.092             & 0.164                   & .364** & -                      & -0.159  \\
		Életkor                 & .338**             & 0.074                   & -0.120 & -0.159                 & -       \\ \bottomrule
	\end{tabular}
	\caption{Korreláció 2}
	\label{table:3}
\end{table}

Ezek az adatok alátámasztják az első számú hipotézist, vagyis, hogy minél több meleg barátja van a személynek, annál pozitívabb lesz az attitűdje az LMBTQ csoport felé.  Az adatainkon végzett Pearson-korreláció alapján, nincs szignifikáns különbség a Bizalom Játékban mutatott bizalomban (vagyis abban, hogy mennyi pénzt adtak a résztvevők a meleg játékosnak) a meleg barátok számától függően ($r=0.114, \ p=0.432$). A korrelációs elemzés alapján nincs szignifikáns különbség a  homofób megjegyzés konfrontálódásában sem a meleg barátok számának ($r=0.279, \ p=0.135$), sem a meleg barátok százalékos arányának ($r=-0.205 \ p=0.276$) függvényében. Ezek az adatok tehát nem támasztják alá a második számú hipotézist, vagyis, hogy azok személyek, akiknek több meleg barátja van, valószínűbb, hogy konfrontálni fogják a homofób megjegyzést a kísérleti helyzetben, mint akiknek nincs, vagy csak kevés ilyen baráti kapcsolatuk van. A demográfiai adatok alapján, az életkor pozitívan korrelált a Homofóbia skálával ($r=0.300, \  p=0.039$) illetve az iskolai végzettség pedig pozitívan korrelált a meleg barátok számával ($r=0.356, \  p=0.011$). Az iskolai végzettség mindezek mellett nem korrelált a konfrontációval ($r=0.159, \  p=0.394$), a bizalommal ($r=0.027, \  p=0.843$), a homofóbiával ($r=-0.099, \  p=0.495$) és a rokonszenvvel ($r=0.184, \  p=0.201$) sem. Az életkor nem korrelált a csoport iránt érzett rokonszenvvel ($r=-0.147, \ p=0.318$), illetve a bizalommal ($r=0.153, \  p=0.274$), a konfrontációval ($r=-0.154, \  p=0.427$), és a meleg barátok számával ($r=0.119, \  p=0.419$) sem. A Rokonszenv és a Homofóbia skálán elért eredmények közt negatív korreláció figyelhető meg ($r=-0.531, \  p=0.000$). A szocioökomómiai státusz és a meleg barátok száma közt pozitív és szignifikáns a korreláció ($r=0.353, \  p=0.012$). A korrelációs eredmények a 2. és 3. számú "Korreláció" című összesített táblázatokban láthatóak.
\page

\section{Diszkusszió}
Ennek az exploratív tanulmánynak a célja az volt, hogy megvizsgálja a melegekkel való baráti kapcsolatok hatásait a heteroszexuális személyek LMBTQ csoporttal szembeni attitűdjére, illetve, hogy felmérje ezen baráti kapcsolatok hatását az explicit viselkedésre. Nevezetesen azt szerettük volna kideríteni, hogy vajon azok a személyek, akiknek több meleg barátja van, jobban fogják-e konfrontálni a homofób viselkedést. A jelen tanulmány több aspektusban is kiterjesztette a témában jelenleg fellelhető vizsgálatokat, hiszen nem csak a baráti kapcsolatok attitűdre gyakorolt hatásait vizsgálta, de azt is feltérképezte, hogy a baráti kapcsolatok során kialakult pozitív attitűd, átfordul-e viselkedéses intervencióba (konfrontációba) a csoporttal szembeni diszkriminatív helyzetekben.

\subsection{Barátok száma és Homofóbia}
A bevezetőben bemutatott számos kutatáshoz hasonlóan \parencite{dasgupta_rivera_2008,pettigrew_1997}  statisztikai analízisünk során azt az eredményt kaptuk, hogy azok az egyének, akiknek több meleg barátja van, illetve barátaik nagyobb százaléka homoszexuális, pozitívabb attitűddel rendelkeznek a csoport iránt.  Tehát azok a résztvevők akiknek több baráti kapcsolata volt melegekkel, alacsonyabb pontszámot értek el a Homofóbia skálán, és magasabbat a Rokonszenv skálán, mint azok a résztvevők, akiknek kevesebb hasonló kapcsolata volt. Ezek az adatok alátá\-masztják az első számú hipotézisünket, s szélesítik azoknak a vizsgálatok körét, melyek az outgroup tagokkal való baráti kapcsolatok korrelációját hangsúlyozzák a pozitív csoporttal szembeni attitűddel. 

\subsection{Barátok száma és Rokonszenv}
A meleg barátok száma és a homoszexuálisok iránt érzett rokonszenv közt elemzésünk során  szignifikáns pozitív korrelációt találtunk. Ez azt jelenti, hogy azok a személyek, akiknek több meleg barátjuk volt, rokonszenvesebbnek találták magát az LMBTQ csoportot. Ezek az adatok újfent támogatják az első számú hipotézisünket, vagyis, hogy az outgroup taggal létesített baráti kapcsolat együtt jár a csoporttal szembeni pozitív attitűddel a heteroszexuális személyek körében. Ennek a kapcsolatnak az irányát azonban számos szerző feszegeti munkájában. \textcite{pettigrew_1997} szerint például, azok a személyek akiknek eleve pozitívabb az attitűdje a kül-csoport felé, nagyobb valószínűséggel létesítenek barátok kapcsolatokat azok tagjaival, aminek hatására még pozitívabb attitűdjük lesz feléjük, s ezáltal a személy még inkább nyitott lesz új baráti kapcsolatok kialakítására \parencite{pettigrew_1997}. Ezek alapján mi sem lehetünk biztosak benne, hogy a csoportok közti baráti kapcsolatok eredményezték a pozitív attitűdöt a csoport felé, vagy éppen fordítva. 

\subsection{Barátok száma és Konfrontáció}
A jelen tanulmány bevezetőjében bemutatott vizsgálatoktól \parencite{poteat_vecho_2016} elté\-rően, elemzésünk során azt találtuk, hogy nem volt szignifikáns eltérés a homofób viselkedés konfrontációjában attól függően, hogy hány meleg barátja volt az illetőnek. A konfrontáció azzal sem függött össze, hogy a résztvevők az enyhe (¨na erre a langyira én pénzt nem bízok¨) vagy erős (¨inkább pofont adnék ennek a buzinak, mint pénzt!!!¨) homofób megjegyzésnek voltak tanúi, ezért a későbbi elemzések soránt ezt a két feltételt összevontunk. Ezek az eredmények nem támogatják a második számú hipotézist, ami a meleg barátok száma és a konfrontáció között összefüggést feltételezett. Ez az eredmény azért különösen érdekes, mert a vizsgált szakirodalmak egyöntetűen azt implikálták, hogy a homoszexuális barátok száma aktív és asszertív intervencióhoz vezet homofób bullying helyzetekben \parencite{poteat_vecho_2016}. Nagyon hasonló eredményeket talált azonban \textcite{kawakami_dunn_karmali_dovidio_2009}, akik szerint az emberek sokszor rosszul mérik fel, hogy hogyan reagálnának le rasszista megszólalásokat \parencite{kawakami_dunn_karmali_dovidio_2009}. Ebben a kutatásban a szerzők megkülönböztettek ¨előrejelző/forecaster¨ illetve ¨átélő/experiencer¨ csoportokat, majd megfigyelték, hogy hogyan reagálnak a résztvevők ugyanarra a rasszista megszólalásra. Eredményeik szerint az előrejelzők úgy gondolták, hogy a rasszista megjegyzés nagyon felzaklatná őket, az átélők közül viszont csak kevesen mutattak érzelmi distresszt annak hallatán \parencite{kawakami_dunn_karmali_dovidio_2009}. Ezek az adatok egybevágnak \textcite{crosby_wilson_2015} eredményeivel, akik kutatásukban összehasonlították a vizsgálati személyek képzelt illetve valós érzelmi és viselkedéses válaszait ugyanarra a becsmérlő homofób kijelentésre. Azok a résztvevők, akik elképzelték, hogy hallják a homofób megjegyzést szignifikánsan erősebb negatív érzelmi státuszt jelentettek, mint azok akik valóban hallották azt \parencite{crosby_wilson_2015}, s majdnem felük úgy gondolta, hogy valós helyzetben konfrontálná az elkövetőt \parencite{crosby_wilson_2015}. Ehhez képest, azok közül a személyek közül, akik valóban hallották a homofób megjegyzést, senki nem konfrontálta azt \parencite{crosby_wilson_2015}. Tehát Crosby és Wilson (2015) adatai egy egyértelmű diszkrepanciát mutatnak a személyek elképzelt, illetve valós reakciói közt \parencite{crosby_wilson_2015}. 
Mindezek alapján érthető az eltérés a jelen tanulmány s a bevezetőben bemutatott vizsgálatok eredményei között, ha figyelembe vesszük, hogy az utóbbiak mind valamilyen önbevalláson alapuló, illetve elképzelt, teoretikus vizsgálati eszközre (pl.: vinyettek) hagyatkoztak \parencite{poteat_vecho_2016, antonio_guerra_moleiro_2017,dasgupta_rivera_2008}.

\subsection{Barátok száma és a Bizalom Játék}
A résztvevőknek lehetősége volt játszani a meleg játékossal a Bizalom Játékban, azonban a meleg barátok száma nem függött össze az irántuk mutatott bizalommal (vagyis azzal, hogy mennyi pénzt adtak nekik a játékban). Ez az eredmény szintén azt mutatja, hogy bár a meleg barátok száma együtt jár a csoport iránti pozitív attitűddel, ez a pozitív attitűd nem fordul át a csoporttal szembeni viselkedésbe. 
\par 
\subsection{Összegzés}
Ezeknek az eredményeknek számos releváns implikációja van. Például felhívják a figyelmet arra, hogy az outgrouppal szembeni pozitív attitűdök nem feltétlenül járnak együtt a csoport védelmezésével, illetve a csoportot érő diszkrimináció konfrontációjával. Ez hétköznapi példákkal élve azt jelenti, hogy bár a személy pozitívan viszonyul az LMBTQ csoport tagjaihoz, nem feltétlenül megy el tüntetni a csoport érdekeit sértő intézkedések ellen, illetve nem feltétlenül lép közbe a csoport tagjait érő (fizikai vagy verbális) bántal\-mazás szemtanújaként. Ezek az eredmények arra is figyelmeztetnek, hogy a napjainkban használa\-tos kontaktintervenciók mellett, szükség lehet egy más fajta megközelítésre is, aminek a fókuszában az attitűdök helyett az aktív viselkedéses intervenciók állnak. Különösen lényeges lehet ezen intervenciók azonosítás és népszerűsítése az áltános- illetve középiskolás gyermekek körében, a homofób bullying megfékezése céljából. \\
\subsection{A kutatás erősségei, limitációi és kitekintés}
Kutatásunk elsősorban azt a célt szolgálta, hogy a melegekkel kialakított baráti kapcsolatok attitűdre és viselkedésre gyakorolt hatásait vizsgálja.
Kutatásunk legnagyobb előnye, hogy nem elképzelt helyzeteket és önbeszámolós kérdőívet használt a homofób diszkrimináció konfrontációjának vizsgálatára, s ezáltal a vizsgált személyek valós viselkedését volt képes felmérni. Úgy gondoljuk, hogy a jövőbeli kutatásoknak mindenképp figyelembe kell venniük a képzelt, illetve valós viselkedések közti diszkrepanciát, ha pontos képet kívánnak adni a témáról. Bár ez a kutatás kifejezetten a meleg személyekkel való baráti kapcsolatokat vizsgálta, a jövőbeli kutatások szempontjából érdemes lehet megvizsgálni az LMBTQ csoport többi tagjával szembeni attitűdöt, illetve viselkedést.
Mindazonáltal fontos rámutatnunk az explorációs kutatásunk korlátaira is. Először is a kutatásban résztvevő személyek 67.9\% - a volt nő, s mindössze 26\% - a volt férfi. Emellett kitöltőink nagy része (60\%) egyetemista volt.  Az online mintavétel bár nagyszámú résztvevő gyűjtésére alkalmas, nem reprezentatív. Továbbá mivel a Bizalom Játékot csak asztali gépen vagy laptopon lehetett játszani (más elektronikai eszköz nem támogatta a játékot) a vizsgálat csak korlátozott számú embert ért el, illetve a hosszú játékidő (30-40 perc) miatt többen abbahagyták a vizsgálatot útközben. 

\pagebreak



