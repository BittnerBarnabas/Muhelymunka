\\
\par A műhelymunka egy fontos társadalmi jelenséget dolgoz fel, amiben a műhelymunkázó a
melegekkel való baráti kapcsolatok összefüggését vizsgálja a heteroszexuális személyek
LMBTQ csoporttal szembeni attitűdjeivel, illetve, ezen baráti kapcsolatok összefüggését konkrét
viselkedéssel. Ebben a kontextusban, érdekes és fontos összefüggést talált arra vonatkozóan,
hogy a szakirodalmat alátámasztva, a baráti kapcsolatok pozitív kapcsolatban állnak a
melegekkel szembeni attitűdökre, azonban szemben azzal amire a szakirodalom következtet,
nincs kapcsolatban a tényleges meleg-támogató viselkedéssel. Ez utóbbi eredmény
megkérdőjelezi a szakirodalmat, ahol eddig csak vignette szituációkban és csak (cselekvéses)
intenciókat mértek. Ebből kifolyólag a jelen kutatás teljes mértékben újat tud nyújtani a jelen
témában.
\\
\par A kutatás kapcsolódik a Szociálpszichológia tanszéken folytatott előző kutatásokhoz, de önálló
kérdést vizsgál meg.
Bevezető, elméleti háttér, hipotézisek megalapozása: A műhelymunkázó a kutatási kérdést
megfelelően helyezi mind társadalmi mind empirikus kontextusba. A szakirodalom
áttekintése átfogó és alapos. Formailag tekintve néhol ugrált a szakirodalom bemutatása
között, így nem tökéletesen valósul meg a tölcséres szerkezet. A bevezető (végé)ből
kimaradt a jelen kutatás rövid ismertetése, ami így a módszertani rész végéig nem áll össze
az olvasónak. Ezentúl, összességében érthető és jól megfogalmazott bevezetőt írt,
megfelelően megalapozza hipotéziseit.
\\
\par Módszertan: Az alacsony elemszám miatt össze lett vonva a két kísérleti feltétel (enyhe és súlyos
inzultus), és a korrelációs teszteket ezen az összesített mintán végezte el, ami elfogadható,
azonban a kutatás így nem kísérleti. Tehát például a bizalmi játék így már nem kísérleti
manipulációt képez, hanem egy eszköz, ami a konfrontáció és bizalom mérését teszi
lehetővé. Ugyanakkor mivel a kutatás kísérletnek indult, ezért ez a fogalmi zavar a
műhelymunkában érthető és elfogadható. Ehhez kapcsolódóan, legalább lábjegyzékben
fontos lett volna megjegyezni, hogy mi volt kezdetben az elméleti megalapozás, vagy akár
exploratív érdeklődés oka a súlyos illetve enyhe feltételre. Ezentúl, maga a kutatás
módszertana kiemelkedően alapos, kreatív és megfontolt. A módszertan közlése is nagyon
átfogó és minden szükséges elemet tartalmaz.
\\
\par Statisztikai elemzés, eredmények: A címben szereplő baráti kapcsolatok „hatása” szó használata
félrevezető lehet, ugyanis a műhelymunkázó korrelációs kapcsolatot vizsgált. Ezentúl,
megfelelő statisztikai elemzést használt, és azt megfelelően közölte. (Apróbb megjegyzések,
hogy a szöveget több bekezdésbe kellett volna tördelni, és az első ábra tartalmát szövegben
is lehetett volna közölni, a második-harmadik táblázat meg nem egészen APA-stílusnak
megfelelő, de valószínű, hogy a táblázat a margó meghaladása miatt lett ilyen
formátumban).
\\
\par Diszkusszió: A kutatási eredmények diszkussziója, ugyan túlságosan tagolt, de ennek ellenére
teljesen megfelelő, jól közli a jelen kutatás újdonságát, közben a dolgozat eredményeit jól
elhelyezi a korábbi eredmények között.

\vspace{3mm}

\begin{minipage}[t]{0.5\linewidth}
	\begin{flushleft}
		Érdemjegy: 5 (jeles) \\
		Szekeres Hanna \\
		Budapest, 2019. május 27.
	\end{flushleft}
\end{minipage}

