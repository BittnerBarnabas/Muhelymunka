\section*{Bevezetés}
A 21. századi fejlett, nyugati társadalmak köztudatában egy túlnyomóan pozitív attitűd figyelhető meg a boldogsággal kapcsolatban. Korábbi kutatások szerint, a boldogság nem csak a szubjektív jóllétünkhöz járul hozzá, de befolyásolja a fizikai és pszichológiai egészségünket is, kihat társas kapcsolatainkra, mindezek mellett pedig az egyik legfontosabb motiváló erő számos ember életében. Sőt, egyes kutatók szerint a szubjektív jóllét, tulajdonképpen maga a pozitív hatások jelenléte, a negatív hatások hiánya és az élettel való elégedettség.\cite{diener_suh_lucas_smith_1999} Arisztotelész, a nyugati filozófia atyja, ismert idézete szerint: \textit{“Happiness is the meaning and the purpose of life, the whole aim and end of human existence.”} \medskip 
\\ Azonban egyre bővül azoknak az empirikus kutatásoknak a száma, melyek megkérdő\-jelezik  a boldogság egyöntetűen pozitív megítélését. Léteznek ugyanis olyan populációk, ahol megfigyelhető egy sokkal ambivalensebb attitűd a boldogság felé. Például, kollektivista társadalmakon végzett vizsgálatok megállapították, hogy számos kultúrá\-ban kifejezetten tartanak a túlzott boldogságtól, nemkívánatosnak tartják azt \cite{joshanloo_weijers_2013} \cite{joshanloo_lepshokova_panyusheva_natalia_poon_yeung_sundaram_achoui_asano_igarashi}, de a nyugati társadalmakban is ugyanúgy megfigyelhető ez a jelenség \cite{gilbert_mcewan_catarino_baiao_palmeira_2013}.  Ennek a tanulmánynak a célja, hogy megvizsgálja a boldogság-averzióval, illetve a boldogságtól való félelemmel kapcsolatos szerteágazó kutatásokat és átfogó képet adjon a témával kapcsolatos jelenlegi, tudományos álláspontról. \medskip 

\pagebreak
\section* {Diszkusszió}
\subsection*{Aversion to Happiness Across Cultures: A Review
	of Where and Why People are Averse to Happiness \cite{joshanloo_weijers_2013}}
Joshanloo és Weijers (2013) cikke volt az egyik első kultúrközi tudományos munka, ami a boldogság-averzió koncepciójával foglalkozott. Tanulmányuk célja az volt, hogy korábbi empirikus kutatásokon keresztül megvizsgálják, hogy más-más kultúrákban milyen eltérő indokok miatt alakul ki averzió az emberekben, a boldogság különféle változataival szemben. Hipotézisük, vagyis hogy létezik a boldogsággal szembeni averzió, beigazolódott, és bár a keleti kultúrákban kétségkívül jobban megfigyelhető ez a jelenség, a nyugati kultúrákban is jelen van. A keleti kultúrákon végzett "fear of happiness" vagyis "boldogságtól való félelem" skálával való mérés során a kutatók azt találták, hogy eltérő mértékben ugyan, de minden vizsgált kultúrára jellemzőek voltak úgynevezett \textit{hiedelmek} \cite{joshanloo_weijers_2013}}. \medskip
 \\ A tanulmány 4 hiedelmet (belief) emel ki a boldogság-averzió vonatkozó mutatójaként \cite{joshanloo_weijers_2013}: 
\begin{enumerate}
	\item A boldogság megnöveli az esélyét, hogy valami rossz dolog fog történni velünk
	\item A boldogság rosszabb emberré tesz minket
	\item A boldogság kimutatása rossz hatással van ránk és másokra
	\item A boldogságra való törekvés rossz hatással van ránk és másokra

\end{enumerate}
Konklúzióként megállapíthatjuk, hogy valószínűleg ezek a szinte babonás hiedelmek állhatnak a túlzott boldogságtól való félelem hátterében. A boldogság-averzió egy érdekes és sokrétű jelenség, mely minden kultúrában megtalálható eltérő mértékben: Míg a keleti kultúrákban a vallás (pl.: buddhizmus és a vágyakról való lemondás), az erős konformitás és a (pozitív) érzelmek moderált kimutatása mind hozzájárulhat a hiedelmek megszilárdulásához, addig az individualista társadalmak egyén-központúsága és a pozitív érzelmek kimutatásának hangsúlyossága korlátozhatja azt \cite{joshanloo_lepshokova_panyusheva_natalia_poon_yeung_sundaram_achoui_asano_igarashi}.

\subsection*{Fears of compassion and happiness in relation	to alexithymia, mindfulness, and self-criticism \cite{gilbert_mcewan_gibbons_chotai_duarte_matos_2011}}
Gilbert és munkatársai (2011) az elsők közt kezdtek foglalkozni a boldogságtól való félelemmel. Empirikus tanulmányuk célja, hogy egy új, a kutatók által kidolgozott "Fear of Happiness" vagyis "Boldogságtól Való Félelem" skálával \cite{gilbert_mcewan_gibbons_chotai_duarte_matos_2011} megvizsgálják az összefüggéseket az odaadás (compassion) és a boldogságtól való félelem (fear of happiness), valamint olyan érzelem-feldolgozó kompetenciák közt, mint az  alexithymia, mindfulness, és empátia, illetve mindezek összefüggését az önkritikával és pszichopatológiával. Ezt a skálát Gilbert terápiás munkássága hatására fejlesztették ki a kutatók és olyan tételek tartalmazott, mint: \textit{"Félek, hogyha jól érzem magam, valami rossz fog történni"} \cite[o. 381]{gilbert_mcewan_gibbons_chotai_duarte_matos_2011}. A vizsgálat hipotézise, hogy összefüggést fognak találni az egyén odaadástól való félelme és az érzelem feldolgozása közt, amit a statisztikai elemzés be is bizonyított. Korrelációt találtak mind az odaadástól mind a boldogságtól való félelem és a szorongás, erős önkritika, és a mindfulness-el és alexithymia-val kapcsolatos nehézségek közt. A boldogságkutatással kapcsolatos legfontosabb eredmény, hogy a kutatók kiemelkedően magas (r=.70) korrelációt találtak a boldogságtól való félelem és a depresszió közt, ami bizonyítja a boldogságkutatás klinikai relevanciáját. A vizsgálat legnagyobb limitációja, hogy a kísérleti személyek 83\%-a nő volt, ami kétségessé teszi a minta reprezentatívságát \cite{gilbert_mcewan_gibbons_chotai_duarte_matos_2011}.

\subsection*{Fears of happiness and compassion in relationship
	with depression, alexithymia, and attachment
	security in a depressed sample \cite{gilbert_mcewan_catarino_baiao_palmeira_2013}}
Gilbert és munkatársainak (2011) eredménye, miszerint a boldogságtól való félelem magasan korrelál (r=0.7) a depresszióval, nagyban befolyásolta ezt a tudományos értekezést. Ennek a tanulmánynak a célja, hogy megvizsgálja a 2011-ben talált, főleg női egyetemistákon végzett kísérleti eredményeket depressziós mintán. A kutatók hipotézise az volt,  hogy \textsubscript{1} \textit{ A depressziós személyek nagyobb félelmet fognak mutatni a boldogság és odaadás (compassion) felé,}  mint a nagyrészt egyetemis\-tákból álló minta \textsubscript{2}A pozitív érzelmektől való félelem \textit{korrelálni fog a depresszióval, stresszel, szorongással, alexithymia-val }   \textsubscript{3} A pozitív érzelmektől való félelem, együtt fog járni \textit{a gyengébb minőségű felnőttkori kötődéssel} \cite{gilbert_mcewan_catarino_baiao_palmeira_2013}. Mindhárom hipotézis beigazolódott, de különösen érdemes megemlíteni, hogy a depressziós mintában magasabb a boldogságtól való félelem, mint a tanulókkal elvégzett kísérletben \cite{gilbert_mcewan_gibbons_chotai_duarte_matos_2011}, tehát a depressziós populáció jobban tart az extrém boldogságtól, mint az egyetemista. A boldogságtól való félelem magasan korrelált a felnőttkori bizonytalan kötődési stílusokkal, alexithymiaval, és a legpontosabban jósolta be a stresszt és szorongást \cite{gilbert_mcewan_catarino_baiao_palmeira_2013}. A vizsgálati eredményeknek gyakorlati relevanciája is van, hiszen segíthet mélyebben megérteni a depressziós populáció érzelmi kvalitását. Azonban fontos limitáció a vizsgálat introspektív mivolta, illetve a szociális kívánatosság befolyásoló ereje, amit nem lehet figyelmen kívül hagyni \cite{gilbert_mcewan_catarino_baiao_palmeira_2013}.

\subsection*{Fears of Negative Emotions in Relation to Fears of Happiness, Compassion,
	Alexithymia and Psychopathology in a Depressed Population \cite{gilbert_2014}}
 Ennek a tanulmánynak a célja, hogy megvizsgálja a \textit{kapcsolatot:}\\
 \textsubscript{1} Három negatív érzelemtől való félelem (szorongás, harag, szomorúság) és ezen érzelmek elkerülése közt \\
 \textsubscript{2} A negatív és pozitív érzelmektől való félelem közt.\\
 \textsubscript{3} Mindezek kapcsolatát az odaadással, alexithymiaval és pszichopatológiával.\\
 A hipotézis, vagyis, hogy a félt érzelmeket hajlamosabbak elkerülni az emberek, illetve, hogy korrelációt fognak találni a negatív és pozitív érzelmektől való félelem közt, beigazolódott. A boldogságkutatás szempontjából lényeges kiemelni, hogy a \textit{pozitív érzelmektől való félelem szignifikánsan korrelált a szorongástól, a haragtól és a szomorúságtól való félelemmel} és ezen érzelmek elkerülésével. \cite{gilbert_2014}. Érdemes azt is megjegyezni, hogy míg a szorongástól való félelem és a szorongás elkerülése közötti korreláció meglehetősen alacsony volt, addig a szomorúságtól való félelem és a szomorúság elkerülése, illetve főként a haragtól való félelem és a harag elkerülése közötti korreláció kifejezetten magas\cite{gilbert_2014}. E tanulmány eredményei alapján elmondhatjuk, hogy egyes érzelmektől való félelem, illetve adott érzelem elkerülése nagyban függ attól, hogy \textit{pontosan melyik} érzelemről beszélünk. A kutatás legfőbb limitációja, hogy viszonylag kis létszámú mintán (52 fő) végezték, többségben női résztvevőkkel.
 
\subsection*{Cross-Cultural Validation of
	Fear of Happiness Scale Across 14 National Groups \cite{joshanloo_lepshokova_panyusheva_natalia_poon_yeung_sundaram_achoui_asano_igarashi}}
Ez a széleskörű kultúrközi tanulmány 14 nemzeten vizsgálta a Joshanloo (2013) által kidolgozott "Fear of Happiness Scale-t (FHS)" azaz a "Boldogságtól Való Félelem Skálát". A kutatás célja részben e skála validitásának ellenőrzése volt, egy széleskörű mintán, és azon az elképzelésen alapul, hogy bizonyos kontextusokban az emberek nemkívánatosnak tartják a boldogságot, vagy egyenesen tartanak tőle \cite{joshanloo_lepshokova_panyusheva_natalia_poon_yeung_sundaram_achoui_asano_igarashi}. A vizsgálati hipotézisek, hogy \textsubscript{1} Egyének szintjén az FHS negatívan fog korrelálni az élettel való elégedettséggel \textsubscript{2} Kulturális szinten az FHS negatívan fog korrelálni a szubjektív-jólléttel \textsubscript{3} Bizonyos vallásos csoportokhoz való tartozás pozitívan fog korrelálni az FHS-el \cite{joshanloo_lepshokova_panyusheva_natalia_poon_yeung_sundaram_achoui_asano_igarashi}. Bár a skála validitása megfelelőnek bizonyult és mindhárom hipotézis teljesült, az FHS és a szubjektív-jóllét negatív korrelációja igen gyenge (\textless.15) volt, míg az FHS és az élettel való elégedettségé viszonylag magas. A tanulmány szerint olyan vallások, mint a \textit{Buddhizmus, Hinduizmus, Taoizmus} pozitívan, míg a \textit{Kereszténység} negatívan korrelál a boldogság-averzióval, ez valószínűleg az egyes vallások tanításaival hozható összefüggésbe \cite{joshanloo_lepshokova_panyusheva_natalia_poon_yeung_sundaram_achoui_asano_igarashi}. A vizsgálat legfőbb limitációja, hogy egyes nemzetek meglehetősen alulreprezentáltak, ami torzíthatja az eredményeket, valamint, hogy a résztvevők túlnyomórészt fiatal felnőttek voltak, ami megkérdőjelezi az eredmények általánosíthatóságát felnőtt populációkra.

\subsection*{Eastern Conceptualizations of Happiness: Fundamental
	Differences with Western Views \cite{joshanloo_2013_eastern}}
Joshanloo számos munkájában hangot adott kritikájának, miszerint a boldogságku\-tatás főként nyugati kultúrák boldogságkoncepcióján alapszik, s így nem feltétlenül általánosítható más kultúrákra \cite{joshanloo_2013}. Ebben a tudományos értekezésében éppen erre a feladatra vállalkozik, elkülöníti a keleti és nyugati kultúrák boldogságkoncepcióit és kiemeli a legalapvetőbb kulturális különbségeket: \\
\textsubscript{1} \textit{Hedonizmus} és \textit{Eudaimónia}\\
E két különböző boldogságkoncepció már az ókori görög filozófiában is heves viták alanya volt. Míg az eudaimónia tana szerint a boldogság tulajdonképpen az erények gyakorlásának következménye, a hedonizmus a testi örömöket a lelki felé helyezi, sze\-rinte a boldogság a gyönyörrel, élvezettel teli élet. Joshanloo a keleti boldogságkoncep\-ciót az antik eudaimóniával, a nyugatit pedig a hedonizmussal azonosította. \\
\textsubscript{2}\textit{Self-Transzcendencia} és \textit{Self-Kiemelés}\\
Míg a nyugati kultúrák általában az individualista értékek alapján határozzák meg a selfet, addig a keleti kultúrák hajlamosak a nagy egész, a kollektív, a kozmosz kis részeként azonosítani azt \cite{joshanloo_2013_eastern}.\\
\textsubscript{3}\textit{Harmónia} és \textit{Uralás}\\
A nyugati világnézet szerint az emberiség egy privilegizált faj, ami intelligenciájának köszönhetően uralni képes a teremtés más entitásait (Sibley idézve\cite{joshanloo_2013_eastern}. Ehhez mérten az egyének törekszenek befolyásolni, uralni és kontrollálni környezetüket, azonban mivel a keleti kultúrákban nincs jelen ez a felsőbbrendűség érzet, itt az egyének inkább a harmónia megteremtésére igyekszenek más élőlényekkel, és a kozmosszal\cite{joshanloo_2013_eastern}.\\
\textsubscript{4}\textit{Elégedettség}}\\
Johanloo (2013) szerint a legnagyobb különbség az élettel való elégedettség dimenzióban, hogy a keleti kultúrákban ez a koncepció magába foglalja a megbékélést önmagunkkal, másokkal és az egész kozmosszal, illetve nem befolyásolhatja az elégedett\-séget a célok elérése és a másokkal való összehasonlítás\cite{joshanloo_2013_eastern}.\\
\textsubscript{5}\textit{Szenvedés értékelése} és \textit{Szenvedés elkerülése}\\
Ahogy már korábban is említettem, egyes kutatók szerint a szubjektív jóllét a pozitív hatások jelenlétének és a negatív hatások hiányának összessége \cite{diener_suh_lucas_smith_1999}. Joshanloo azonban kiemeli, hogy a legtöbb keleti kultúrában elfogadott teória, hogy a bánat része az igazán boldog életnek sőt, a szenvedés elengedhetetlen összetevője a boldogságnak, a szenvedés nélküli boldogság nem teljes. \cite{joshanloo_2013_eastern}.\\
\textsubscript{6}\textit{Spiritualitás és Vallás}\\
Az egyik legnagyobb boldogságkoncepcióbeli kulturális különbség ebben a dimenzi\-óban található: Míg a nyugati kultúrákban alapvető, hogy a boldogságot és jóllétet ebben a földi életben kell megtapasztalnunk \cite{joshanloo_2013_eastern}, a keleti kultúrákban ez nem ilyen egyértelmű. Egyrészt számos keleti vallás rendelkezik transzcendens tanításokkal, ami a jelen életbeli lemondásokra utasít a következő életbeli boldogság eléréséhez. Másrészt olyan spirituális élmények, mint az Istennel való egység, kétség\-kívül része a keleti boldogságkoncepcióknak, ám számos nyugati modell elveti ezeket az élményeket, babonásnak és primitívnek címkézve őket \cite{joshanloo_2013_eastern}.\\
Összességében elmondható tehát, hogy számos kulturális különbség létezik az emberek boldogságkoncepciójában, ami pedig szükségessé teszi egy árnyaltabb, kulturális eltérésekre érzékenyebb boldogságkutatás kialakítását.

\subsection*{Összegzés}
Láthatjuk tehát, hogy a boldogsággal kapcsolatos attitűdök közel sem olyan egyhangúak, mint azt korábbi kutatásokból gondoltuk volna. Megfigyelhetőek ugyanis bizonyos hiedelmek \cite{joshanloo_weijers_2013} melyek kimondottan nemkívána\-tossá teszik számunkra a túlzott boldogság átélését, mint például, hogy \textit{minden jót valami rossz kell kövessen}, tehát ha jelenleg boldogok vagyunk, \textit{valami rossz fog következni}. Keleti kultúrákban megfigyelhető például az a hiedelem is, hogy saját boldogságunk kimutatása rossz hatással van másokra \cite{joshanloo_weijers_2013}, mert iriggyé teszi őket, illetve, hogy a boldogságra való törekvés nem helyes, mert \textit{elvonja az egyén figyelmét} más, fontosabb dolgokról \cite{joshanloo_weijers_2013}. Gilbert és munkatársai megállapították, hogy az ezen hiedelmek hatására kialakult, pozitív érzelmektől való félelem, pozitívan korrelál a \textit{depresszióval, a negatív érzelmektől való félelemmel, stresszel, szorongással}l és alexithymiaval, illetve, hogy együtt jár a felnőttkori bizonytalan kötődési stílusokkal \cite{gilbert_mcewan_catarino_baiao_palmeira_2013}. Gilbert és munkatársainak eredményei nem csak bizonyítják a boldogságkutatás \textit{klinikai relevanciáját}, de termékeny talajt is biztosítanak jövőbeli empirikus kutatásokhoz, hiszen a boldogságtól való félelem még mindig egy viszonylag fiatal témának számít a pozitív pszichológiában. \\
Azonban a boldogságkutatásnak jelene számos limitációja van, többek közt az introspektív jelleg és a nem megfelelő kulturális érzékenység. Joshanloo (2013) kultúrközi boldogságkoncepció vizsgálatából láthatjuk, hogy nem csak, hogy a boldog\-sággal kapcsolatos attitűdökben figyelhetőek meg kulturális különbségek, hanem már eleve a \textit{boldogságkoncepcióban is.}













































