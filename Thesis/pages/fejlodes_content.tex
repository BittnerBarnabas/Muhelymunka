\section*{Bevezetés}
A tudatelmélet, vagyis az a képesség, hogy másoknak sajátunktól eltérő mentális állapoto\-kat,
illetve szándékokat tulajdonítunk, a gyermekkori kognitív fejlődés fontos állomása. A
tudatelmélet lehetséges mérőeszközei a téves-vélekedés tesztek, melyek azt vizsgálják, hogy a
gyermekek képesek-e felismerni, hogy az emberek információ hiányában hamis vélekedést
alakíthatnak ki egy szituációról. Egy másik lehetséges mérőeszköz a látszat-valóság teszt,
aminek szintén számos fajtája létezik és elterjedten alkalmazzák a tudatelmélet vizsgálatára.
Ezen tesztek segítségével a kutatók kialakítottak egy konszenzust, mely szerint a tipikusan
fejlődő gyermekek körülbelül 4 éves korukra képesek hamis vélekedést tulajdonítani
másoknak, tehát 4 éves korban a gyermekek szignifikánsan jobban teljesítenek mind a hamis-
vélekedés, mind a látszat-valóság teszteken, mint fiatalabb társaik \autocite{wimmer_1983,perner_leekam_wimmer_1987}. \\
\\
Azonban újabb vizsgálatok kimutatták, hogy egyéni szinten megfigyelhető egy viszonylag széles variancia az életkorban, amikor a gyermekek képesek jól megoldani a hamis-vélekedés tesztet \autocite{jenkins_astington_1996}. \textcite{jenkins_astington_1996} kutatásukban azt találták, hogy számos gyermek már 3 éves korban is
átment a teszten, míg másoknak 5 éves korukra sikerült csak megoldaniuk azt. Mi okozhatja
tehát ezt a drámai egyéni különbséget a gyermekek tudatelméletének fejlődésében? Az egyik
válasz a \textit{nyelv}. Számos vizsgálat foglalkozik azzal, hogy hogyan befolyásolja a gyermekek
nyelvi képessége tudatelméletük fejlődését. \textcite{milligan_astington_dack_2007} például
kimutatta, hogy korreláció figyelhető meg a gyermekek nyelvi képességei és a téves-
vélekedés megértése között, életkortól függetlenül. Ezek alapján különösen izgalmas kérdés,
hogy milyen szintű tudatelmélettel rendelkeznek azok a siket gyermekek, akiknek nyelvi és
kommunikációs fejlődése eltér a tipikustól. Ennek a tanulmánynak a célja, hogy feltérképezze
a halláskáro\-sult gyermekek tudatelméletével foglalkozó szerteágazó vizsgálatokat, és átfogó
képet adjon a jelenlegi, tudományos álláspontról.

\pagebreak

\section*{Diszkusszió}

A legtöbb, halláskárosult gyermekek tudatelméletével foglalkozó tanulmány, megkülön-böztet
három csoportot a siket gyermekek közt, az alapján, hogy mennyire férnek hozzá a
jelnyelven, vagy beszédben folytatott kommunikációhoz családjukon belül:
\begin{enumerate}
	\item Azon gyermekek csoportja, akik anyanyelve a jelnyelv és jelnyelven képesek
	beszél\-getést folytatni valamelyik családtagukkal
	\item Orálisan képzett siket gyermekek, akik képesek szóban beszélgetést folytatni
	\item Azon siket gyermekek csoportja, akik csak később, iskolás éveik alatt sajátítják el a
	jelbeszédet, éveken át tartó kommunikációs hiány után
\end{enumerate}
Az ezen tanulmány keretein belül feldolgozott kutatások egyöntetűen azt az álláspontot
hangsúlyozzák, hogy azok a gyermekek, akik az első csoportba tartoznak, vagyis anyanyelvük
a jelnyelv, s azt már születésüktől fogva tanulták, szignifikánsan jobban teljesítenek a hamis-
vélekedés teszteken, mint azok a társaik, akik csak később sajátították el azt \autocite{woolfe_want_siegal_2002,peterson_siegal_1999,peterson_slaughter_2006,schick_villiers_villiers_hoffmeister_2007}.
\textcite{woolfe_want_siegal_2002} kutatásukban azt találták, hogy a jelnyelven anyanyelvi
szinten kommunikáló csoport még akkor is jobban teljesített a hamis-vélekedés teszten, ha a
kutatók olyan faktorokra is kontrolláltak, mint a végrehajtó funkciók, illetve a gyermekek
mondattani képességei.\\
\\
 \textcite{peterson_siegal_1999} vizsgálatukban egy egész sor téves-
vélekedés tesztet töltettek ki a résztvevő siket és autista gyermekekkel. Ezek közül az egyik például a híres Sally-Anne teszt \autocite{baron-cohen_leslie_frith_1985} volt, amit ma is széles körben alkalmaznak a tudatelmélet vizsgálatára. Az előző kutatáshoz
hasonlóan azt találták, hogy a jel anyanyelvű gyermekek szignifikánsan jobban teljesítettek
azoknál a társaiknál, akik később tanulták meg a jelnyelvet \autocite{peterson_siegal_1999}. Sőt, a jel anyanyelvű,
illetve orálisan képzett gyermekek ugyanolyan jól teljesítettek a teszteken, mint a halló
gyermekekből álló kontrollcsoport \autocite{peterson_siegal_1999}. Peterson és Siegal (1999) szerint ez abból a
különbségből adódik, hogy a siket szülők ugyanolyan könnyedséggel kommunikálnak siket
gyermekükkel távoli, hipotetikus, vagy elképzelt dolgokról, mint halló szülők a halló
gyermekükkel \autocite{peterson_siegal_1999}. Azonban közös nyelv hiányában a halló szülők egyáltalán nem, vagy
csak nagyon limitált formában képesek kommunikálni halláskárosult gyermekükkel minden
olyan dologról, ami nem az „itt és most”-hoz tartozik. \autocite{peterson_siegal_1999}. Ezek az adatok azért
kiemelkedően fontosak, mert alátámasztják azt az elképzelést, hogy a korai dialógusnak
központi szerepe van a gyermekek tudatelméletének fejlődésében. \\
\\
Ezt az elképzelést támasztja alá \textcite{peterson_slaughter_2006} vizsgálata is, ahol nem csak siket gyermekek
tudatelméletét vizsgálták, hanem spontán narratív beszédüket is olyan kategóriákban, mint \textit{a
	valóság-orientált kogníció} (melléknevek, főnevek, igék pl.: okos, tudja, gondolja), \textit{képzelet-
	orientált kogníció } (melléknevek, főnevek, igék pl.: színlel, álmodik), \textit{percepció} (melléknevek,
főnevek, igék, amik az információszerzésre utalnak, valamilyen érzékszerven keresztül pl.:
lát, hall), illetve \textit{a vágyak és érzelmek}. A vizsgálatban 21 siket (mindegyikük csak iskolás kortól
tanult jelnyelvet) és 13 halló gyermek vett részt. Az előző tanulmányokhoz hasonlóan, a
jelnyelvet későn elsajátító siket gyermekek itt is szignifikánsan rosszabbul teljesítettek a
hamis-vélekedés teszten, mint halló társaik \autocite{peterson_slaughter_2006}. Azonban különösen figyelemre méltó
eredmény, hogy azok a halláskárosult gyerekek, akik a narratív vizsgálatban többször
beszéltek a mesében szereplő karakterek kognícióiról (és főként a képzelet-orientált
kogní\-ciókról, mint pl.: a színlelés) nagyobb valószínűséggel oldották meg a téves-vélekedés
tesztet is \autocite{peterson_slaughter_2006}.\\
\\
Egy másik megközelítésben a kutatók arra voltak kíváncsiak, hogy vajon a halláskárosult
gyermekek tudatelméletének fejlődési hátránya akkor is jelentkezik-e, ha a hamis-véleke-dés
tesztben a nyelvi követelményeket minimalizálják, s a tesztnek egy non-verbális válto\-zatát
prezentálják a gyermekeknek \autocite{figueras-costa_harris_2001}. \textcite{figueras-costa_harris_2001}
vizsgálatukban 21 fő orálisan képzett, hallókészüléket viselő halláskárosult kisgyermekkel
dolgozott, s céljuk az volt, hogy megállapítsák: a gyermekeknek valóban fejlődési hátránya
van a tudatelmélet elsajátításában vagy ez csak egy látszólagos lemaradás, amit a feladat
megértési nehézsége okoz? Ennek a kérdésnek a megválaszolására a kutatók mind verbális
mind non-verbális hamis-vélekedés tesztnek is alávetették a résztve-vőket, s azt találták, hogy
a siket gyermekek a non-verbális teszten minden esetben jobban teljesítettek, életkortól
függetlenül \autocite{figueras-costa_harris_2001}. Azonban a szerzők figyelmeztetnek arra, hogy ez az eredmény nem azt
jelenti, hogy a halláskárosult gyermekek tudatelméletének fejlődésében megfigyelt hátrány
csupán a verbálisan prezentált feladat megértésének nehézségét tükrözi. Bár az kétségtelen,
hogy a non-verbális feladat facilitálta a gyermekek teljesítményét, a gyermekek még így is
rosszabbul teljesítettek, mint az elvárt lett volna az életkoruk alapján \autocite{figueras-costa_harris_2001}. A non-verbális
téves-vélekedés feladat megoldásának életkori átlaga 8 év 10 hónap volt ebben a vizsgálatban,
ami körülbelül 4 évnyi fejlődési hátrányt sugall a tipikus fejlődésű gyermekekhez képest
\autocite{figueras-costa_harris_2001}.

\pagebreak

\section*{Összegzés}

Az ezen tanulmány keretein belül áttektintett releváns irodalmak alapvetően két fontos
pontban találkoznak. Először is mindegyik vizsgálat azt az eredményt hozta, hogy a jelnyelvet
csak később, iskolás korban elsajátító gyermekek rosszabbul teljesítenek a hamis-vélekedés
teszteken, nem csak halló, de jel anyanyelvű, illetve orálisan képzett társaiknál is \autocite{peterson_slaughter_2006,woolfe_want_siegal_2002,peterson_siegal_1999}.
Egyesek szerint ezt a különbséget az a kommunikációs depriválás okozza, ami a halló
családok körébe született halláskárosult gyermekek életének első éveit jellemzi \autocite{perner_leekam_wimmer_1987}. Mivel a
halló szülőknek sok esetben nincs kielégítő kommunikációs csatornája a siket gyermekkel, a
szülő-gyermek kommunikáció így csupán a jelenre, s a fizikai környezetben könnyen
referálható dolgokra korlátozódik \autocite{peterson_siegal_2000}. Mivel a mentális állapotok elvont, nehezen
referálható entitások, a halló szülők szinte egyáltalán nem, vagy nagyon ritkán képesek
megosztani azokat siket utódaikkal, ez pedig hozzájárulhat a tudatelméletük hátrányos
fejlődéséhez \autocite{peterson_siegal_2000}. Másodszor, a vizsgálatok mindegyike egyetért abban, hogy bár a
jel anyanyelvű családba született gyermekek szinte ugyanúgy teljesítenek a hamis-vélekedés
feladatokban, mint halló társaik, akik csak később tanulnak meg jelnyelven kommunikálni,
szignifikánsan rosszabbul teljesítenek, mint az azonos életkorú, tipikus fejlődésű társaik \autocite{peterson_slaughter_2006,woolfe_want_siegal_2002,peterson_siegal_1999}.
Ez a jelenség akkor is fennállt, amikor a nyelvi követelményeket minimalizálták a kutatók
\autocite{figueras-costa_harris_2001}. Ezek az eredmények különösen fontosak a halláskárosult gyermekek integrációja
szempontjából, hiszen más emberek érzéseinek, vágyainak feltételezése és megértése alapvető
követelmény a legtöbb társas helyzetben. Az eredmények azonban arra is rávilágítanak, hogy
a korai kommunikációnak elengedhetetlenül fontos szerepe van a gyermekek
tudatelméletének fejlődésében, annak modalitásától függetlenül \autocite{peterson_2004}.