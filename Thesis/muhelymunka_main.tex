\documentclass[12pt]{article}

\usepackage[utf8]{inputenc}
\usepackage{listings}
\usepackage{color}
\usepackage[pdftex]{graphicx}
\usepackage{setspace}
\usepackage[nottoc, numbib]{tocbibind}
\usepackage[
	a4paper,
	top=2.5cm,
	bottom=2.5cm,
	inner=3.5cm,
	outer=2.5cm
]{geometry}
\usepackage{times}
\usepackage[document]{ragged2e}
\usepackage{hyperref}
\linespread{1.5}
\usepackage{float}
\usepackage{listings}
\usepackage{textcomp}
\usepackage{subfiles}
\usepackage{xcolor}
\usepackage{titlesec}
\setcounter{secnumdepth}{4}

\usepackage{fancyhdr}
\pagestyle{fancy}
\fancyhf{}
\fancyhead[L]{\rightmark }
\fancyhead[R]{\thepage}
\usepackage{spverbatim}
\usepackage[magyar]{babel}
\usepackage[backend = biber, backref = true, style=apa, maxcitenames=2]{biblatex}
\usepackage{hyperref}

%%START- This is needed to replace ampersands with and 
\DeclareDelimFormat*{finalnamedelim}
{\ifnum\value{liststop}>2 \finalandcomma\fi\addspace\bibstring{and}\space}

% the bibliography also needs another conditional, so we can't wrap
% everything up with just the two lines above
\DeclareDelimFormat[bib,biblist]{finalnamedelim}{%
	\ifthenelse{\value{listcount}>\maxprtauth}
	{}
	{\ifthenelse{\value{liststop}>2}
		{\finalandcomma\addspace\bibstring{and}\space}
		{\addspace\bibstring{and}\space}}}

% this is a special delimiter to solve the bugs reported in
% https://tex.stackexchange.com/q/417648/35864
\DeclareDelimFormat*{finalnamedelim:apa:family-given}{%
	\ifthenelse{\value{listcount}>\maxprtauth}
	{}
	{\finalandcomma\addspace\bibstring{and}\space}}
%%END - This is needed to replace ampersands with and 

\DefineBibliographyStrings{english}{%
		bibliography     = {Irodalomjegyz\'ek},
		references       = {Hivatkoz\'asok},
		shorthands       = {R\"ovid\'it\'esek jegyz\'eke},
		editor           = {szerkeszt\H{o}},
		editors          = {szerkeszt\H{o}k},
		compiler         = {\"ossze\'all\'it\'o},
		compilers        = {\"ossze\'all\'it\'ok},
		redactor         = {sajt\'o al\'a rendez\H{o}},
		redactors        = {sajt\'o al\'a rendez\H{o}k},
		reviser          = {korrektor},
		revisers         = {korrektorok},
		founder          = {alap\'it\'o},
		founders         = {alap\'it\'ok},
		%continuator      = {},
		%continuators     = {},% FIXME missing
		collaborator     = {k\"ozrem\H{u}k\"od\H{o}},
		collaborators    = {k\"ozrem\H{u}k\"od\H{o}k},
		translator       = {ford\'it\'o},
		translators      = {ford\'it\'ok},
		commentator      = {komment\'ator},
		commentators     = {komment\'atorok},
		annotator        = {jegyzetek \'ir\'oja},
		annotators       = {jegyzetek \'ir\'oi},
		commentary       = {komment\'ar},
		annotations      = {jegyzetek},
		introduction     = {bevezet\'es},
		foreword         = {el\H{o}sz\'o},
		afterword        = {ut\'osz\'o},
		editortr         = {szerkeszt\H{o} \'es ford\'it\'o},
		editorstr        = {szerkeszt\H{o}k \'es ford\'it\'ok},,
		editorco         = {szerkeszt\H{o} \'es komment\'ar},
		editorsco        = {szerkeszt\H{o} \'es komment\'ar},
		editoran         = {szerkeszt\H{o} \'es jegyzet},
		editorsan        = {szerkeszt\H{o} \'es jegyzet},
		editorin         = {szerkeszt\H{o} \'es bevezet\H{o}},
		editorsin        = {szerkeszt\H{o}k \'es bevezet\H{o}},
		editorfo         = {szerkeszt\H{o} \'es el\H{o}sz\'o},
		editorsfo        = {szerkeszt\H{o}k \'es el\H{o}sz\'o},
		editoraf         = {szerkeszt\H{o} \'es ut\'osz\'o}
		editorsaf        = {szerkeszt\H{o}k \'es ut\'osz\'o},
		editortrco       = {szerkeszt\H{o}, ford\'it\'o \'es komment\'ar},
		editorstrco      = {szerkeszt\H{o}, ford\'it\'o \'es komment\'ar},
		editortran       = {szerkeszt\H{o}, ford\'it\'o \'es jegyzet},
		editorstran      = {szerkeszt\H{o}, ford\'it\'o \'es jegyzet},
		editortrin       = {szerkeszt\H{o}, ford\'it\'o \'es bevezet\H{o}},
		editorstrin      = {szerkeszt\H{o}, ford\'it\'o \'es bevezet\H{o}},
		editortrfo       = {szerkeszt\H{o}, ford\'it\'o \'es el\H{o}sz\'o},
		editorstrfo      = {szerkeszt\H{o}, ford\'it\'o \'es el\H{o}sz\'o},
		editortraf       = {szerkeszt\H{o}, ford\'it\'o \'es ut\'osz\'o},
		editorstraf      = {szerkeszt\H{o}, ford\'it\'o \'es ut\'osz\'o},
		editorcoin       = {szerkeszt\H{o}, komment\'ar \'es bevezet\H{o}},
		editorscoin      = {szerkeszt\H{o}, komment\'ar \'es bevezet\H{o}},
		editorcofo       = {szerkeszt\H{o}, komment\'ar \'es el\H{o}sz\'o},
		editorscofo      = {szerkeszt\H{o}, komment\'ar \'es el\H{o}sz\'o},
		editorcoaf       = {szerkeszt\H{o}, komment\'ar \'es ut\'osz\'o},
		editorscoaf      = {szerkeszt\H{o}, komment\'ar \'es ut\'osz\'o},
		editoranin       = {szerkeszt\H{o}, jegyzetek \'es bevezet\H{o}},
		editorsanin      = {szerkeszt\H{o}, jegyzetek \'es bevezet\H{o}},
		editoranfo       = {szerkeszt\H{o}, jegyzetek \'es el\H{o}sz\'o},
		editorsanfo      = {szerkeszt\H{o}, jegyzetek \'es el\H{o}sz\'o},
		editoranaf       = {szerkeszt\H{o}, jegyzetek \'es ut\'osz\'o},
		editorsanaf      = {szerkeszt\H{o}, jegyzetek \'es ut\'osz\'o},
		editortrcoin     = {szerkeszt\H{o}, ford\'it\'o, komment\'ar \'es bevezet\H{o}},
		editorstrcoin    = {szerkeszt\H{o}, ford\'it\'o, komment\'ar \'es bevezet\H{o}},
		editortrcofo     = {szerkeszt\H{o}, ford\'it\'o, komment\'ar \'es el\H{o}sz\'o},
		editorstrcofo    = {szerkeszt\H{o}, ford\'it\'o, komment\'ar \'es el\H{o}sz\'o},
		editortrcoaf     = {szerkeszt\H{o}, ford\'it\'o, komment\'ar \'es ut\'osz\'o},
		editorstrcoaf    = {szerkeszt\H{o}, ford\'it\'o, komment\'ar \'es ut\'osz\'o},
		editortranin     = {szerkeszt\H{o}, ford\'it\'o, jegyzetek \'es bevezet\H{o}},
		editorstranin    = {szerkeszt\H{o}, ford\'it\'o, jegyzetek \'es bevezet\H{o}},
		editortranfo     = {szerkeszt\H{o}, ford\'it\'o, jegyzetek \'es el\H{o}sz\'o},
		editorstranfo    = {szerkeszt\H{o}, ford\'it\'o, jegyzetek \'es el\H{o}sz\'o},
		editortranaf     = {szerkeszt\H{o}, ford\'it\'o, jegyzetek \'es ut\'osz\'o},
		editorstranaf    = {szerkeszt\H{o}, ford\'it\'o, jegyzetek \'es ut\'osz\'o},
		translatorco     = {ford\'it\'o \'es komment\'ar},
		translatorsco    = {ford\'it\'o \'es komment\'ar},
		translatoran     = {ford\'it\'o \'es jegyzetek},
		translatorsan    = {ford\'it\'o \'es jegyzetek},
		translatorin     = {ford\'it\'o \'es bevezet\H{o}},
		translatorsin    = {ford\'it\'o \'es bevezet\H{o}},
		translatorfo     = {ford\'it\'o \'es el\H{o}sz\'o},
		translatorsfo    = {ford\'it\'o \'es el\H{o}sz\'o},
		translatoraf     = {ford\'it\'o \'es ut\'osz\'o},
		translatorsaf    = {ford\'it\'o \'es ut\'osz\'o},
		translatorcoin   = {ford\'it\'o, komment\'ar \'es bevezet\H{o}},
		translatorscoin  = {ford\'it\'o, komment\'ar \'es bevezet\H{o}},
		translatorcofo   = {ford\'it\'o, komment\'ar \'es el\H{o}sz\'o},
		translatorscofo  = {ford\'it\'o, komment\'ar \'es el\H{o}sz\'o},
		translatorcoaf   = {ford\'it\'o, komment\'ar \'es ut\'osz\'o},
		translatorscoaf  = {ford\'it\'o, komment\'ar \'es ut\'osz\'o},
		translatoranin   = {ford\'it\'o, jegyzetek \'es bevezet\H{o}},
		translatorsanin  = {ford\'it\'o, jegyzetek \'es bevezet\H{o}},
		translatoranfo   = {ford\'it\'o, jegyzetek \'es el\H{o}sz\'o},
		translatorsanfo  = {ford\'it\'o, jegyzetek \'es el\H{o}sz\'o},
		translatoranaf   = {ford\'it\'o, jegyzetek \'es ut\'osz\'o},
		translatorsanaf  = {ford\'it\'o, jegyzetek \'es ut\'osz\'o},
		organizer        = {szervez\H{o}},
		organizers       = {szervez\H{o}k},
		byorganizer      = {szervezte},
		byauthor         = {\'irta},
		byeditor         = {szerkesztette},
		bycompiler       = {\"ossze\'all\'itotta},
		byredactor       = {sajt\'o al\'a rendezte},
		byreviser        = {jav\'itotta},
		byreviewer       = {b\'ir\'alta},
		byfounder        = {alap\'itotta},
		% bycontinuator    = {},% FIXME missing
		bycollaborator   = {k\"ozrem\H{u}k\"od\"ott},
		bytranslator     = {\lbx@lfromlang ford\'itotta},
		bycommentator    = {komment\'alta},
		byannotator      = {jegyzetekkel ell\'atta},
		withcommentator  = {komment\'arral ell\'atta},
		withannotator    = {jegyzetekkel ell\'atta},
		withintroduction = {bevezet\H{o}vel ell\'atta},
		withforeword     = {el\H{o}sz\'oval ell\'atta},
		withafterword    = {ut\'osz\'oval ell\'atta},
		byeditortr       = {szerkesztette \'es \lbx@lfromlang ford\'itotta},
		byeditorco       = {szerkesztette \'es komment\'alta},
		byeditoran       = {szerkesztette \'es jegyzetekkel ell\'atta},
		byeditorin       = {szerkesztette \'es bevezet\H{o}vel ell\'atta},
		byeditorfo       = {szerkesztette \'es el\H{o}sz\'oval ell\'atta},
		byeditoraf       = {szerkesztette \'es ut\'osz\'oval ell\'atta},
		byeditortrco     = {szerkesztette, \lbx@lfromlang ford\'itotta \'es komment\'alta},
		byeditortran     = {szerkesztette, \lbx@lfromlang ford\'itotta \'es jegyzetekkel ell\'atta},
		byeditortrin     = {szerkesztette, \lbx@lfromlang ford\'itotta \'es bevezet\H{o}vel ell\'atta},
		byeditortrfo     = {szerkesztette, \lbx@lfromlang ford\'itotta \'es el\H{o}sz\'oval ell\'atta},
		byeditortraf     = {szerkesztette, \lbx@lfromlang ford\'itotta \'es ut\'osz\'oval ell\'atta},
		byeditorcoin     = {szerkesztette, komment\'alta \'es bevezet\H{o}vel ell\'atta},
		byeditorcofo     = {szerkesztette, komment\'alta \'es el\H{o}sz\'oval ell\'atta},
		byeditorcoaf     = {szerkesztette, komment\'alta \'es ut\'osz\'oval ell\'atta},
		byeditoranin     = {szerkesztette, jegyzetekkel \'es bevezet\H{o}vel ell\'atta},
		byeditoranfo     = {szerkesztette, jegyzetekkel \'es el\H{o}sz\'oval ell\'atta},
		byeditoranaf     = {szerkesztette, jegyzetekkel \'es ut\'osz\'oval ell\'atta},
		byeditortrcoin   = {szerkesztette, \lbx@lfromlang ford\'itotta, komment\'alta \'es bevezet\H{o}vel ell\'atta},
		byeditortrcofo   = {szerkesztette, \lbx@lfromlang ford\'itotta, komment\'alta \'es el\H{o}sz\'oval ell\'atta},
		byeditortrcoaf   = {szerkesztette, \lbx@lfromlang ford\'itotta, komment\'alta \'es ut\'osz\'oval ell\'atta},
		byeditortranin   = {szerkesztette, \lbx@lfromlang ford\'itotta, jegyzetekkel \'es bevezet\H{o}vel ell\'atta},
		byeditortranfo   = {szerkesztette, \lbx@lfromlang ford\'itotta, jegyzetekkel \'es el\H{o}sz\'oval ell\'atta},
		byeditortranaf   = {szerkesztette, \lbx@lfromlang ford\'itotta, jegyzetekkel \'es ut\'osz\'oval ell\'atta},
		bytranslatorco   = {\lbx@lfromlang ford\'itotta \'es komment\'alta},
		bytranslatoran   = {\lbx@lfromlang ford\'itotta \'es jegyzetekkel ell\'atta},
		bytranslatorin   = {\lbx@lfromlang ford\'itotta \'es bevezet\H{o}vel ell\'atta},
		bytranslatorfo   = {\lbx@lfromlang ford\'itotta \'es el\H{o}sz\'oval ell\'atta},
		bytranslatoraf   = {\lbx@lfromlang ford\'itotta \'es ut\'osz\'oval ell\'atta},
		bytranslatorcoin = {\lbx@lfromlang ford\'itotta, komment\'alta \'es bevezet\H{o}vel ell\'atta},
		bytranslatorcofo = {\lbx@lfromlang ford\'itotta, komment\'alta \'es el\H{o}sz\'oval ell\'atta},
		bytranslatorcoaf = {\lbx@lfromlang ford\'itotta, komment\'alta \'es ut\'osz\'oval ell\'atta},
		bytranslatoranin = {\lbx@lfromlang ford\'itotta, jegyzetekkel \'es bevezet\H{o}vel ell\'atta},
		bytranslatoranfo = {\lbx@lfromlang ford\'itotta, jegyzetekkel \'es el\H{o}sz\'oval ell\'atta},
		bytranslatoranaf = {\lbx@lfromlang ford\'itotta, jegyzetekkel \'es ut\'osz\'oval ell\'atta},
		and              = {\'es},
		andothers        = {\'es mtsai\addot},
		andmore          = {et\addabbrvspace al\adddot},
		volume           = {k\"otet},
		volumes          = {k\"otetek},
		involumes        = {},
		jourvol          = {\'evfolyam},
		jourser          = {sorozat},
		book             = {k\"onyv},
		part             = {r\'esz},
		issue            = {sz\'am},
		newseries        = {\'uj sorozat},
		oldseries        = {r\'egi sorozat},
		edition          = {kiad\'as},
		reprint          = {ut\'annyom\'as},
		reprintof        = {ut\'annyom\'asa},% FIXME swap word order
		reprintas        = {ut\'annyom\'as c\'ime},
		% reprintfrom      = {},% FIXME need example
		translationof    = {ford\'it\'asa},% FIXME swap word order
		translationas    = {ford\'it\'as c\'ime},
		% translationfrom  = {},% FIXME need example
		reviewof         = {recenzi\'oja},% FIXME swap word order
		origpubas        = {eredeti c\'ime},
		origpubin        = {eredeti kiad\'as},
		% astitle          = {},% FIXME need example
		% bypublisher      = {\addcomma},% FIXME: this is a cludge that can not be guaranteed to work in most situations, so it stays commented out
		nodate           = {\'ev n\'elk\"ul},
		page             = {oldal},
		pages            = {oldal},
		column           = {has\'ab},
		columns          = {has\'ab},
		line             = {sor},
		lines            = {sor},
		verse            = {versszak},
		verses           = {versszak},
		section          = {paragrafus},
		sections         = {paragrafus},
		paragraph        = {bekezd\'es},
		paragraphs       = {bekezd\'es},
		pagetotal        = {oldal},
		pagetotals       = {oldal},
		columntotal      = {has\'ab},
		columntotals     = {has\'ab},
		linetotal        = {sor},
		linetotals       = {sor},
		versetotal       = {versszak},
		versetotals      = {versszak},
		sectiontotal     = {paragrafus},
		sectiontotals    = {paragrafus},
		paragraphtotal   = {bekezd\'es},
		paragraphtotals  = {bekezd\'es},
		in               = {},
		% inseries         = {},% FIXME need example
		% ofseries         = {},% FIXME need example
		number           = {sz\'am},
		chapter          = {fejezet},
		bathesis         = {szakdolgozat},
		mathesis         = {diplomaterv},
		phdthesis        = {disszert\'aci\'o},
		candthesis       = {disszert\'aci\'o},
		resreport        = {kutat\'asi jelent\'es},
		techreport       = {technikai jelent\'es},
		software         = {szoftver},
		datacd           = {CD-ROM},
		audiocd          = {audio CD},
		version          = {verzi\'o},
		url              = {c\'im},
		urlfrom          = {el\'erhet\H{o}},
		urlseen          = {el\'er\'es d\'atuma},
		inpreparation    = {el\H{o}k\'esz\'it\'es alatt},
		submitted        = {bek\"uld\"ott},
		forthcoming      = {elfogadott},
		inpress          = {nyomd\'aban},
		prepublished     = {preprint},
		citedas          = {tov\'abbiakban},
		thiscite         = {itt},
		seenote          = {l\'asd},
		quotedin         = {id\'ezte},
		idem             = {ugyan\H{o}},
		idemsf           = {ugyan\H{o}},
		idemsm           = {ugyan\H{o}},
		idemsn           = {ugyan\H{o}},
		idempf           = {ugyan\H{o}k},
		idempm           = {ugyan\H{o}k},
		idempn           = {ugyan\H{o}k},
		idempp           = {ugyan\H{o}k},
		ibidem           = {ugyanitt},
		opcit            = {id\'ezett m\H{u}},
		loccit           = {id\'ezett hely},
		confer           = {v\"o\adddot},
		sequens          = {sk\adddot},
		sequentes        = {skk\adddot},
		passim           = {passim},
		see              = {l\'asd},
		seealso          = {l\'asd m\'eg},
		backrefpage      = {hivatkoz\'asi oldal},
		backrefpages     = {hivatkoz\'asi oldalak},
		january          = {janu\'ar},
		february         = {febru\'ar},
		march            = {m\'arcius},
		april            = {\'aprilis},
		may              = {m\'ajus},
		june             = {j\'unius},
		july             = {j\'ulius},
		august           = {augusztus},
		september        = {szeptember},
		october          = {okt\'ober},
		november         = {november},
		december         = {december},
		langamerican     = {angol},
		langbrazilian    = {brazil},
		langbulgarian    = {bolg\'ar},
		langcatalan      = {katal\'an},
		langcroatian     = {horv\'at},
		langczech        = {cseh},
		langdanish       = {d\'an},
		langdutch        = {holland},
		langenglish      = {angol},
		langestonian     = {\'eszt},
		langfinnish      = {finn},
		langfrench       = {francia},
		langgalician     = {gal\'iciai},
		langgerman       = {n\'emet},
		langgreek        = {g\"or\"og},
		langitalian      = {olasz},
		langjapanese     = {jap\'an},
		langlatin        = {latin},
		langlatvian      = {lett},
		langnorwegian    = {norv\'eg},
		langpolish       = {lengyel},
		langportuguese   = {portug\'al},
		langrussian      = {orosz},
		langslovak       = {szlov\'ak},
		langslovene      = {szlov\'en},
		langspanish      = {spanyol},
		langswedish      = {sv\'ed},
		langukrainian    = {ukr\'an},
		fromamerican     = {angolb\'ol},
		frombrazilian    = {brazilb\'ol},
		frombulgarian    = {brazilb\'ol},
		fromcatalan      = {katal\'anb\'ol},
		fromcroatian     = {horv\'atb\'ol},
		fromczech        = {csehb\H{o}l},
		fromdanish       = {d\'anb\'ol},
		fromdutch        = {hollandb\'ol},
		fromenglish      = {angolb\'ol},
		fromestonian     = {\'esztb\H{o}l},
		fromfinnish      = {finnb\H{o}l},
		fromfrench       = {franci\'ab\'ol},
		fromgalician     = {gal\'iciaib\'ol},
		fromgerman       = {n\'emetb\H{o}l},
		fromgreek        = {g\"or\"ogb\H{o}l},
		fromitalian      = {olaszb\'ol},
		fromjapanese     = {jap\'anb\'ol},
		fromlatin        = {latinb\'ol},
		fromlatvian      = {lettb\H{o}l},
		fromnorwegian    = {norv\'egb\'ol},
		frompolish       = {lengyelb\H{o}l},
		fromportuguese   = {portug\'alb\'ol},
		fromrussian      = {oroszb\'ol},
		fromslovak       = {szlov\'akb\'ol},
		fromslovene      = {szlov\'enb\'ol},
		fromspanish      = {spanyolb\'ol},
		fromswedish      = {sv\'edb\H{o}l},
		fromukrainian    = {ukr\'anb\'ol},
		countryde        = {N\'emetorsz\'ag},
		countryep        = {Eur\'opai Uni\'o},
		countryeu        = {Eur\'opai Uni\'o},
		countryfr        = {Franciaorsz\'ag},
		countryuk        = {Egyes\"ult Kir\'alys\'ag},
		countryus        = {Amerikai Egyes\"ult \'Allamok},
		patent           = {szabadalom},
		patentde         = {n\'emet szabadalom},
		patenteu         = {eur\'opai szabadalom},
		patentfr         = {francia szabadalom},
		patentuk         = {brit szabadalom},
		patentus         = {amerikai szabadalom},
		patreq           = {szabadalmi k\'erelem},
		patreqde         = {n\'emet szabadalmi k\'erelem},
		patreqeu         = {eur\'opai szabadalmi k\'erelem},
		patreqfr         = {francia szabadalmi k\'erelem},
		patrequk         = {brit szabadalmi k\'erelem},
		patrequs         = {amerikai szabadalmi k\'erelem},
		file             = {f\'ajl},
		library          = {k\"onyvt\'ar},
		abstract         = {kivonat},
		annotation       = {jegyzet},
		commonera        = {id\H{o}sz\'am\'it\'asunk szerint},
		beforecommonera  = {id\H{o}sz\'am\'it\'asunk el\H{o}tt},
		annodomini       = {Krisztus ut\'an},
		beforechrist     = {Krisztus el\H{o}tt},
		circa            = {k\"or\"ulbel\"ul},
		spring           = {tavasz},
		summer           = {ny\'ar},
		autumn           = {\H{o}sz},
		winter           = {t\'el},
		am               = {d\'elel\H{o}tt},
		pm               = {d\'elut\'an},
}
\def\Title{Fejlődés-lélektani műhelymunka}
\def\Author{Bagaméri Fanni Gabriella}
\def\AuthortTitle{Pszichológia BA}
\def\SupervisorName{Dr. Oláh Katalin}
\def\SubTitle{A nyelv szerepe halláskárosult gyermekek tudatelméletének fejlődésében}
\def\SupervisorTitle{PhD Adjunktus}


\partfont{\fontsize{12}{15}\selectfont}

\renewcommand{\contentsname}{Tartalom}
 \addbibresource{references.bib} %opening 

 
\begin{document}
\begin{titlepage}
\begin{center}
	EÖTVÖS LORÁND TUDOMÁNYEGYETEM \\
	PEDAGÓGIAI ÉS PSZICHOLÓGIAI KAR \\
	PSZICHOLÓGIA SZAK 
\end{center}


\begin{center}
	\includegraphics[scale=0.3]{images/elte_logo}
\end{center}

 \vfill

\begin{center}
\Huge
\textbf{VISELKEDÉSELEMZŐI PORTFÓLIÓ}
\normalsize
\end{center}

\vfill

\begin{minipage}[t]{0.6\linewidth}
\begin{flushleft}
	\vspace{2cm}
\textbf{Bagaméri Fanni Gabriella} \\
Pszichológia BA
\end{flushleft}
\end{minipage}
\begin{minipage}[t]{0.45\linewidth}
\begin{center}
	Témavezetők:\\
\textbf{Szekeres Hanna} \\
\SupervisorTitle \\
\textbf{Dr. Honbolygó Ferenc} \\
PhD Adjunktus \\
\textbf{Dr. Oláh Katalin} \\
PhD Adjunktus \\
\textbf{Dr. Bányai-Nagy Henriett} \\
PhD Adjunktus \\
\end{center}
\end{minipage}

\vfill

\begin{center}
Budapest, 2019
\end{center}

\end{titlepage}


\includepdf[pages=-,offset=0 -25]{pages/eredetisegnyilatkozat_szakdolgozat}

\pagebreak
\setcounter{tocdepth}{2}
\tableofcontents
\listoftables
\pagebreak
\justify
\newrefsection
\part{Szociálpszichológia műhelymunka}
\begin{large}
	\Large A baráti kapcsolatok hatása a  homoszexualitással kapcsolatos attitűdökre és viselkedésre
\end{large}
\section{Előszó}
\section{Bevezetés}
\begin{table}[]
	\centering
	\begin{tabular}{|l|l|l|l|}
		\hline
		& \textit{N}  & \textit{M}    & \textit{SD}   \\ \hline
		Homofóbia skála  & 50 & 2.42 & 0.96 \\ \hline
		Rokonszenv skála & 50 & 63   & 29   \\ \hline
		Barátok száma    & 50 & 2.38 & 0.87 \\ \hline
	\end{tabular}
	\caption{Leíró statisztika}
	\label{table:1}
\end{table}
\pagebreak
\section{Hipotézisek}
\section{Módszer}
\subsection{Résztvevők}
\subsection{Eljárás}
\subsubsection{Adatfelvétel}
\subsubsection{Megtévesztés}
\subsubsection{Kérdőív}
\subsubsection{Statisztikai módszerek}
\subsection{Mérőeszközök}
\subsubsection{Homofóbia skála}
\subsubsection{Rokonszenv skála}
\subsubsection{Barátok száma}
\section{Eredmények}
\section{Diszkusszió}
\subsection{Barátok száma és Homofóbia}
\subsection{Barátok száma és Rokonszenv}
\subsection{Barátok száma és Konfrontáció}
\subsection{Barátok száma és a Bizalom Játék}
\subsection{A kutatás erősségei, limitációi és kitekintés}

\section{Összegzés}


\printbibliography[title={Hivatkoz\'asok}]
\pagebreak
\part{Oktatói értékelés}
\begin{large}
	\large Vélemény Bagaméri Fanni \textit{„A baráti kapcsolatok hatása a homoszexualitással kapcsolatos attitűdökre és viselkedésre”} című műhelymunkájáról
\end{large}
\\
\par A műhelymunka egy fontos társadalmi jelenséget dolgoz fel, amiben a műhelymunkázó a
melegekkel való baráti kapcsolatok összefüggését vizsgálja a heteroszexuális személyek
LMBTQ csoporttal szembeni attitűdjeivel, illetve, ezen baráti kapcsolatok összefüggését konkrét
viselkedéssel. Ebben a kontextusban, érdekes és fontos összefüggést talált arra vonatkozóan,
hogy a szakirodalmat alátámasztva, a baráti kapcsolatok pozitív kapcsolatban állnak a
melegekkel szembeni attitűdökre, azonban szemben azzal amire a szakirodalom következtet,
nincs kapcsolatban a tényleges meleg-támogató viselkedéssel. Ez utóbbi eredmény
megkérdőjelezi a szakirodalmat, ahol eddig csak vignette szituációkban és csak (cselekvéses)
intenciókat mértek. Ebből kifolyólag a jelen kutatás teljes mértékben újat tud nyújtani a jelen
témában.
\\
\par A kutatás kapcsolódik a Szociálpszichológia tanszéken folytatott előző kutatásokhoz, de önálló
kérdést vizsgál meg.
Bevezető, elméleti háttér, hipotézisek megalapozása: A műhelymunkázó a kutatási kérdést
megfelelően helyezi mind társadalmi mind empirikus kontextusba. A szakirodalom
áttekintése átfogó és alapos. Formailag tekintve néhol ugrált a szakirodalom bemutatása
között, így nem tökéletesen valósul meg a tölcséres szerkezet. A bevezető (végé)ből
kimaradt a jelen kutatás rövid ismertetése, ami így a módszertani rész végéig nem áll össze
az olvasónak. Ezentúl, összességében érthető és jól megfogalmazott bevezetőt írt,
megfelelően megalapozza hipotéziseit.
\\
\par Módszertan: Az alacsony elemszám miatt össze lett vonva a két kísérleti feltétel (enyhe és súlyos
inzultus), és a korrelációs teszteket ezen az összesített mintán végezte el, ami elfogadható,
azonban a kutatás így nem kísérleti. Tehát például a bizalmi játék így már nem kísérleti
manipulációt képez, hanem egy eszköz, ami a konfrontáció és bizalom mérését teszi
lehetővé. Ugyanakkor mivel a kutatás kísérletnek indult, ezért ez a fogalmi zavar a
műhelymunkában érthető és elfogadható. Ehhez kapcsolódóan, legalább lábjegyzékben
fontos lett volna megjegyezni, hogy mi volt kezdetben az elméleti megalapozás, vagy akár
exploratív érdeklődés oka a súlyos illetve enyhe feltételre. Ezentúl, maga a kutatás
módszertana kiemelkedően alapos, kreatív és megfontolt. A módszertan közlése is nagyon
átfogó és minden szükséges elemet tartalmaz.
\\
\par Statisztikai elemzés, eredmények: A címben szereplő baráti kapcsolatok „hatása” szó használata
félrevezető lehet, ugyanis a műhelymunkázó korrelációs kapcsolatot vizsgált. Ezentúl,
megfelelő statisztikai elemzést használt, és azt megfelelően közölte. (Apróbb megjegyzések,
hogy a szöveget több bekezdésbe kellett volna tördelni, és az első ábra tartalmát szövegben
is lehetett volna közölni, a második-harmadik táblázat meg nem egészen APA-stílusnak
megfelelő, de valószínű, hogy a táblázat a margó meghaladása miatt lett ilyen
formátumban).
\\
\par Diszkusszió: A kutatási eredmények diszkussziója, ugyan túlságosan tagolt, de ennek ellenére
teljesen megfelelő, jól közli a jelen kutatás újdonságát, közben a dolgozat eredményeit jól
elhelyezi a korábbi eredmények között.

\vspace{3mm}

\begin{minipage}[t]{0.5\linewidth}
	\begin{flushleft}
		Érdemjegy: 5 (jeles) \\
		Szekeres Hanna \\
		Budapest, 2019. május 27.
	\end{flushleft}
\end{minipage}


\changelocaltocdepth{1}
\newrefsection
\pagebreak
\part{Általános pszichológia műhelymunka}
\begin{large}
	\Large A zeneterápia hatása autista gyerekek szociális képességeire
\end{large}
\section*{Bevezetés}

Az Egészségügyi Világszervezet szerint világ\-szerte 160 gyermekből 1 megtalálható az autizmus spektrumon, s bár sokukból független, önellátó felnőtt válik, mások egy életen át tartó támogatásra szorulnak (\cite{WHO}). Az autizmus spektrum zavar vagy ASD olyan fejlődési zavar, melyben a személy szociális, kommunikációs és beszédkészsége sérült, érdeklődési köre szűk, viselkedése ritualizált, erősen ragaszkodik az állandósághoz, rutinjaihoz. Kétségtelen, hogy az autizmus spektrum zavar szignifikánsan korlátozhatja az autista személyt mindennapi feladatainak elvégzésé\-ben, azonban megfelelő pszichoszociális intervencióval fejleszthetőek az ASD-s gyermekek kommunikációs, illetve szociális viselkedéses képességei (\textcite{boso_emanuele_minazzi_abbamonte_politi_2007}; \textcite{finnigan_starr_2010}; \textcite{kern_wolery_aldridge_2006}; \textcite{kim_wigram_gold_2009}). Ezen intervenciók egyik, egyre elterjedtebb formája a \textit{zeneterápia.} Ennek a tanulmánynak a célja, hogy feltérképezze a zeneterápia autista gyerekekre gyakorolt hatásaival foglalkozó, szerteágazó vizsgá\-latokat és átfogó képet adjon a témával kapcsolatos jelenlegi, tudományos álláspontról.
\pagebreak
\section*{Diszkusszió}

Bár a zenét, illetve hangszereket egyre elterjedtebben alkalmazzák ASD-s személyek terápiájában, a zeneterápiának számos különböző fajtája van. Kern, Wolery és Aldrigde (2007) például egyénre szabott, zeneterapeuták által költött, zenés mondóká\-kat tanítottak a vizsgálatukban résztvevő gyermekek tanítóinak. A mondókák mind a reggeli rutinhoz kapcsolódtak, s céljük az otthon-óvóda közötti átmenet könnyebbé tétele, valamint a független funkcionálás és a csoporttársakkal való interakció elősegí\-tése volt. A mondó\-káknak mindegyik esetben \textit{5 lépése} volt:

\begin{enumerate}
	\item A gyermek önállóan belép a terembe
	
	\item A gyermek üdvözöl valakit a teremben (tanítót vagy társat) verbálisan vagy non-verbálisan
	
	\item A gyermek üdvözöl valaki mást a teremben (tanítót vagy társat) verbálisan vagy non-verbálisan
	
	\item A gyermek elköszön a gondozójától (szóban vagy integet), aki ezután elhagyja a termet
	
	\item A gyermek játszani kezd egy általa választott játékkal
\end{enumerate}

A mondókák mindkét esetben hatékonynak bizonyultak, azaz a személyre szabott ének hatására mindkét kisfiú könnyebben tudott alkalmazkodni az új környezethez és több lépést végeztek el tanítói közbeavatkozás nélkül, mint korábban (\cite{kern_wolery_aldridge_2006}). Érdekes eredmény az is, hogy nem csak a kisfiúk alkalmazkodását segítette elő a zeneterápia, hanem a csoporttársaik is szívesebben üdvözölték őket, miután közösen elénekelték a rutint (\cite{kern_wolery_aldridge_2006}) Ezen eredmények nem csak bizonyítják, hogy a zeneterápia egy rendkívül hasznos intervenciós stratégia, de gyakorlati megoldást is kínálnak a spektrumon lévő gyermekek szociális és kommunikációs képességeinek fejlesztésére. Ennél azonban létezik egy sokkal interaktívabb megközelítés is. Kim, Wigram és Gold (2009) vizsgálatukban egyéni improvizációs zeneterápiát tartott gyermekeknek, ahol többek között énekelhettek, dobolhattak, zongorázhattak. 
\\ \\ \\ 
A vizsgálatban \textit{4 feltételt} különböztettek meg:
\begin{enumerate}
	\item A terapeuta által irányított közös zenélés
	\item A gyermek által irányított közös zenélés
	\item A terapeuta által irányított játék (nem zenés)
	\item A gyermek által irányított játék (nem zenés)
\end{enumerate}    
A szerzők a gyermekek \textit{érzelmi, motivációs és interperszonális válaszkészségét } mérték fel, a különböző helyzetekben:

\begin{enumerate}
	\item Mérték a gyermekek örömét, a különböző helyzetekben, vagyis a mosolygásuk, nevetésük gyakoriságát és időtartamát.
	\item Mérték a gyermek-terapeuta közti érzelmi szinkronizációt (amikor a közös tevékeny\-ség során mindketten szomorúságot vagy örömöt élnek át).
	\item Ezek mellett mérték, hogy a gyermekek kezdeményeznek, engedelmeskednek, vagy figyelmen kívül hagyják a terapeuta interakciós kezdeményezését a különböző felté\-telekben.
\end{enumerate} 

Az eredmények szerint a gyermekek gyakrabban éltek át örömöt, s ez az öröm tovább is tartott a zeneterápiában, mint a sima játékban (\cite{kim_wigram_gold_2009}). Továbbá a gyermekek több örömöt éltek át az általuk irányított zeneterápiás foglalkozások keretében, mint a terapeuta által irányítottakban (\cite{kim_wigram_gold_2009}). Az érzelmi szinkronizáció, illetve a kezdeményezés is szignifikánsan többször fordult elő a zeneterápiában - azon belül is a gyermek által irányított feltételben - mint a játékban (\cite{kim_wigram_gold_2009}). Különösen izgalmas eredmény, hogy a gyermekek kétszer olyan valószínűséggel hagyták figyelmen kívül a terapeuta interakciós kezdeményezéseit a sima játék során, mint a zeneterápiában (\cite{kim_wigram_gold_2009}). Ezeknek az eredmények fontos klinikai implikációi vannak. Először is, megállapíthatjuk, hogy a gyermekek \textit{akkor élték át a legtöbb örömöt, amikor ők irányíthatták a foglalkozást}, mindkét feltételben. Másodszor, mivel a közös zenélés kétségtelenül érzelmi bevonódást igényel, a zeneterápia ily módon segíthet fejleszteni az autizmus spektrum zavaros gyermekek adekvát érzelmi reakcióit.

Ezeket az eredményeket támasztja alá egy, az előbbihez hasonló kutatás is. Boso és munkatársai (2007) 52 héten át tartó vizsgálatukban heti rendszerességgel tartottak csoportos zeneterápiát 8 spektrumon lévő fiatal felnőttnek. A résztvevőket 3 alkalommal értékelte saját pszichiáterük a vizsgálat során a CGI-I és BPRS skálán. A majdnem egy éven tartó vizsgálat végére a résztvevők CGI-I és BPRS pontszámai nagyban fejlődtek (tehát kevesebb és kevésbé extrém tüneteket mutattak), illetve különösen izgalmas eredmény, hogy a résztvevők zenés képességei szignifikánsan javultak a kezdeti kompetenciájukhoz képest (\cite{boso_emanuele_minazzi_abbamonte_politi_2007}). Finnigan és Starr (2010) vizsgálatukban a zeneterápia egy újabb formáját alkalmazták. Tanulmányukban \textit{2feltételt} különböztettek meg:

\begin{enumerate}
	\item Játék zenés aláfestéssel a terapeutától
	\item Játék zene nélkül
\end{enumerate}  

Finnigan és Starr arra volt kíváncsi, hogy vajon a zeneterápia növeli-e az ASD-s gyermek szociális válaszkészségét (a szemkontaktus gyakoriságát, a terapeuta imitálá\-sát és a sorban  következés kivárását, megértését). A gyermek tehát az előző vizsgála\-toktól eltérően, itt mindkét esetben játékokkal játszott (pl: labda, autó, dob), s a szerzők azt figyelték meg, hogy e játék során hogyan viselkedik a különböző feltételektől függően. Eredményeik szinte teljes összhangot mutatnak az ezen tanulmányban korábban bemutatott vizsgálatokéval. A zeneterápiás feltételben a gyermek szignifikánsan többször létesí\-tett szemkontaktust, imitálta a terapeutát, és követte a sorban következés szabályait, mint a terápia kezdete előtt, illetve mint a zene nélküli feltételben (\cite{finnigan_starr_2010}). A vizsgálatból levonható legfontosabb következtetés, hogy a zeneterápia növelheti a szociális válaszkészséget az autizmus spektrum zavaros személyekben, illetve kiváló motivációs eszköznek is bizonyult (\cite{finnigan_starr_2010}).
\pagebreak

\section*{Összegzés}

Összességében elmondható, hogy a tanulmány keretein belül feldolgozott vizsgálatok szignifikánsan egybevágnak. Mindegyik cikkben, ahol a zeneterápiát hasonlították össze a szerzők valamilyen zene nélküli játékos terápiával, a zeneterápia szignifikánsan jobb eredményeket hozott (\cite{finnigan_starr_2010}; \cite{kim_wigram_gold_2009}). Ez azt jelenti, hogy a zeneterápia nem csak az olyan szociális képességeket  tudja fejleszteni, mint a szemkontaktus fenntartása (\cite{finnigan_starr_2010}), vagy a kezdeményezés (\cite{kim_wigram_gold_2009}), de a független funkcionálást is elősegíti (\cite{kern_wolery_aldridge_2006}). Ezek mellett Kim, Wigram és Gold (2009) cikkéből tisztán látszik, hogy az autizmus spektrum zavaros gyermekek egyszerűen \textit{jobban élvezték} a zeneterápiát, mint a sima játékos foglalkozásokat, ami egy kiemelkedően fontos szempont a megfelelő intervenció kiválasztásánál. Boso és munkatársai (2007) vizsgálata pedig arra is felhívja az olvasó figyelmét, hogy a zeneterápia nemcsak az ASD-s gyermekek szociális képességeit fejleszti, hanem a \textit{zenei kompetenciájukat} is. Mindezek ellenére fontos megjegyezni, hogy az ezen tanulmány keretein belül felsorolt vizsgálatok mindegyike aggasztóan kevés vizsgálati személlyel dolgozott, s emiatt nem lehetünk biztosak benne, hogy eredményeik általánosíthatóak nagyobb populációkra. A jövőben a zeneterápiát s annak hatásait vizsgáló tanulmányoknak egyik központi célja lehet ezen módszertani hiba kiküszöbölése.
\printbibliography[title={Hivatkoz\'asok}]
\newrefsection
\pagebreak
\part{Fejlődés-lélektan műhelymunka}
\begin{large}
	\Large A nyelv szerepe halláskárosult gyermekek tudatelméletének fejlődésében
\end{large}
\section*{Bevezetés}
A tudatelmélet, vagyis az a képesség, hogy másoknak sajátunktól eltérő mentális állapoto\-kat,
illetve szándékokat tulajdonítunk, a gyermekkori kognitív fejlődés fontos állomása. A
tudatelmélet lehetséges mérőeszközei a téves-vélekedés tesztek, melyek azt vizsgálják, hogy a
gyermekek képesek-e felismerni, hogy az emberek információ hiányában hamis vélekedést
alakíthatnak ki egy szituációról. Egy másik lehetséges mérőeszköz a látszat-valóság teszt,
aminek szintén számos fajtája létezik és elterjedten alkalmazzák a tudatelmélet vizsgálatára.
Ezen tesztek segítségével a kutatók kialakítottak egy konszenzust, mely szerint a tipikusan
fejlődő gyermekek körülbelül 4 éves korukra képesek hamis vélekedést tulajdonítani
másoknak, tehát 4 éves korban a gyermekek szignifikánsan jobban teljesítenek mind a hamis-
vélekedés, mind a látszat-valóság teszteken, mint fiatalabb társaik \autocite{wimmer_1983,perner_leekam_wimmer_1987}. \\
\\
Azonban újabb vizsgálatok kimutatták, hogy egyéni szinten megfigyelhető egy viszonylag széles variancia az életkorban, amikor a gyermekek képesek jól megoldani a hamis-vélekedés tesztet \autocite{jenkins_astington_1996}. \textcite{jenkins_astington_1996} kutatásukban azt találták, hogy számos gyermek már 3 éves korban is
átment a teszten, míg másoknak 5 éves korukra sikerült csak megoldaniuk azt. Mi okozhatja
tehát ezt a drámai egyéni különbséget a gyermekek tudatelméletének fejlődésében? Az egyik
válasz a \textit{nyelv}. Számos vizsgálat foglalkozik azzal, hogy hogyan befolyásolja a gyermekek
nyelvi képessége tudatelméletük fejlődését. \textcite{milligan_astington_dack_2007} például
kimutatta, hogy korreláció figyelhető meg a gyermekek nyelvi képességei és a téves-
vélekedés megértése között, életkortól függetlenül. Ezek alapján különösen izgalmas kérdés,
hogy milyen szintű tudatelmélettel rendelkeznek azok a siket gyermekek, akiknek nyelvi és
kommunikációs fejlődése eltér a tipikustól. Ennek a tanulmánynak a célja, hogy feltérképezze
a halláskáro\-sult gyermekek tudatelméletével foglalkozó szerteágazó vizsgálatokat, és átfogó
képet adjon a jelenlegi, tudományos álláspontról.

\pagebreak

\section*{Diszkusszió}

A legtöbb, halláskárosult gyermekek tudatelméletével foglalkozó tanulmány, megkülönböztet
három csoportot a siket gyermekek közt, az alapján, hogy mennyire férnek hozzá a
jelnyelven, vagy beszédben folytatott kommunikációhoz családjukon belül:
\begin{enumerate}
	\item Azon gyermekek csoportja, akik anyanyelve a jelnyelv és jelnyelven képesek
	beszél\-getést folytatni valamelyik családtagjukkal
	\item Orálisan képzett siket gyermekek, akik képesek szóban beszélgetést folytatni
	\item Azon siket gyermekek csoportja, akik csak később, iskolás éveik alatt sajátítják el a
	jelbeszédet, éveken át tartó kommunikációs hiány után
\end{enumerate}
Az ezen tanulmány keretein belül feldolgozott kutatások egyöntetűen azt az álláspontot
hangsúlyozzák, hogy azok a gyermekek, akik az első csoportba tartoznak, vagyis anyanyelvük
a jelnyelv, s azt már születésüktől fogva tanulták, szignifikánsan jobban teljesítenek a hamis-
vélekedés teszteken, mint azok a társaik, akik csak később sajátították el azt \autocite{woolfe_want_siegal_2002,peterson_siegal_1999,peterson_slaughter_2006,schick_villiers_villiers_hoffmeister_2007}.
\textcite{woolfe_want_siegal_2002} kutatásukban azt találták, hogy a jelnyelven anyanyelvi
szinten kommunikáló csoport még akkor is jobban teljesített a hamis-vélekedés teszten, ha a
kutatók olyan faktorokra is kontrolláltak, mint a végrehajtó funkciók, illetve a gyermekek
mondattani képességei.\\
\\
 \textcite{peterson_siegal_1999} vizsgálatukban egy egész sor téves-
vélekedés tesztet töltettek ki a résztvevő siket és autista gyermekekkel. Ezek közül az egyik például a híres Sally-Anne teszt \autocite{baron-cohen_leslie_frith_1985} volt, amit ma is széles körben alkalmaznak a tudatelmélet vizsgálatára. Az előző kutatáshoz
hasonlóan azt találták, hogy a jel anyanyelvű gyermekek szignifikánsan jobban teljesítettek
azoknál a társaiknál, akik később tanulták meg a jelnyelvet \autocite{peterson_siegal_1999}. Sőt, a jel anyanyelvű,
illetve orálisan képzett gyermekek ugyanolyan jól teljesítettek a teszteken, mint a halló
gyermekekből álló kontrollcsoport \autocite{peterson_siegal_1999}. Peterson és Siegal (1999) szerint ez abból a
különbségből adódik, hogy a siket szülők ugyanolyan könnyedséggel kommunikálnak siket
gyermekükkel távoli, hipotetikus, vagy elképzelt dolgokról, mint halló szülők a halló
gyermekükkel \autocite{peterson_siegal_1999}. Azonban közös nyelv hiányában a halló szülők egyáltalán nem, vagy
csak nagyon limitált formában képesek kommunikálni halláskárosult gyermekükkel minden
olyan dologról, ami nem az „itt és most”-hoz tartozik. \autocite{peterson_siegal_1999}. Ezek az adatok azért
kiemelkedően fontosak, mert alátámasztják azt az elképzelést, hogy a korai dialógusnak
központi szerepe van a gyermekek tudatelméletének fejlődésében. \\
\\
Ezt az elképzelést támasztja alá \textcite{peterson_slaughter_2006} vizsgálata is, ahol nem csak siket gyermekek
tudatelméletét vizsgálták, hanem spontán narratív beszédüket is olyan kategóriákban, mint \textit{a
	valóság-orientált kogníció} (melléknevek, főnevek, igék pl.: okos, tudja, gondolja), \textit{képzelet-
	orientált kogníció } (melléknevek, főnevek, igék pl.: színlel, álmodik), \textit{percepció} (melléknevek,
főnevek, igék, amik az információszerzésre utalnak, valamilyen érzékszerven keresztül pl.:
lát, hall), illetve \textit{a vágyak és érzelmek}. A vizsgálatban 21 siket (mindegyikük csak iskolás kortól
tanult jelnyelvet) és 13 halló gyermek vett részt. Az előző tanulmányokhoz hasonlóan, a
jelnyelvet későn elsajátító siket gyermekek itt is szignifikánsan rosszabbul teljesítettek a
hamis-vélekedés teszten, mint halló társaik \autocite{peterson_slaughter_2006}. Azonban különösen figyelemre méltó
eredmény, hogy azok a halláskárosult gyerekek, akik a narratív vizsgálatban többször
beszéltek a mesében szereplő karakterek kognícióiról (és főként a képzelet-orientált
kogní\-ciókról, mint pl.: a színlelés) nagyobb valószínűséggel oldották meg a téves-vélekedés
tesztet is \autocite{peterson_slaughter_2006}.\\
\\
Egy másik megközelítésben a kutatók arra voltak kíváncsiak, hogy vajon a halláskárosult
gyermekek tudatelméletének fejlődési hátránya akkor is jelentkezik-e, ha a hamis-vélekedés
tesztben a nyelvi követelményeket minimalizálják, s a tesztnek egy non-verbális válto\-zatát
prezentálják a gyermekeknek \autocite{figueras-costa_harris_2001}. \textcite{figueras-costa_harris_2001}
vizsgálatukban 21 fő orálisan képzett, hallókészüléket viselő halláskárosult kisgyermekkel
dolgozott, s céljuk az volt, hogy megállapítsák: a gyermekeknek valóban fejlődési hátránya
van a tudatelmélet elsajátításában vagy ez csak egy látszólagos lemaradás, amit a feladat
megértési nehézsége okoz? Ennek a kérdésnek a megválaszolására a kutatók mind verbális
mind non-verbális hamis-vélekedés tesztnek is alávetették a résztvevőket, s azt találták, hogy
a siket gyermekek a non-verbális teszten minden esetben jobban teljesítettek, életkortól
függetlenül \autocite{figueras-costa_harris_2001}. Azonban a szerzők figyelmeztetnek arra, hogy ez az eredmény nem azt
jelenti, hogy a halláskárosult gyermekek tudatelméletének fejlődésében megfigyelt hátrány
csupán a verbálisan prezentált feladat megértésének nehézségét tükrözi. Bár az kétségtelen,
hogy a non-verbális feladat facilitálta a gyermekek teljesítményét, a gyermekek még így is
rosszabbul teljesítettek, mint az elvárt lett volna az életkoruk alapján \autocite{figueras-costa_harris_2001}. A non-verbális
téves-vélekedés feladat megoldásának életkori átlaga 8 év 10 hónap volt ebben a vizsgálatban,
ami körülbelül 4 évnyi fejlődési hátrányt sugall a tipikus fejlődésű gyermekekhez képest
\autocite{figueras-costa_harris_2001}.

\pagebreak

\section*{Összegzés}

Az ezen tanulmány keretein belül áttekintett releváns irodalmak alapvetően két fontos
pontban találkoznak. Először is mindegyik vizsgálat azt az eredményt hozta, hogy a jelnyelvet
csak később, iskolás korban elsajátító gyermekek rosszabbul teljesítenek a hamis-vélekedés
teszteken, nem csak halló, de jel anyanyelvű, illetve orálisan képzett társaiknál is \autocite{peterson_slaughter_2006,woolfe_want_siegal_2002,peterson_siegal_1999}.
Egyesek szerint ezt a különbséget az a kommunikációs depriválás okozza, ami a halló
családok körébe született halláskárosult gyermekek életének első éveit jellemzi \autocite{perner_leekam_wimmer_1987}. Mivel a
halló szülőknek sok esetben nincs kielégítő kommunikációs csatornája a siket gyermekkel, a
szülő-gyermek kommunikáció így csupán a jelenre, s a fizikai környezetben könnyen
referálható dolgokra korlátozódik \autocite{peterson_siegal_2000}. Mivel a mentális állapotok elvont, nehezen
referálható entitások, a halló szülők szinte egyáltalán nem, vagy nagyon ritkán képesek
megosztani azokat siket utódaikkal, ez pedig hozzájárulhat a tudatelméletük hátrányos
fejlődéséhez \autocite{peterson_siegal_2000}. Másodszor, a vizsgálatok mindegyike egyetért abban, hogy bár a
jel anyanyelvű családba született gyermekek szinte ugyanúgy teljesítenek a hamis-vélekedés
feladatokban, mint halló társaik, akik csak később tanulnak meg jelnyelven kommunikálni,
szignifikánsan rosszabbul teljesítenek, mint az azonos életkorú, tipikus fejlődésű társaik \autocite{peterson_slaughter_2006,woolfe_want_siegal_2002,peterson_siegal_1999}.
Ez a jelenség akkor is fennállt, amikor a nyelvi követelményeket minimalizálták a kutatók
\autocite{figueras-costa_harris_2001}. Ezek az eredmények különösen fontosak a halláskárosult gyermekek integrációja
szempontjából, hiszen más emberek érzéseinek, vágyainak feltételezése és megértése alapvető
követelmény a legtöbb társas helyzetben. Az eredmények azonban arra is rávilágítanak, hogy
a korai kommunikációnak elengedhetetlenül fontos szerepe van a gyermekek
tudatelméletének fejlődésében, annak modalitásától függetlenül \autocite{peterson_2004}.
\printbibliography[title={Hivatkoz\'asok}]
\newrefsection
\pagebreak
\part{Személyiségpszichológia műhelymunka}
\begin{large}
	\Large Boldogság-averzió, félelem a boldogságtól
\end{large}
\section{Bevezetés}
A 21. századi fejlett, nyugati társadalmak köztudatában egy túlnyomóan pozitív attitűd figyelhető meg a boldogsággal kapcsolatban. Korábbi kutatások szerint, a boldogság nem csak a szubjektív jóllétünkhöz járul hozzá, de befolyásolja a fizikai és pszichológiai egészségünket is, kihat társas kapcsolatainkra, mindezek mellett pedig az egyik legfontosabb motiváló erő számos ember életében. Sőt, egyes kutatók szerint a szubjektív jóllét, tulajdonképpen maga a pozitív hatások jelenléte, a negatív hatások hiánya és az élettel való elégedettség.\cite{diener_suh_lucas_smith_1999} Arisztotelész, a nyugati filozófia atyja, ismert idézete szerint: \textit{“Happiness is the meaning and the purpose of life, the whole aim and end of human existence.”} \medskip 
\\ Azonban egyre bővül azoknak az empirikus kutatásoknak a száma, melyek megkérdő\-jelezik  a boldogság egyöntetűen pozitív megítélését. Léteznek ugyanis olyan populációk, ahol megfigyelhető egy sokkal ambivalensebb attitűd a boldogság felé. Például, kollektivista társadalmakon végzett vizsgálatok megállapították, hogy számos kultúrá\-ban kifejezetten tartanak a túlzott boldogságtól, nemkívánatosnak tartják azt \cite{joshanloo_weijers_2013} \cite{joshanloo_lepshokova_panyusheva_natalia_poon_yeung_sundaram_achoui_asano_igarashi}, de a nyugati társadalmakban is ugyanúgy megfigyelhető ez a jelenség \cite{gilbert_mcewan_catarino_baiao_palmeira_2013}.  Ennek a tanulmánynak a célja, hogy megvizsgálja a boldogság-averzióval, illetve a boldogságtól való félelemmel kapcsolatos szerteágazó kutatásokat és átfogó képet adjon a témával kapcsolatos jelenlegi, tudományos álláspontról. \medskip 

\pagebreak
\section {Diszkusszió}
\subsection{Aversion to Happiness Across Cultures: A Review
	of Where and Why People are Averse to Happiness \cite{joshanloo_weijers_2013}}
Joshanloo és Weijers (2013) cikke volt az egyik első kultúrközi tudományos munka, ami a boldogság-averzió koncepciójával foglalkozott. Tanulmányuk célja az volt, hogy korábbi empirikus kutatásokon keresztül megvizsgálják, hogy más-más kultúrákban milyen eltérő indokok miatt alakul ki averzió az emberekben, a boldogság különféle változataival szemben. Hipotézisük, vagyis hogy létezik a boldogsággal szembeni averzió, beigazolódott, és bár a keleti kultúrákban kétségkívül jobban megfigyelhető ez a jelenség, a nyugati kultúrákban is jelen van. A keleti kultúrákon végzett "fear of happiness" vagyis "boldogságtól való félelem" skálával való mérés során a kutatók azt találták, hogy eltérő mértékben ugyan, de minden vizsgált kultúrára jellemzőek voltak úgynevezett \textit{hiedelmek} \cite{joshanloo_weijers_2013}}. \medskip
 \\ A tanulmány 4 hiedelmet (belief) emel ki a boldogság-averzió vonatkozó mutatójaként \cite{joshanloo_weijers_2013}: 
\begin{enumerate}
	\item A boldogság megnöveli az esélyét, hogy valami rossz dolog fog történni velünk
	\item A boldogság rosszabb emberré tesz minket
	\item A boldogság kimutatása rossz hatással van ránk és másokra
	\item A boldogságra való törekvés rossz hatással van ránk és másokra

\end{enumerate}
Konklúzióként megállapíthatjuk, hogy valószínűleg ezek a szinte babonás hiedelmek állhatnak a túlzott boldogságtól való félelem hátterében. A boldogság-averzió egy érdekes és sokrétű jelenség, mely minden kultúrában megtalálható eltérő mértékben: Míg a keleti kultúrákban a vallás (pl.: buddhizmus és a vágyakról való lemondás), az erős konformitás és a (pozitív) érzelmek moderált kimutatása mind hozzájárulhat a hiedelmek megszilárdulásához, addig az individualista társadalmak egyén-központúsága és a pozitív érzelmek kimutatásának hangsúlyossága korlátozhatja azt \cite{joshanloo_lepshokova_panyusheva_natalia_poon_yeung_sundaram_achoui_asano_igarashi}.

\subsection{Fears of compassion and happiness in relation
	to alexithymia, mindfulness, and self-criticism \cite{gilbert_mcewan_gibbons_chotai_duarte_matos_2011}}
Gilbert és munkatársai (2011) az elsők közt kezdtek foglalkozni a boldogságtól való félelemmel. Empirikus tanulmányuk célja, hogy egy új, a kutatók által kidolgozott "Fear of Happiness" vagyis "Boldogságtól Való Félelem" skálával \cite{gilbert_mcewan_gibbons_chotai_duarte_matos_2011} megvizsgálják az összefüggéseket az odaadás (compassion) és a boldogságtól való félelem (fear of happiness), valamint olyan érzelem-feldolgozó kompetenciák közt, mint az  alexithymia, mindfulness, és empátia, illetve mindezek összefüggését az önkritikával és pszichopatológiával. Ezt a skálát Gilbert terápiás munkassága hatására fejlesztették ki a kutatók és olyan tételek tartalmazott, mint: \textit{"Félek, hogyha jól érzem magam, valami rossz fog történni"} \cite[o. 381]{gilbert_mcewan_gibbons_chotai_duarte_matos_2011}. A vizsgálat hipotézise, hogy összefüggést fognak találni az egyén odaadástól való félelme és az érzelem feldolgozása közt, amit a statisztikai elemzés be is bizonyított. Korrelációt találtak mind az odaadástól mind a boldogságtól való félelem és a szorongás, erős önkritika, és a mindfulness-el és alexithymia-val kapcsolatos nehézségek közt. A boldogságkutatással kapcsolatos legfontosabb eredmény, hogy a kutatók kiemelkedően magas (r=.70) korrelációt találtak a boldogságtól való félelem és a depresszió közt, ami bizonyítja a boldogságkutatás klinikai relevanciáját. A vizsgálat legnagyobb limitációja, hogy a kísérleti személyek 83\%-a nő volt, ami kétségessé teszi a minta reprezentatívságát \cite{gilbert_mcewan_gibbons_chotai_duarte_matos_2011}.

\subsection{Fears of happiness and compassion in relationship
	with depression, alexithymia, and attachment
	security in a depressed sample \cite{gilbert_mcewan_catarino_baiao_palmeira_2013}}
Gilbert és munkatársainak (2011) eredménye, miszerint a boldogságtól való félelem magasan korrelál (r=0.7) a depresszióval, nagyban befolyásolta ezt a tudományos értekezést. Ennek a tanulmánynak a célja, hogy megvizsgálja a 2011-ben talált, főleg női egyetemistákon végzett kísérleti eredményeket depressziós mintán. A kutatók hipotézise az volt,  hogy \textsubscript{1} \textit{ A depressziós személyek nagyobb félelmet fognak mutatni a boldogság és odaadás (compassion) felé,}  mint a nagyrészt egyetemis\-tákból álló minta \textsubscript{2}A pozitív érzelmektől való félelem \textit{korrelálni fog a depresszióval, stresszel, szorongással, alexithymia-val }   \textsubscript{3} A pozitív érzelmektől való félelem, együtt fog járni \textit{a gyengébb minőségű felnőttkori kötődéssel} \cite{gilbert_mcewan_catarino_baiao_palmeira_2013}. Mindhárom hipotézis beigazolódott, de különösen érdemes megemlíteni, hogy a depressziós mintában magasabb a boldogságtól való félelem, mint a tanulókkal elvégzett kísérletben \cite{gilbert_mcewan_gibbons_chotai_duarte_matos_2011}, tehát a depressziós populáció jobban tart az extrém boldogságtól, mint az egyetemista. A boldogságtól való félelem magasan korrelált a felnőttkori bizonytalan kötődési stílusokkal, alexithymiaval, és a legpontosabban jósolta be a stresszt és szorongást \cite{gilbert_mcewan_catarino_baiao_palmeira_2013}. A vizsgálati eredményeknek gyakorlati relevanciája is van, hiszen segíthet mélyebben megérteni a depressziós populáció érzelmi kvalitását. Azonban fontos limitáció a vizsgálat introspektív mivolta, illetve a szociális kívánatosság befolyásoló ereje, amit nem lehet figyelmen kívül hagyni \cite{gilbert_mcewan_catarino_baiao_palmeira_2013}.

\subsection{Fears of Negative Emotions in Relation to Fears of Happiness, Compassion,
	Alexithymia and Psychopathology in a Depressed Population \cite{gilbert_2014}}
 Ennek a tanulmánynak a célja, hogy megvizsgálja a \textit{kapcsolatot:}\\
 \textsubscript{1} Három negatív érzelemtől való félelem (szorongás, harag, szomorúság) és ezen érzelmek elkerülése közt \\
 \textsubscript{2} A negatív és pozitív érzelmektől való félelem közt.\\
 \textsubscript{3} Mindezek kapcsolatát az odaadással, alexithymiaval és pszichopatológiával.\\
 A hipotézis, vagyis, hogy a félt érzelmeket hajlamosabbak elkerülni az emberek, illetve, hogy korrelációt fognak találni a negatív és pozitív érzelmektől való félelem közt, beigazolódott. A boldogságkutatás szempontjából lényeges kiemelni, hogy a \textit{pozitív érzelmektől való félelem szignifikánsan korrelált a szorongástól, a haragtól és a szomorúságtól való félelemmel} és ezen érzelmek elkerülésével. \cite{gilbert_2014}. Érdemes azt is megjegyezni, hogy míg a szorongástól való félelem és a szorongás elkerülése közötti korreláció meglehetősen alacsony volt, addig a szomorúságtól való félelem és a szomorúság elkerülése, illetve főként a haragtól való félelem és a harag elkerülése közötti korreláció kifejezetten magas\cite{gilbert_2014}. E tanulmány eredményei alapján elmondhatjuk, hogy egyes érzelmektől való félelem, illetve adott érzelem elkerülése nagyban függ attól, hogy \textit{pontosan melyik} érzelemről beszélünk. A kutatás legfőbb limitációja, hogy viszonylag kis létszámú mintán (52 fő) végezték, többségben női résztvevőkkel.
 
\subsection{Cross-Cultural Validation of
	Fear of Happiness Scale Across 14 National Groups \cite{joshanloo_lepshokova_panyusheva_natalia_poon_yeung_sundaram_achoui_asano_igarashi}}
Ez a széleskörű kultúrközi tanulmány 14 nemzeten vizsgálta a Joshanloo (2013) által kidolgozott "Fear of Happiness Scale-t (FHS)" azaz a "Boldogságtól Való Félelem Skálát". A kutatás célja részben e skála validitásának ellenőrzése volt, egy széleskörű mintán, és azon az elképzelésen alapul, hogy bizonyos kontextusokban az emberek nemkívánatosnak tartják a boldogságot, vagy egyenesen tartanak tőle \cite{joshanloo_lepshokova_panyusheva_natalia_poon_yeung_sundaram_achoui_asano_igarashi}. A vizsgálati hipotézisek, hogy \textsubscript{1} Egyének szintjén az FHS negatívan fog korrelálni az élettel való elégedettséggel \textsubscript{2} Kultúrális szinten az FHS negatívan fog korrelálni a szubjektív-jólléttel \textsubscript{3} Bizonyos vallásos csoportokhoz való tartozás pozitívan fog korrelálni az FHS-el \cite{joshanloo_lepshokova_panyusheva_natalia_poon_yeung_sundaram_achoui_asano_igarashi}. Bár a skála validitása megfelelőnek bizonyult és mindhárom hipotézis teljesült, az FHS és a szubjektív-jóllét negatív korrelációja igen gyenge (\textless.15) volt, míg az FHS és az élettel való elégedettségé viszonylag magas. A tanulmány szerint olyan vallások, mint a \textit{Buddhizmus, Hinduizmus, Taoizmus} pozitívan, míg a \textit{Kereszténység} negatívan korrelál a boldogság-averzióval, ez valószínűleg az egyes vallások tanításaival hozható összefüggésbe \cite{joshanloo_lepshokova_panyusheva_natalia_poon_yeung_sundaram_achoui_asano_igarashi}. A vizsgálat legfőbb limitációja, hogy egyes nemzetek meglehetősen alulreprezentáltak, ami torzíthatja az eredményeket, valamint, hogy a résztvevők túlnyomórészt fiatal felnőttek voltak, ami megkérdőjelezi az eredmények általánosíthatóságát felnőtt populációkra.

\subsection{Eastern Conceptualizations of Happiness: Fundamental
	Differences with Western Views \cite{joshanloo_2013_eastern}}
Joshanloo számos munkájában hangott adott kritikájának, miszerint a boldogságku\-tatás főként nyugati kultúrák boldogságkoncepcióján alapszik, s így nem feltétlenül általánosítható más kultúrákra \cite{joshanloo_2013}. Ebben a tudományos értekezésében éppen erre a feladatra vállalkozik, elkülöníti a keleti és nyugati kultúrák boldogságkoncepcióit és kiemeli a legalapvetőbb kultúrális különbségeket: \\
\textsubscript{1} \textit{Hedonizmus} és \textit{Eudaimónia}\\
E két különböző boldogságkoncepció már az ókori görög filozófiában is heves viták alanya volt. Míg az eudaimónia tana szerint a boldogság tulajdonképpen az erények gyakorlásának következménye, a hedonizmus a testi örömöket a lelki felé helyezi, sze\-rinte a boldogság a gyönyörrel, élvezettel teli élet. Joshanloo a keleti boldogságkoncep\-ciót az antik eudaimóniával, a nyugatit pedig a hedonizmussal azonosította. \\
\textsubscript{2}\textit{Self-Transzcendencia} és \textit{Self-Kiemelés}\\
Míg a nyugati kultúrák általában az individualista értékek alapján határozzák meg a selfet, addig a keleti kultúrák hajlamosak a nagy egész, a kollektív, a kozmosz kis részeként azonosítani azt \cite{joshanloo_2013_eastern}.\\
\textsubscript{3}\textit{Harmónia} és \textit{Uralás}\\
A nyugati világnézet szerint az emberiség egy privilegizált faj, ami intelligenciájának köszönhetően uralni képes a teremtés más entitásait (Sibley idézve\cite{joshanloo_2013_eastern}. Ehhez mérten az egyének törekszenek befolyásolni, uralni és kontrollálni környezetüket, azonban mivel a keleti kultúrákban nincs jelen ez a felsőbbrendűség érzet, itt az egyének inkább a harmónia megteremtésére igyekszenek más élőlényekkel, és a kozmosszal\cite{joshanloo_2013_eastern}.\\
\textsubscript{4}\textit{Elégedettség}}\\
Johanloo (2013) szerint a legnagyobb különbség az élettel való elégedettség dimenzióban, hogy a keleti kultúrákban ez a koncepció magába foglalja a megbékélést önmagunkkal, másokkal és az egész kozmosszal, illetve nem befolyásolhatja az elégedett\-séget a célok elérése és a másokkal való összehasonlítás\cite{joshanloo_2013_eastern}.\\
\textsubscript{5}\textit{Szenvedés értékelése} és \textit{Szenvedés elkerülése}\\
Ahogy már korábban is említettem, egyes kutatók szerint a szubjektív jóllét a pozitív hatások jelenlétének és a negatív hatások hiányának összessége \cite{diener_suh_lucas_smith_1999}. Joshanloo azonban kiemeli, hogy a legtöbb keleti kultúrában elfogadott teória, hogy a bánat része az igazán boldog életnek sőt, a szenvedés elengedhetetlen összetevője a boldogságnak, a szenvedés nélküli boldogság nem teljes. \cite{joshanloo_2013_eastern}.\\
\textsubscript{6}\textit{Spiritualitás és Vallás}\\
Az egyik legnagyobb boldogságkoncepcióbeli kultúrális különbség ebben a dimenzi\-óban található: Míg a nyugati kultúrákban alapvető, hogy a boldogságot és jóllétet ebben a földi életben kell megtapasztalnunk \cite{joshanloo_2013_eastern}, a keleti kultúrákban ez nem ilyen egyértelmű. Egyrészt számos keleti vallás rendelkezik transzcendens tanításokkal, ami a jelen életbeli lemondásokra utasít a következő életbeli boldogság eléréséhez. Másrészt olyan spirituális élmények, mint az Istennel való egység, kétség\-kívül része a keleti boldogságkoncepcióknak, ám számos nyugati modell elveti ezeket az élményeket, babonásnak és primitívnek címkézve őket \cite{joshanloo_2013_eastern}.\\
Összességében elmondható tehát, hogy számos kultúrális különbség létezik az emberek boldogságkoncepciójában, ami pedig szükségessé teszi egy árnyaltabb, kultúrális eltérésekre érzékenyebb boldogságkutatás kialakítását.

\section{Összegzés}
Láthatjuk tehát, hogy a boldogsággal kapcsolatos attitűdök közel sem olyan egyhangúak, mint azt korábbi kutatásokból gondoltuk volna. Megfigyelhetőek ugyanis bizonyos hiedelmek \cite{joshanloo_weijers_2013} melyek kimondottan nemkívána\-tossá teszik számunkra a túlzott boldogság átélését, mint például, hogy \textit{minden jót valami rossz kell kövessen}, tehát ha jelenleg boldogok vagyunk, \textit{valami rossz fog következni}. Keleti kultúrákban megfigyelhető például az a hiedelem is, hogy saját boldogságunk kimutatása rossz hatással van másokra \cite{joshanloo_weijers_2013}, mert iriggyé teszi őket, illetve, hogy a boldogságra való törekvés nem helyes, mert \textit{elvonja az egyén figyelmét} más, fontosabb dolgokról \cite{joshanloo_weijers_2013}. Gilbert és munkatársai megállapították, hogy az ezen hiedelmek hatására kialakult, pozitív érzelmektől való félelem, pozitívan korrelál a \textit{depresszióval, a negatív érzelmektől való félelemmel, stresszel, szorongássa}l és alexithymiaval, illetve, hogy együtt jár a felnőttkori bizonytalan kötődési stílusokkal \cite{gilbert_mcewan_catarino_baiao_palmeira_2013}. Gilbert és munkatársainak eredményei nem csak bizonyítják a boldogságkutatás \textit{klinikai relevanciáját}, de termékeny talajt is biztosítanak jövőbeli empirikus kutatásokhoz, hiszen a boldogságtól való félelem még mindig egy viszonylag fiatal témának számít a pozitív pszichológiában. \\
Azonban a boldogságkutatásnak jeleneg számos limitációja van, többek közt az introspektív jelleg és a nem megfelelő kultúrális érzékenység. Joshanloo (2013) kultúrközi boldogságkoncepció vizsgálatából láthatjuk, hogy nem csak, hogy a boldog\-sággal kapcsolatos attitűdökben figyelhetőek meg kultúrális különbségek, hanem már eleve a \textit{boldogságkoncepcióban is.}














































\printbibliography[title={Hivatkoz\'asok}]
\newrefsection
\pagebreak
\part{Szakmai önreflexió}
\par Mióta elkezdtem a pszichológia alapképzést 2016-ban az ELTE Pedagógiai és Pszichológiai Karán, hatalmas fejlődésen mentem keresztül nem csak szakmailag, de emberileg is. Egyetemi pályafutásom alatt számos olyan kurzuson vettem részt, melyek biztos elméleti és gyakorlati tudással ajándékoztak meg, s melyeket a későbbi tanulmányaim során is lehetőségem lesz hasznosítani. Többek közt megtanultam szakirodalmi összefoglalót írni, s ezáltal bepillantást nyerhettem a pszichológiai kutatások szerteágazó s mégis oly összetett világába. A teljes műhelymunkám elvégzése közben pedig alkalmam nyílt elsőkézből is megtapasztalni, hogy milyen komplex és izgalmas folyamat egy kutatás lebonyolítása, a tervezéstől az adatfelvételen át, az elemzésig. Az alapképzés alatt négy műhelymunkát készítettem, melyből három rész műhelymunka, s egy teljes műhelymunka volt. 

\subsubsection*{Személyiségpszichológia műhelymunka}

\par A legelső rész műhelymunkámat a boldogság-averzió, boldogságtól való félelem témában írtam, Bányai-Nagy Henriett segítségével. A kurzus során elsajátítottam a szakirodalmi összefoglaló írásának alapjait. Megtanultam a megfelelő empirikus források felkeresésének és használatának fortélyait, valamint elsajátítottam az azonos témájú, szerteágazó kutatások összefoglalásának képességét. A kurzusnak köszönhetően célzottan tudok szakirodalmat keresni, s mondandómat képes vagyok megfelelő nyelvezettel megfogalmazni, empirikus adatokkal alátámasztani.

\subsubsection*{Fejlődés-lélektani műhelymunka}

\par A hatodik félévemben három műhelymunka megírására is vállalkoztam. Ezek közül az egyik, Oláh Katalin vezetésével készült el, s témája a halláskárosult gyermekek tudatelmélete volt. A témában számos cikket, kutatást, illetve vizsgálatot kellett áttekintenem, hogy megfelelő képet tudjak adni a jelenlegi, tudományos álláspontról. A kurzus során tovább fejleszthettem az igényes szakirodalmi összefoglaló megírására vonatkozó képességeimet, s rengeteg új információt sajátítottam el a témában. 

\subsubsection*{Általános pszichológia műhelymunka}

\par A harmadik rész műhelymunkámat Honbolygó Ferenc segítségével írtam meg, mely szintén szakirodalmi összefoglaló volt, s témája a zeneterápia hatása volt, autista gyermekek szociális képességeire. A témában folytatott alapos tájékozódás során nem csak rengeteg új információt tanulhattam meg a zeneterápiáról s annak hatásairól, de hobbi zenészként különösen érdekesnek találtam ezt a fajta intervenciót. A rész műhelymunka megírása során megtanultam hogyan is kell esszészerűen összefoglalni az adott témán belüli kutatásokat, egy vázlatos, darabos leírás helyett, melynek hatalmas hasznát vettem a nagy műhelymunka megírása során.

\subsubsection*{Szociálpszichológia műhelymunka}

\par A teljes műhelymunkámat Szekeres Hanna Flóra PhD hallgató segítségével írtam meg, mely a heteroszexuális személyek homoszexuális férfiakkal szembeni attitűdjét és viselkedését vizsgálta. Egészen pontosan arra voltam kíváncsi, hogy azok a heteroszexuális személyek, akiknek több meleg barátja van, vajon pozitívabb attitűddel is rendelkeznek-e a csoport felé, illetve valószínűbben konfrontálják-e a csoporttal szembeni diszkriminációt. A kurzus során megtapasztaltam milyen lebonyolítani egy kutatást a kutatási kérdés megfogalmazásától, az adatfelvételen, kódoláson és elemzésen át, az eredmények publikálásáig. A vizsgálat során szembekerültem az adatfelvétel nehézségeivel, illetve rutint szerezhettem a különböző mérőeszközök kialakításában is, s mindeközben felelevenítettem az SPSS tudásomat. A műhelymunka megírása során megtanultam hogyan illesszem be az általam végzett vizsgálatot egy általános elméleti keretbe, s hasonlítsam össze saját vizsgálatom eredményeit másokéval. A vizsgálat végeredményére különösen büszke vagyok, hiszen egyrészt, a magyar mintán végzett hasonló kutatások száma igen limitált, másrészt olyan eredményeket kaptunk, melyek egy teljesen új ágát térképezik fel a témának, s egy kevésbé ismert terület szakirodalmát bővítik. A vizsgálatunk során mi ugyanis valós viselkedéses helyzeteket alkalmaztunk, vignettek, illetve önbeszámolók helyett, s így az eredményeink is teljesen mások lettek, mint amire a szakirodalmi összegzés alapján számítottunk. Úgy gondolom kifejezetten hasznos volt számomra a kurzus, hiszen olyan új készségeket sajátíthattam el általa, melyek nem csak a jövőbeli tanulmányaim során, de a karrieremben is hasznomra válhatnak.
\\
\par Összességében úgy gondolom, hogy a négy műhelymunkám által elsajátítottam az alapvető készségeket egy önálló kutatás megtervezéséhez s kivitelezéséhez. Többek közt megtanultam hogyan írjak tudományosan igényes szakirodalmi összefoglalót, hogyan vegyek fel, illetve elemezzek adatokat, állítsak fel hipotéziseket, illetve hogyan alkalmazzam az American Psychological Association (APA) által előírt formát. Tisztában vagyok a kutatási eredményeket torzító tényezőkkel, csapdákkal, veszélyforrásokkal és tudom hogyan kerüljem el őket. Megfelelő tudással rendelkezem a statisztikai elemzőprogramok használatát illetően, s nem jelent gondot számomra az angol nyelvű szakirodalom megértése sem. Mindazonáltal megtanultam, hogyan nézzem kritikus szemmel a saját kutatási tervemet, ötleteimet, s ezáltal kiküszöböltem számos módszertani hibát. 

\subsubsection*{További megszerzett kompetenciák}

\par Az ELTE Pedagógiai és Pszichológiai Karán eltöltött három évem alatt számos olyan kurzuson vettem részt, melyek megerősítettek abban, hogy jól döntöttem, amikor ezt a szakot és egyetemet választottam. Magyaródi Tímea Autogén Tréning gyakorlata például nem csak feledhetetlen saját élményt nyújtott számomra, de emellett olyan önszuggesztiós technikával ruházott fel, melyet hatékonyan tudok alkalmazni a mindennapjaimban. A Konfliktuskezelés és asszertív kommunikáció című kurzus által pedig lehetőségem nyílt a kommunikációs szokásaim, illetve problémamegoldó stratégiáim fejlesztésére. A Klinikai pszichológia gyakorlat keretein belül pedig kipróbálhattam magam első interjús helyzetben egy pszichiátriai osztályon, ami az egyik legemlékezetesebb momentuma volt az egyetemi éveimnek. 2017-ben segédkeztem a Pszinapszis promóciós videójának elkészítésében, illetve tanulmányaim kezdete óta minden évben részt is vettem azon. A rendezvény minden évben rendkívül inspirált engem, illetve sokat formált a szakmai hozzáállásomon is. Lehetőségem nyílt többet között kipróbálni egy kutyaterápiás foglalkozást a Pszinapszis keretein belül, ami megerősített abban, hogy a jövőben ezzel szeretnék foglalkozni. Azonban tanulmányaim során igyekeztem minden lehetőséget megragadni, hogy ne csak szakmailag, de emberileg is fejlődhessek. Éppen ezért már elsőévesként jelentkeztem az Erasmus Student Network (ESN) csapatba önkéntesnek. A csoportban eltöltött három évem alatt olyan feladataim voltak, mint: események, csapatépítő találkozók megszervezése, promóciója, lebonyolítása, nemzetközi tanulók orientációja, tanulmányi és magánéleti asszisztálása. Ezen feladataimnak köszönhetően rengeteget fejlődtem olyan aspektusokban mint az asszertivitás, csapatmunka, időmenedzsment és problémamegoldás. Ráadásul a külföldi hallgatókkal való szoros kapcsolatom által lehetőségem nyílt az angol nyelv gyakorlására, illetve nemzetközi baráti kapcsolatok kialakítására. Mindezek mellett a Pedagógiai és Pszichológiai Karon eltöltött idő rengeteg önismeretet adott számomra, illetve egy teljesen új perspektívával szélesítette a látókörömet. 
\\
\par Összességében úgy érzem, hogy a Pedagógiai és Pszichológiai Karon töltött éveim alatt maximálisan kihasználtam az egyetem által nyújtott lehetőségeket és olyan élményeket, illetve tudást szereztem, melyek meghatározó részeivé váltak az identitásomnak. Bár a tanulmányaimat külföldön tervezem folytatni, úgy gondolom, hogy az ELTE által nyújtott stabil alapok a továbbiakban is elkísérnek majd engem s segítenek az esetleges akadályok legyőzésében.
\pagebreak

\includepdf[pages=1-2,nup=1x2,offset=0 -50,delta=0 25, scale=0.65, pagecommand={\part{Prezentáció}}]{pages/thesis_pres}
\includepdf[pages=3-,nup=1x2,offset=0 0,delta=0 25, scale=0.65, pagecommand={}]{pages/thesis_pres}
\footnotesize
\end{document}
