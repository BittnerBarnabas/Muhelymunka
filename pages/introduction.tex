\section{Bevezetés}
A 21. századi fejlett, nyugati társadalmakban szinte univerzálisan pozitív attitűd figyelhető meg a boldogsággal, mint érzéssel kapcsolatban. Korábbi kutatások szerint, a boldogság nem csak a szubjektív jóllétünkhöz járul hozzá, de befolyásolja a fizikai és pszichológiai egészségünket is, kihat társas kapcsolatainkra, mindezek mellett pedig az egyik legfontosabb motiváló erő az emberek életében. Sőt, egyes kutatók szerint a szubjektív jóllét, tulajdonképpen maga a pozitív hatások jelenléte, a negatív hatások hiánya és az élettel való elégedettség.\cite{diener_suh_lucas_smith_1999} Arisztotelész, a nyugati filozófia atyja, ismert idézete szerint: \textit{“Happiness is the meaning and the purpose of life, the whole aim and end of human existence.”} \medskip 
\\ Azonban egyre bővül azoknak a kultúrközi kutatásoknak a száma, melyek megkérdő\-jelezik  a boldogság egyoldalúan pozitív megítélését. Bizonyos, többnyire kollektivista társadalmakon végzett kutatások megállapítottak egy, az individualista társa\-dalmaktól némire eltérő tendenciát, a boldogságtól való félelem, averzió felé (\cite{joshanloo_weijers_2013}; \cite{gruber_mauss_tamir_2011}). Ennek a tanulmánynak a célja, hogy megvizsgálja a boldogság-averzióval kapcsolatos szerteágazó kutatásokat és átfogó képet adjon a témával kapcsolatos jelenlegi, tudományos álláspontról. \medskip 

\pagebreak
\section {Diszkusszió}
\subsection{Aversion to Happiness Across Cultures: A Review
	of Where and Why People are Averse to Happiness \cite{joshanloo_weijers_2013}}
Joshanloo és Weijers (2013) cikke volt az egyik első tudományos munka, ami a boldogság-averzió fogalmával és koncepciójával foglalkozott. Tanulmányuk célja az volt, hogy korábbi empirikus kutatásokon keresztül megvizsgálják, hogy más-más kultúrákban milyen eltérő indokok miatt alakul ki averzió az emberekben, a boldogság különféle változataival szemben. Hipotézisük, vagyis hogy létezik a boldogsággal szembeni averzió, beigazolódott, és bár a keleti kultúrákban kétségkívül jobban megfigyelhető ez a jelenség, a nyugati kultúrákban is jelen van. 





