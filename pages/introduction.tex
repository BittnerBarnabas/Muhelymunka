\section{Bevezetés}
A 21. századi fejlett, nyugati társadalmakban szinte univerzálisan pozitív attitűd figyelhető meg a boldogsággal, mint érzéssel kapcsolatban. Korábbi kutatások szerint, a boldogság nem csak a szubjektív jóllétünkhöz járul hozzá, de befolyásolja a fizikai és pszichológiai egészségünket is, kihat társas kapcsolatainkra, mindezek mellett pedig az egyik legfontosabb motiváló erő az emberek életében. Sőt, egyes kutatók szerint a szubjektív jóllét, tulajdonképpen maga a pozitív hatások jelenléte, a negatív hatások hiánya és az élettel való elégedettség.\cite{diener_suh_lucas_smith_1999} Arisztotelész, a nyugati filozófia atyja, ismert idézete szerint: \textit{“Happiness is the meaning and the purpose of life, the whole aim and end of human existence.”} \medskip 
\\ Azonban egyre bővül azoknak az empirikus kutatásoknak a száma, melyek megkérdő\-jelezik  a boldogság egyoldalúan pozitív megítélését. Léteznek ugyanis olyan populációk, ahol kimutatható egy sokkal ambivalensebb attitűd a boldogság felé. Például, kollektivista társadalmakon végzett kutatások megállapítottak egy, az individualista társa\-dalmaknál némileg erősebb tendenciát, a boldogságtól való félelem, averzió felé \cite{joshanloo_weijers_2013} \cite{joshanloo_lepshokova_panyusheva_natalia_poon_yeung_sundaram_achoui_asano_igarashi}, de a nyugati társadalmakban is ugyanúgy megtalálható ez a jelenség \cite{gilbert_mcewan_catarino_baiao_palmeira_2013}.  Ennek a tanulmánynak a célja, hogy megvizsgálja a boldogság-averzióval, illetve a boldogságtól való félelemmel kapcsolatos szerteágazó kutatásokat és átfogó képet adjon a témával kapcsolatos jelenlegi, tudományos álláspontról. \medskip 

\pagebreak
\section {Diszkusszió}
\subsection{Aversion to Happiness Across Cultures: A Review
	of Where and Why People are Averse to Happiness \cite{joshanloo_weijers_2013}}
Joshanloo és Weijers (2013) cikke volt az egyik első tudományos munka, ami a boldogság-averzió fogalmával és koncepciójával foglalkozott. Tanulmányuk célja az volt, hogy korábbi empirikus kutatásokon keresztül megvizsgálják, hogy más-más kultúrákban milyen eltérő indokok miatt alakul ki averzió az emberekben, a boldogság különféle változataival szemben. Hipotézisük, vagyis hogy létezik a boldogsággal szembeni averzió, beigazolódott, és bár a keleti kultúrákban kétségkívül jobban megfigyelhető ez a jelenség, a nyugati kultúrákban is jelen van. A keleti kultúrákon végzett "fear of happiness" vagyis "boldogságtól való félelem" mérés során a kutatók azt találták, hogy eltérő mértékben ugyan, de minden vizsgált kultúrára jellemzőek voltak úgynevezett \textit{hiedelmek} \cite{joshanloo_weijers_2013}}. \medskip
 \\ A tanulmány 4 hiedelmet (belief) emel ki a boldogság-averzió vonatkozó mutatójaként \cite{joshanloo_weijers_2013}: 
\begin{enumerate}
	\item A boldogság megnöveli az esélyét, hogy valami rossz dolog fog történni velünk
	\item A boldogság rosszabb emberré tesz minket
	\item A boldogság kimutatása rossz hatással van ránk és másokra
	\item A boldogságra való törekvés rossz hatással van ránk és másokra

\end{enumerate}
Konklúzióként megállapíthatjuk, hogy a boldogság-averzió egy érdekes és sokrétű jelenség, mely minden kultúrában megtalálható eltérő mértékben: Míg a keleti kultúrákban a vallás (pl.: buddhizmus és a vágyakról való lemondás), az erős konformitás és a (pozitív) érzelmek moderált kimutatása mind hozzájárulhat a hiedelmek megszilárdulásához, addig az individualista társadalmak egyén-központúsága és a pozitív érzelmek kimutatásának hangsúlyossága korlátozhatja azt \cite{joshanloo_lepshokova_panyusheva_natalia_poon_yeung_sundaram_achoui_asano_igarashi}.

\subsection{Fears of compassion and happiness in relation
	to alexithymia, mindfulness, and self-criticism \cite{gilbert_mcewan_gibbons_chotai_duarte_matos_2011}}
Gilbert és munkatársai (2011) az elsők közt kezdtek foglalkozni a boldogságtól való félelemmel. Empirikus tanulmányuk célja, hogy egy új, a kutatók által kidolgozott "Fear of Happiness" vagyis "Boldogságtól Való Félelem" skálával \cite{gilbert_mcewan_gibbons_chotai_duarte_matos_2011} megvizsgálják az összefüggéseket az odaadás (compassion) és a boldogságtól való félelem (fear of happiness), valamint olyan érzelem-feldolgozó kompetenciák közt, mint az  alexithymia, mindfulness, és empátia, illetve mindezek összefüggését az önkritikával és pszichopatológiával. Ezt a skálát Gilbert terápiás munkassága hatására fejlesztették ki a kutatók és olyan tételek tartalmazott, mint: \textit{"Félek, hogyha jól érzem magam, valami rossz fog történni"} \cite[o. 381]{gilbert_mcewan_gibbons_chotai_duarte_matos_2011}. A kutatás hipotézise, hogy összefüggést fognak találni az egyén odaadástól való félelme és az érzelem feldolgozása közt, amit a statisztikai elemzés be is bizonyított. Korrelációt találtak mind az odaadástól mind a boldogságtól való félelem és a szorongás, erős önkritika, és a mindfulness-el és alexithymia-val kapcsolatos nehézségek közt. A boldogságkutatással kapcsolatos legfontosabb eredmény, hogy a kutatók kiemelkedően magas (r=.70) korrelációt találtak a boldogságtól való félelem és a depresszió közt, ami bizonyítja a boldogságkutatás klinikai relevanciáját. A kutatás legnagyobb limitációja, hogy a kísérleti személyek 83\%-a nő volt, ami kétségessé teszi a minta reprezentatívságát \cite{gilbert_mcewan_gibbons_chotai_duarte_matos_2011}.

\subsection{Fears of happiness and compassion in relationship
	with depression, alexithymia, and attachment
	security in a depressed sample \cite{gilbert_mcewan_catarino_baiao_palmeira_2013}}
Gilbert et. al (2011) kutatási eredménye, miszerint a boldogságtól való félelem magasan korrelál (r=0.7) a depresszióval, nagyban befolyásolták ezt a tudományos értekezést. Ennek a tanulmánynak a célja, hogy megvizsgálja a 2011-ben talált, főleg női egyetemistákon végzett kísérleti eredményeket depressziós mintán. A kutatók hipotézise az volt,  hogy \textsubscript{1} \textit{ A depressziós kísérleti személyek nagyobb félelmet fognak mutatni a boldogság és odaadás (compassion) felé,}  mint a nagyrészt egyetemis\-tákból álló minta \textsubscript{2}A pozitív érzelmektől való félelem \textit{korrelálni fog a depresszióval, stresszel, szorongással, alexithymia-val }   \textsubscript{3} A pozitív érzelmektől való félelem, együtt fog járni \textit{a gyengébb minőségű felnőttkori kötődéssel} \cite{gilbert_mcewan_catarino_baiao_palmeira_2013}. Mindhárom hipotézis beigazolódott, de különösen érdemes megemlíteni, hogy a depressziós mintában magasabb a boldogságtól való félelem, mint a tanulókkal elvégzett kísérletben \cite{gilbert_mcewan_gibbons_chotai_duarte_matos_2011}, tehát a depressziós populáció jobban tart a boldogságtól, mint az egyetemista. A boldogságtól való félelem magasan korrelált a felnőttkori bizonytalan kötődési stílusokkal, alexithymiaval, és a legpontosabban jósolta be a stresszt és szorongást \cite{gilbert_mcewan_catarino_baiao_palmeira_2013}. A kutatási eredményeknek gyakorlati relevanciája is van, hiszen segíthet mélyebben megérteni a depressziós populáció érzelmi kvalitását. Azonban fontos limitáció a kísérlet introspektív mivolta, illetve a szociális kívánatosság befolyásoló ereje, amit nem lehet figyelmen kívül hagyni \cite{gilbert_mcewan_catarino_baiao_palmeira_2013}.

\subsection{Cross-Cultural Validation of
	Fear of Happiness Scale Across 14 National Groups \cite{joshanloo_lepshokova_panyusheva_natalia_poon_yeung_sundaram_achoui_asano_igarashi}}
Ez a kultúrközi tanulmány 14 nemzeten vizsgálja a Joshanloo (2013) által kidolgozott "Fear of Happinness Scale-t (FHS)" azaz a "Boldogságtól Való Félelem Skálát". A kutatás célja részben e skála validitásának ellenőrzése, egy széleskörű mintán, és azon az elképzelésen alapul, hogy bizonyos kontextusokban az emberek nem-kívánatosnak tartják a boldogságot, vagy egyenesen tartanak tőle \cite{joshanloo_lepshokova_panyusheva_natalia_poon_yeung_sundaram_achoui_asano_igarashi}. A kutatási hipotézisek, hogy \textsubscript{1} Egyének szintjén az FHS negatívan fog korrelálni az élettel való elégedettséggel \textsubscript{2} Kultúrális szinten az FHS negatívan fog korrelálni a szubjektív-jólléttel \textsubscript{3} Bizonyos vallásos csoportokhoz való tartozás pozitívan fog korrelálni az FHS-el \cite{joshanloo_lepshokova_panyusheva_natalia_poon_yeung_sundaram_achoui_asano_igarashi}. Bár a skála validitása megfelelőnek bizonyult és mindhárom hipotézis teljesült, az FHS és a szubjektív-jóllét negatív korrelációja igen gyenge (\textless.15) volt, míg az FHS és az élettel való elégedettségé viszonylag magas. A tanulmány szerint olyan vallások, mint a \textit{Buddhizmus, Hinduizmus, Taoizmus} pozitívan, míg a \textit{Kereszténység} negatívan korrelál a boldogság-averzióval. \cite{joshanloo_lepshokova_panyusheva_natalia_poon_yeung_sundaram_achoui_asano_igarashi}. A kutatás legfőbb limitációja, hogy egyes nemzetek alulreprezentáltak, ami torzíthatja az eredményeket, valamint, hogy a résztvevők túlnyomórészt fiatal felnőttek voltak, ami megkérdőjelezi az eredmények általánosíthatóságát felnőtt populációkra.





