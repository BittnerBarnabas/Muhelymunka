\section{Előszó}
A leszbikus, meleg, biszexuális, transznemű és queer (LMBTQ) csoporttal szembeni általános attitűd vizsgálata központi szerepet tölt be a modern szociálpszichológiában, s a témában folytatott kutatásoknak általában jelentős társadalmi implikációi vannak. Több kutatás is kimutatta például, hogy azok a személyek, akik pozitív attitűddel rendelkeznek a csoport felé, hajlamosabbak voltak támogatni a melegek egyenjogúságára törekvő intézkedéseket (cite).  Poteat és Vecho (2016) pedig kimutatta, hogy azok a személyek, akiknek vannak homoszexuális barátaik, aktívabb védelmező viselkedést mutatnak az áldozat felé homofób bullying helyzetekben. Ezekből az adatok arra utalnak, hogy a csoportok közti barátság nem csak attitűdbeli változáshoz vezet de az explicit viselkedésben is kifejtheti hatását. Ennek a vizsgálatnak alapvetően két célja van. Egyrészt, hogy feltérképezze, befolyásolja-e a heteroszexuális egyének homoszexualitással kapcsolatos implicit attitűdjét az, hogy van meleg barátjuk. Másrészt, hogy felmérje ezen baráti kapcsolatok hatását a heteroszexuális személyek explicit viselkedésére, az LMBTQ csoport tagjait diszkrimináló kísérleti helyzetben. A kutatás mögötti motivációt egyrészt a magyar mintán végzett hasonló témájú kutatások limitált száma, másrészt a téma társadalmi és aktuálpolitikai relevanciája adta. 
\pagebreak
\section{Bevezetés}
Bár az Egészségügyi Világszervezet revíziója során törölte a homoszexualitáshoz kapcsolódó tételeket (F66) a Betegségek Nemzetközi Osztályozásából (BNO), a heteroszexuálistól eltérő szexuális orientációk patologizálását igazoló empirikus bizonyítékok hiányában (cite), a magyar populáció számottevő része devianciaként gondol arra (citehomofobiamagyarországon), és többségük elutasítja az azonos nemű párok örökbefogadáshoz való jogát (citehomofobiamagyarországon). Ezeknek az adatoknak a tükrében különösen fontos és érdekes  feladat azoknak a tényezőknek az azonosítása, melyek egy elfogadóbb attitűd kialakításához vezetnek az LMBTQ+ csoporttal szemben. Allport kontaktushipotézise óta tudjuk, hogy a megfelelő csoportközi kontaktus csökkenti az előítéletet (citeallport). A minőségi csoportközi kontaktus (barátság) pedig számos kutatás szerint növeli a csoporttal szembeni empátiát (citeseveral, Abbot, pettigrew), sőt António, Guerra és Moleiro (2017) kutatásában még a kiterjesztett kontaktus is (olyan barátok, akiknek vannak homoszexuális barátaik) növelte azt. 
Anderssen (2002) 2 éven át tartó longitudinális vizsgálatában 
A saját csoporton kívüli, azaz outgroup tagokkal való kontaktusnak számos más előnye is van. Capozza, Falvo és Trifiletti (2014) vizsgálatukban azt találták, hogy a kiterjesztett kontaktus összefügg a csökkent outgroup infrahumanizációval. Ez azért fontos eredmény, mert az infrahumanizáció során az egyén az ingroup-ot felruházza olyan humánspecifikus jellemzőkkel, mint a másodlagos érzelmek (pl.: bűntudat) és a tudatosság, míg az outgroup tagjait kevésbé, s ez diszkriminációhoz, illetve agresszióhoz is vezethet (cite Waytz, Epley 2012). 
\\ \par
Azonban a csoportközi barátságok nem csak attitűdbeli változást vonhatnak maguk után. Egy másik, a homoszexualitással kapcsolatos implicit attitűdöket és explicit viselkedéses szándékokat vizsgáló kísérletben a kutatók a hosszútávú közvetlen kontaktus hatásait hasonlították össze az outgroup személynek való rövidtávú kitetséggel. Az eredmények azt mutatták, hogy azok a személyek, akiknek limitált kontaktusa volt melegekkel és leszbikusokkal nem csak szignifikánsan magasabb melegellenes implicit attitűddel rendelkeztek (cite dasgupta rivera), de hajlamosabbak voltak diszkrimináló szavazói szándékot mutatni a viselkedéses helyzetben (cite Dasgupta, Rivera). Ezzel szemben azok a személyek, akiknek már volt előzetes hosszútávú kapcsolata a csoporttal, alacsonyabb melegellenes implicit attitűddel rendelkeztek, és hajlamosabbak voltak megszavazni az egyenlő jogokat (pl.: házasság) a csoportnak (citerivera). Ezek az eredmények azért különösen lényegesek, mert egyrészt bizonyítják a csoportközi barátságok implicit attitűdre gyakorolt hatását, másrészt arra utalnak, hogy a minőségi csoportközi kontaktus az explicit viselkedéses helyzetekben is hatást gyakorol. Ezeket az eredményeket támasztja alá António és munkatársainak (2017) korábban már említett vizsgálata, hiszen a kiterjesztett kontaktus nem csak a személyek megnövekedett empátiáját jósolta be a csoport iránt, de aktívabb intervenciókhoz is vezetett a vinyetták által leírt homofób bullying helyzetekben (cite). Pettigrew (1997) szerint a csoportközi barátság azért is nagyon lényeges, mert az általa létrejött kapcsolat megfelel a kontaktushipotézis (citeallport) feltételeinek, ami pedig az előítéletesség csökkentésének alapja (citepettigrew).  Ezek a feltételek a következők:
\begin{enumerate}
	\item Egyenlő státusz a csoportok közt
	\item Közös célok
	\item Együttműködés
	\item Támogató társadalmi normák és intézmények
\end{enumerate}

Az LMBTQ személyekkel kötött barátságok viselkedésre gyakorolta hatása azért is nagyon fontos, mert Poteat és Vecho (2016) adatai szerint a 722 fő középiskolás mintájuknak 66.8\%-a tapasztalt homofób viselkedést legalább egyszer az elmúlt egy hónapban (cite). Ezek az adatok különösen riasztóak, ha figyelembe vesszük, hogy azoknak a gyermekeknek, akik voltak már bullying áldozatai, többször vannak öngyilkos gondolataik (citehinduja,patchin), illetve nagyobb valószínűséggel követnek el öngyilkosságot (citehinduja). Poteat és Vecho (2016) kutatásában   mindazonáltal kiválasztották, azokat a gyermekeket, akik tapasztaltak homofób viselkedést az elmúlt egy hónapban, majd megkérdezték őket, hogy az elmúlt 30 napban milyen gyakran csinálták a következőket:
\begin{enumerate}
	\item Szóltak egy felnőttnek az incidensről 
	\item Megpróbálták megállítani az elkövetőt 
	\item Támogatták az áldozatot 
	\item Rávettek másokat, hogy támogassák az áldozatot 
	\item Felszólaltak az áldozat érdekeiért 
	\item Segítették, bátorították az áldozatot, hogy szóljon az incidensről egy felnőttnek 
	\item Kifejezték a nemtetszésüket a történtekkel kapcsolatban 
	\item Nem járultak hozzá a szituációhoz (pl.: cyberbullying kapcsán nem osztották meg képet, posztot)
\end{enumerate}

Az eredmények szerint, azok a gyermekek, akiknek volt LMBTQ barátjuk, gyakrabban vettek részt a felsorolt intervenciókban(citepoteat) illetve többször választottak aktívabb intervenciókat (pl.: konfrontáció), mint azok a társaik, akiknek nem volt ilyen kapcsolata (citepoteat). 
Ezek az adatok pedig szintén támogatják azt az elképzelést, hogy a csoportközi barátságok a viselkedésre is hatást gyakorolnak. Az LMBTQ csoport tagjaival való kontaktus még arra is hatással van, hogy az iskolai alkalmazottak (pl.:tanárok) mennyire legitimizálják, vagy ellenzik a homofób bullying-ot az iskolai környezetben (citeZotti). Zotti és munkatársai (2018) friss tanulmányukban azt találták, hogy a leszbikus, illetve meleg személyekkel való kontaktus hiánya meggátolta az intervenciót a csoport elleni diszkriminációs helyzetekeben (citezotti), illetve elősegítette a  homofób bullying személyes legitimizációját az iskola dolgozói körében (cite zotti).
\\
\par
Mindezek alapján megállapíthatjuk, hogy az LMBTQ csoporttal való kontaktus növeli az empátiát (cite), növeli az egyenjogúságra való törekvő intézkedések támogatását (cite), pozitívabb implicit attitűdöt eredményez (citedasgupta), aktívabb intervenciós szándékhoz vezet (citepoteat), és csökkenti a homofób bullying legitimizációját (citezotti), valamint az outgroup  tagok infrahumanizációját (citecapozza). Azonban kulcsfontosságú meghatározni ezen kontaktus jellegét. Baunach, Burgess és Muse (2009) amerikai egyetemistákon folytatok kísérlete kimutatta, hogy a különböző típusú érintkezések a csoporttal, más-más hatással voltak a heteroszexuálisok előítéletességére. Először is azt találták, hogy a kapcsolatok minősége nagyobb szerepet játszott az előítélet csökkentésében, mint azok mennyisége (citebaunach). Másodszor pedig megállapították, hogy bizonyos fajta kapcsolatok fontosabbak voltak, mint mások (citebaunach). Nevezetesen, míg tulajdonképpen az összes fajta kontaktus csökkentette az előítéletességet a heteroszexuálisokban (citebaunach), az előítéletességre gyakorolt legnagyobb hatást a minőségi baráti kapcsolatok megléte okozta (citebaunach). A minőségi baráti kapcsolatok még a meleg családtagokkal való kapcsolatnál is szignifikánsabb hatást fejtett ki az egyének előítéletességére (citebaunach). Ennek oka lehet például az, hogy a sztereotípiák fenntartása érdekében a személy a meleg családtagokra alkategóriát képez a csoporton belül, s így nem csökkenti az outgroup iránti előítéletességét.
\\
\par
Összeségében elmondhatjuk tehát, hogy az LMBTQ csoporttal létesített kontaktus, de főleg a barátság (citebaunach) számos kutatás szerint nem csak pozitív attitűdbeli változásokkal jár (citedasgupta), de emellett az explicit viselkedéses helyzetekben is aktívabb illetve asszertívabb intervencióhoz vezet (citezotti,poteat).

\pagebreak
\section{Hipotézisek}
Ennek a tanulmánynak alapvetően két fő hipotézise van:
\begin{enumerate}
	\item Azok a személyek, akiknek van meleg barátja, pozitívabb implicit attitűddel fognak rendelkezni a csoport felé, mint akiknek nincs ilyen kapcsolata.
	\item Azok a személyek, akiknek van meleg barátja, nagyobb valószínűséggel fogják konfrontálni a homofób megszólalást az explicit viselkedéses helyzetben.
\end{enumerate}


\section{Módszer}
\subsection{Résztvevők}
106 fő magyar nemzetiségű személy (75 nő és 28 férfi) vett részt a kutatásban. A résztvevők életkora 18 évtől, 51 évig terjedt (M=24.67, SD=7.34). A résztvevők többsége felsőfokú tanulmányokat folytató diák volt (57\%), 28\% felsőfokú végzettséggel rendelkezett, 9\% - nak középiskolai végzettsége és 3\% -nak általános iskolai végzettsége volt. A kísérleti személyek 83\% - a vallotta heteroszexuálisnak magát, s mivel a tanulmányban vizsgált célcsoport a homoszexuálisok voltak, azokat a személyeket, akik homoszexuálisnak, biszexuálisnak, vagy egyébnek vallották magukat, rájöttek a megtévesztésre, illetve nem kívánták meghatározni a nemüket vagy szexuális orientációjukat, eliminiáltuk az elemzésből. Így a végső minta 96 személyből állt (72 nő és 26 férfi). 
\subsection{Eljárás}
\subsubsection{Adatfelvétel}
A kutatás az Eötvös Loránd Tudományegyetem Pedagógiai és Pszichológia Karának és Szociálpszichológia Tanszékének keretein belül jött létre. Az adatok online kerültek begyűjtésre. A résztvevőket (\textgreater 18) a közösségi médiumon keresztül toboroztuk, majd tájékoztattuk őket, hogy a részvétel önkéntes, anonim, illetve bármikor  indoklás nélkül megszakítható. A kérdőív kitöltése kizárólag asztali gépről, vagy laptopról volt lehetséges, mivel a Bizalom Játékot más eszköz nem támogatta. 
\subsubsection{Megtévesztés}
A résztvevők azt a tájékoztatást kapták, hogy először egy online játékot fognak játszani, majd egy kérdőív kitöltésére kértük meg őket. A kísérleti személyek úgy tudták, hogy a ¨Bizalom Játék¨ - kal (citehanna) fognak játszani, ahol azt szeretnénk megfigyelni, hogy változik-e az emberek bizalma, ha valaki megfigyeli őket, illetve befolyásolja-e a bizalmat a megfigyelő neme. A résztvevők ezután ezt az instrukciót kapták:
\begin{quoting}[\itshape]
	¨Hogyan játsszák a Bizalom Játékot? \\
	- Két ember játszik egymással. Az első játékos kap egy kezdő összeget (például 200 Ft-ot) \\
	- Az első játékos három opció közül választhat: a második játékosnak odaadja az egészet
	(200Ft), a felét (100Ft) vagy semmit (0Ft, tehát megtartja magának a 200Ft-ot). Ezeket az
	opciókat a játékban az „MINDEN”, „FELE” és „SEMMI” gombok jelölik majd. Tegyük fel, hogy az első játékos úgy dönt, hogy a pénze felét (100Ft) odaadja. \\
	- Ez az összeg automatikusan megháromszorozódik, és ezt a tripla összeget kapja meg a második játékos (300Ft). \\
	- A második játékos két opció közül választhat: vagy megosztja ezt a tripla összeget (tehát mindketten 150Ft-ot kapnak) vagy semmit nem ad vissza (tehát megtartja magának a 300Ft-ot). Tegyük fel, hogy a második játékos úgy dönt, hogy megosztja (150Ft-ot ad vissza az első játékosnak). \\
	- Ezzel vége a játéknak. \\
	- A példa alapján, az első játékos összesen 250 Ft-ot nyert (100 Ft, amit az elején megtartott, 
	plusz 150 Ft, amit a második játékos visszaküldött), míg a második játékos 150 Ft-ot nyert.
	Figyeljük meg, hogy a játékosok akkor nyernek a legtöbb pénzt, ha az első játékos megbízik a második játékosban, és odaadja az összes pénzt, a második játékos pedig megbízható és fair módon megosztja azt, amit kapott. Ebben az optimális bizalmi helyzetben, ha 100 Ft a kezdő összeg, akkor mindkét játékos a végén 300 Ft-ot keresne.¨
\end{quoting}

A résztvevők tehát úgy tudták, hogy valódi személyekkel fognak játszani a Bizalom Játékban, azonban a többi résztvevő valójában nem volt valós. A játékban a kísérleti személy, először mindig megfigyelő szerepet töltött be, azaz láthatta ahogy másik két ¨játékos¨ játssza a Bizalom Játékot. A játék során a résztvevők által megfigyelt személy (Márk) készségesen megosztotta vagyonát partnereivel, mindaddig, amíg a harmadik körben össze nem került egy Dani nevű játékossal, akinek neve mellett megjelent a szivárványzászló, a meleg közösség egyik legfontosabb szimbóluma. A kísérleti személy ekkor azt láthatta, hogy Márk nem osztja meg vagyonát Danival, majd homofób üzenetet írt a résztvevőnek (pl.: ¨na erre a langyira én pénzt nem bízok¨). A résztvevőknek ekkor lehetőségük nyílt válaszolni az üzenetre (konfrontálódni) vagy továbblépni válaszolás nélkül. A következő körben a kísérleti személyek is játszhattak Danival, majd egy előre kódolt hiba véget vetett a játéknak. Erre a megtévesztésre azért volt szükség, mert anélkül a társas kívánatosság nagyban befolyásolta volna a viselkedést, s  ez megakadályozta volna, hogy a valóságot tükröző eredményeket kapjunk.


\subsubsection{Kérdőív}
A játék után a résztvevők egy homoszexualitással kapcsolatos kérdőívcsomagot töltöttek ki, ami egy Homofóbia skálából, illetve Rokonszenv skálából állt. Ezután megkérdeztük a résztvevőket, hogy hány meleg barátjuk van, illetve, hogy véleményük szerint barátaiknak hány százaléka meleg. A játék, illetve a kérdőív kitöltése összesen nagyjából 30-40 percet vett igénybe. A kérdőív kitöltése után a résztvevők utólagos tájékoztatásban részesültek, majd megköszöntük a részvételüket. 

\subsubsection{Statisztikai módszerek}
Az adatok elemzése az IBM SPSS szoftver 25.0.0.-ás verziójával történt. A hipotézis 1 Pearson -féle korrelációval, a hipotézis 2 egyszempontos variancianalízissel (ANOVA) került ellenőrzésre. 

\subsection{Mérőeszközök}
\subsubsection{Homofóbia skála}
A résztvevők egy általunk összerakott 10 elemből álló attitűdskálát töltöttek ki, ami tulajdonképpen egy módosított Attitudes Toward Homosexuality Scale (Anderson, 2017) és módosított Attitudes Toward Gays and Policy Support Index (Jang, 2014) volt. A skála egy ötfokú Liker-skála volt, ahol az 1= egyáltalán nem értek egyet, 5=teljesen egyetértek választ jelölte. A skála olyan tételeket tartalmazott, mint ¨Támogatom a meleg házasságot¨, ¨A homoszexualitás természetellenes¨ vagy ¨Egyáltalán nem zavarna, ha kiderülne, hogy a gyermekem meleg vagy leszbikus¨. A skála megbízhatósága jó (L=0.88).

\subsubsection{Rokonszenv skála}
A  szintén általunk készített rokonszenv skálán a résztvevőknek egy 100 fokos skálán kellett megjelelölniük, hogy az egyes csoportok átlagos tagjait mennyire találják ellenszenvesnek, vagy rokonszenvesnek, ahol a 0 fok = rendkívül ellenszenves, 100 fok = rendkívül rokonszenves választ jelölte. A rokonszenv skálán megítélt csoportok: muszlimok, zsidók, romák, bevándorlók, melegek.

\subsubsection{Barátok száma}
A résztvevők először egy 4 fokos skálán ítélték meg, hogy hány meleg barátjuk van: 1 = egyáltalán nincs, 2 = inkább nincs, 3 = inkább van, 4 = sok van. Ezután azt kérdeztük tőlük, hogy megítélésük szerint barátaiknak hány százaléka meleg (0\% - 100\%).

\pagebreak
\section{Eredmények}
A kutatási kérdések megválaszolása előtt leíró statisztikát hajtottam végre a mérőeszközökre (Homofóbia skála, Rokonszenvskála, Barátok száma). A játékban résztvevő 96 főből sajnos csak 50 fő töltötte ki a kérdőívet is, a többiek a játék után megszakították a részvételt. A Homofóbia skála megbízható ($\alpha=0.88,\  N=50, \  M=2.42, \ SD=0.96$). A rokonszenv skála ($N=50,\  M=63,\  SD=29$) és barátok számának ($N=50, \ M=2.38, \ SD=0.87$) adatai normálisak. 

\begin{table}[h]
	\centering
	\begin{tabular}{|l|l|l|l|}
		\hline
		& \textit{N}  & \textit{M}    & \textit{SD}   \\ \hline
		Homofóbia skála  & 50 & 2.42 & 0.96 \\ \hline
		Rokonszenv skála & 50 & 63   & 29   \\ \hline
		Barátok száma    & 50 & 2.38 & 0.87 \\ \hline
	\end{tabular}
	\caption{Leíró statisztika}
	\label{table:1}
\end{table}

Az adatainkon végzett Pearson-korrelációs tesztek alapján a Homofóbia skála és a meleg barátok száma között negatív korreláció figyelhető meg ($r= -0.492, \  p<0.05$), illetve ugyanez a negatív korreláció mutatkozik a Homofóbia skála és a meleg barátok százalékos aránya közt ($r= -0.335, \  p=0.017$). Mindezekhez hasonlóan a Rokonszenv skála és a meleg barátok száma közt pozitív korreláció figyelhető meg ($r= 0.413, \  p=0.003$), illetve a Rokonszenv skála és a meleg barátok százalékos aránya közt is pozitív a korreláció ($r=0.318, \  p=0.024$). 

\begin{table}[h]
		\centering
	\begin{tabular}{|l|l|l|l|}
		\hline
		&   & Meleg barátok száma & Meleg barátok százalékos aránya \\ \hline
		\multirow{2}{*}{Homofóbia skála}  & r & -0.492              & -0.335                          \\ \cline{2-4} 
		& p & \textless{}0.05     & 0.017                           \\ \hline
		\multirow{2}{*}{Rokonszenv skála} & r & 0.413               & 0.318                           \\ \cline{2-4} 
		& p & 0.003               & 0.024                           \\ \hline
	\end{tabular}
	\caption{Korreláció}
	\label{table:2}
\end{table}

Ezek az adatok alátámasztják az első számú hipotézist, vagyis, hogy a meleg barátok száma, pozitívabb implicit attitűdhöz fog vezetni. Az egyszempontos varianciaanalízis (ANOVA) alapján, nincs szignifikáns különbség a Bizalom Játékban mutatott bizalomban (vagyis abban, hogy mennyi pénzt adtak a résztvevők a meleg játékosnak) azok között, akiknek van meleg barátja, illetve akiknek nincs ($F=0.846, \ p=0.476$). Az egyszempontos variancianalízis (ANOVA) alapján nincs szignifikáns különbség a  homofób megjegyzés konfrontálódásában azok között a személyek között, akiknek van, illetve nincs meleg barátja ($F=1.090, \ p=0.371$). Ezek az adatok tehát nem támasztják alá a második számú hipotézist, vagyis, hogy azok személyek, akiknek van meleg barátja valószínűbb, hogy konfrontálni fogják a homofób megjegyzést a kísérleti helyzetben, mint akiknek nincs ilyen baráti kapcsolatuk. 

\section{Diszkusszió}
Ennek az exploratív tanulmánynak a célja az volt, hogy megvizsgálja a melegekkel való baráti kapcsolatok hatásait a heteroszexuális személyek LMBTQ csoporttal szembeni attitűdjére, illetve, hogy felmérje ezen baráti kapcsolatok hatását az explicit viselkedésre. A jelen tanulmány több aspektusban is kiterjesztette a témában jelenleg fellelhető vizsgálatokat, hiszen nem csak a baráti kapcsolatok attitűdre gyakorolt hatásait vizsgálta, de azt is feltérképezte, hogy a baráti kapcsolatok során kialakult pozitív attitűd átfordul-e viselkedéses intervencióba (konfrontációba) a csoporttal szembeni diszkriminatív helyzetekben.

\subsection{Barátok száma és Homofóbia}
A bevezetőben bemutatott számos kutatáshoz hasonlóan (citedasgupta, pettigrew) statisztikai analízisünk során azt az eredményt kaptuk, hogy azok az egyének, akiknek van meleg barátja, pozitívabb implicit attitűddel rendelkeznek a csoport iránt.  Tehát azok a résztvevők akiknek volt baráti kapcsolata melegekkel, alacsonyabb pontszámot értek el a Homofóbia skálán, mint azok a résztvevők, akiknek nem volt ilyen kapcsolata. Ezek az adatok alátámasztják az első számú hipotézisünket, s szélesítik azoknak a vizsgálatok körét, melyek az outgroup tagokkal való baráti kapcsolatok jelentőségét hangsúlyozzák az előíteletesség csökkentésének szempontjából.

\subsection{Barátok száma és Rokonszenv}
A meleg barátok száma és a homoszexuálisok iránt érzett rokonszenv közt elemzésünk során  szignifikáns pozitív korrelációt találtunk. Ez azt jelenti, hogy azok a személyek, akiknek volt  legalább egy meleg barátjuk, rokonszenvsebbnek találták magát az LMBTQ csoportot. Ezek az adatok újfent támogatják az első számú hipotézisünket, vagyis, hogy az outgroup taggal létesített baráti kapcsolat együttjár a csoporttal szembeni pozitív attitűddel a heteroszexuális személyek körében. Ennek a kapcsolatnak az irányát azonban számos szerző feszegeti munkájában. Pettigrew (1997?) szerint például, azok a személyek akiknek eleve pozitívabb az attitűdje a kül-csoport felé, nagyobb valószínűséggel létesítenek barátok kapcsolatokat azok tagjaival, aminek hatására még pozitívabb attitűdjük lesz feléjük, s ezáltal a személy méginkább nyitott lesz új baráti kapcsolatok kialakítására (citepettigrew).

\subsection{Barátok száma és Konfrontáció}
A jelen tanulmány bevezetőjében bemutatott vizsgálatoktól eltérően, elemzésünk során azt találtuk, hogy nem volt szignifikáns eltérés a csoporttal szembeni diszkriminációs helyzet konfrontációjában attől függően, hogy volt-e a személyeknek meleg barátja. A konfrontáció azzal sem függött össze, hogy a résztvevők az enyhe (¨na erre a langyira én pénzt nem bízok¨) vagy erős (¨inkább pofont adnék ennek a buzinak, mint pénzt!!!¨) homofób megjegyzésnek voltak tanúi. Ezek az eredmények nem támogatják a második számú hipotézist, ami a meleg barátok száma és a konfrontáció között összefüggést feltételezett. Ez az eredmény azért különösen érdekes, mert a vizsgált szakirodalmak egyöntetűen azt implikálták, hogy a homoszexuális barátok száma aktív és asszertív intervencióhoz vezet homofób bullying helyzetekben (cite). Ezek az adatok egybevágnak azonban Crosby és Wilson (2015) eredményeivel, akik kutatásukban összehasonlították a vizsgálati személyek képzelt illetve valós érzelmi és viselkedéses válaszait ugyanarra a becsmérlő homofób kijelentésre. Azok a résztvevők, akik elképzelték, hogy hallják a homofób megjegyzést szignifikánsan erősebb negatív érzelmi státuszt jelentettek, mint azok akik valóban hallották azt (citecrosby), s majdnem felük úgy gondolta, hogy valós helyzetben konfrontálná az elkövetőt (citecrosby). Ehhez képest, azok közül a személyek közül, akik valóban hallották a homofób megjegyzést, senki nem konfrontálta azt (citecrosby). Tehát Crosby és Wilson (2015) adatai egy egyértelmű diszkrepanciát mutatnak a személyek elképzelt, illetve valós reakciói közt (citecrosby). Nagyon hasonló eredményeket talált Kawakami, Dunn, Karmali és Dovidio (2009), akik szerint az emberek sokszor rosszul mérik fel, hogy hogyan reagálnának le rasszista megszólalásokat (citekawakami). Ebben a kutatásban a szerzők megkülönböztettek ¨előrejelző/forecaster¨ illetve ¨átélő/experiencer¨ csoportokat, majd megfigyelték, hogy hogyan reagálnak a résztvevők ugyanarra a rasszista megszólalásra (citekawakami). Eredményeik szerint az előrejelzők úgy gondolták, hogy a rasszista megjegyzés nagyon felzaklatná őket, az átélők közül viszont csak nagyon kevesen mutattak érzelmi disztresszt annak hallatán (citekawakami).
Mindezek alapján érthető az eltérés a jelen tanulmány s a bevezetőben bemutatott vizsgálatok eredményei között, ha figyelembe vesszük, hogy az utóbbiak mind valamilyen önbevallálos, illetve elképzelt, teoretikus vizsgálati eszközre (pl.: vinyettek) hagyatkoztak. 

\subsection{Barátok száma és a Bizalom Játék}
Ahogy a ¨Módszer¨ pont alatt említettük a résztvevőknek lehetősége volt játszani a meleg játékossal a Bizalom Játékban, azonban a meleg barátok száma nem függött össze az irántuk mutatott bizalommal (vagyis azzal, hogy mennyi pénzt adtak nekik a játékban). Ez az eredmény szintén azt mutatja, hogy bár a meleg barátok száma együttjár a csoport iránti pozitív attitűddel, ez a pozitív attitűd nem fordul át a csoporttal szembeni viselkedésbe. 
\par 
Ezeknek az eredményeknek számos releváns implikációja van. Például felhívják a figyelmet arra, hogy az outgrouppal szembeni pozitív attitűdök nem feltétlenül járnak együtt a csoport védelmezésével, illetve a csoportot érő diszkrimináció konfrontációjával. Ez hétköznapi példákkal élve azt jelenti, hogy bár a személy pozitívan viszonyul az LMBTQ csoport tagjaihoz, nem feltétlenül megy el tüntetni a csoport érdekeit sértő intézkedések ellen, illetve nem feltétlenül lép közbe a csoport tagjait érő bullying szemtanújaként.

\subsection{A kutatás erősségei, limitációi és kitekintés}
Kutatásunk elsősorban azt a célt szolgálta, hogy a meleggek kialakított baráti kapcsolatok attitűdre és viselkedésre gyakorolt hatásait vizsgálja.
Kutatásunk legnagyobb előnye, hogy nem elképzelt helyzeteket és önbeszámolós kérdőívet használt a homofób diszkrimináció konfrontációjának vizsgálatára, s ezáltal a vizsgált személyek valós viselkedését volt képes felmérni. Úgy gondoljuk, hogy a jövőbeli kutatásoknak mindenképp figyelembe kell venniük a képzelt, illetve valós viselkedések közti diszkrepanciát, ha pontos képet kívánnak adni a témáról.
Mindazonáltal fontos rámutatnunk az explorációs kutatásunk korlátaira is. Először is a kutatásban résztvevő személye 67.9\% - a volt nő, s mindössze 26\% - a volt férfi. Emellett kitöltőink túlnyomó része (60\%) egyetemista volt. Ez azért befolyásolhatta az eredményeket, mert az egyetemisták politikai irányultsága nagyrészt liberális, s ezáltal alapvetően kevésbé előítéletesek. Az online mintavétel bár nagyszámú résztvevő gyűjtésére alkalmas, nem reprezentatív. Továbbá mivel a Bizalom Játékot csak asztali gépen vagy laptopon lehetett játszani (más elektronikai eszkösz nem támogatta a játékot) a vizsgálat csak korlátozott számú embert ért el, illetve a hosszú játékidő (30-40 perc) miatt többen abbahagyták a vizsgálatot útközben. 

\pagebreak
\section{Összegzés}

