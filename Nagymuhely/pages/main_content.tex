\section{Előszó}
\section{Bevezetés}
\begin{table}[]
	\centering
	\begin{tabular}{|l|l|l|l|}
		\hline
		& \textit{N}  & \textit{M}    & \textit{SD}   \\ \hline
		Homofóbia skála  & 50 & 2.42 & 0.96 \\ \hline
		Rokonszenv skála & 50 & 63   & 29   \\ \hline
		Barátok száma    & 50 & 2.38 & 0.87 \\ \hline
	\end{tabular}
	\caption{Leíró statisztika}
	\label{table:1}
\end{table}
\pagebreak
\section{Hipotézisek}
\section{Módszer}
\subsection{Résztvevők}
\subsection{Eljárás}
\subsubsection{Adatfelvétel}
\subsubsection{Megtévesztés}
\subsubsection{Kérdőív}
\subsubsection{Statisztikai módszerek}
\subsection{Mérőeszközök}
\subsubsection{Homofóbia skála}
\subsubsection{Rokonszenv skála}
\subsubsection{Barátok száma}
\section{Eredmények}
\section{Diszkusszió}
\subsection{Barátok száma és Homofóbia}
\subsection{Barátok száma és Rokonszenv}
\subsection{Barátok száma és Konfrontáció}
\subsection{Barátok száma és a Bizalom Játék}
\subsection{A kutatás erősségei, limitációi és kitekintés}

\section{Összegzés}

